
\chapter{Chain complexes}

\section{Chain complexes and their homology modules}


\begin{definition}
Let $R$ be a ring. A \emph{chain complex} of $R$-modules consists of
\begin{enumerate}
\item for every $i\in \bZ$ an $R$-module $M_i$,
\item for every $i\in \bZ$ an $R$-linear map $d_i\colon M_i \to M_{i-1}$
\end{enumerate}
such that for every $i$ the identity $d_i \circ d_{i+1} = 0 $ holds.
\end{definition}

We depict a chain complex as a diagram
\[
	\cdots \longto M_2 \overset{d_2}{\longto} M_1 \overset{d_1}{\longto}  M_0 
	\overset{d_0}{\longto} M_{-1} \longto \cdots 
\]
and often denote the chain complex by $M_\bullet$.

\begin{definition}
A \emph{morphism of chain complexes} $M_\bullet \to M'_\bullet$ consists of an $R$-module
homomorphism $f_i\colon M_i \to M_i'$ for every $i$, such that the resulting diagram
\[
\begin{tikzcd}
\cdots \arrow{r}
	& M_{2} \arrow{r}{d_{2}} \arrow{d}{f_{2}} 
	& M_{1} \arrow{r}{d_{1}} \arrow{d}{f_{1}} 
	& M_{0} \arrow{r} \arrow{d}{f_{0}} 
	& \cdots \\
\cdots \arrow{r}
	& M'_{2} \arrow{r}{d'_{2}} 
	& M'_{1} \arrow{r}{d'_{1}} 
	& M'_{0} \arrow{r} 
	& \cdots 
\end{tikzcd}
\]
commutes.  The resulting category of chain complexes of left $R$-modules is denoted 
${}_R\Ch$. The similarly-defined category of chain complexes of right $R$-modules is denoted $\Ch_R$.
\end{definition}

The condition $d_i \circ d_{i+1} = 0$ in a chain complex
 \[
	\cdots \longto M_{i+1} \overset{d_{i+1}}{\longto} M_i \overset{d_i}{\longto}  M_{i-1}
	\longto \cdots
\]
implies that $\im d_{i+1} \subset \ker d_i$ inside $M_i$. 

\begin{definition}
Let $M_\bullet$ be a chain complex of $R$-modules and $i$ an integer. The \emph{$i$-th homology module of $M_\bullet$} is the $R$-module
\[
	\rH_i(M_\bullet) := \frac{\ker d_i}{\im d_{i+1}}.
\]
\end{definition}

If $f\colon M_\bullet \to M'_\bullet$ is a morphism of chain complexes, then $f_i\colon M_i \to M_i'$ induces a morphism
\[
	\rH_i(f)  \colon \rH_i(M_\bullet) \to \rH_i(M'_\bullet)
\]
(see Exercise \ref{exc:induced-morphism-on-homology}). We obtain for every $i$ a functor
\[
	\rH_i \colon {}_R\Ch \to {}_R\Mod.
\]

These modules measure the failure of the sequence $M_\bullet$ to be exact, in the following sense.

\begin{lemma}
A chain complex
\[
	\cdots \longto M_2 \overset{d_2}{\longto} M_1 \overset{d_1}{\longto}  M_0 
	\overset{d_0}{\longto} M_{-1} \longto \cdots 
\]
is exact if and only if $\rH_i(M_\bullet)=0$ for all $i\in \bZ$. \qed
\end{lemma}

\section{The long exact sequence}

Let $M_\bullet$, $N_\bullet$ and $P_\bullet$ be chain complexes of $R$-modules. 
We say that a sequence of morphisms  
\[
	0 \longto M_\bullet \overset{\alpha}{\longto} N_\bullet 
	\overset{\beta}{\longto} P_\bullet \longto 0
\]
is a short exact sequence in ${}_R\Ch$ if it is termwise exact. In other words, 
if in the commutative diagram
\[
\begin{tikzcd}
	& 0  \arrow{d}
	& 0 \arrow{d}
	& 0 \arrow{d}
	&  \\
\cdots \arrow{r}
	& M_{i+1} \arrow{r}{} \arrow{d}{\alpha_{i+1}} 
	& M_{i} \arrow{r}{} \arrow{d}{\alpha_{i}} 
	& M_{i-1} \arrow{r} \arrow{d}{\beta_{i-1}} 
	& \cdots \\
\cdots \arrow{r}
	& N_{i+1} \arrow{r}{} \arrow{d}{\beta_{i+1}} 
	& N_{i} \arrow{r}{} \arrow{d}{\beta_{i}} 
	& N_{i-1} \arrow{r} \arrow{d}{\beta_{i-1}} 
	& \cdots \\
\cdots \arrow{r}
	& P_{i+1} \arrow{r}{} \arrow{d}
	& P_{i} \arrow{r}{}  \arrow{d}
	& P_{i-1} \arrow{r} \arrow{d}
	& \cdots \\
	& 0 
	& 0 
	& 0  
	&  
\end{tikzcd}
\]
all the columns are short exact sequences of $R$-modules. 

\begin{theorem}\label{thm:long-exact-sequence}
Let 
\[
	0 \longto M_\bullet \overset{\alpha}{\longto} N_\bullet 
	\overset{\beta}{\longto} P_\bullet \longto 0
\]
be a short exact sequence of chain complexes of $R$-modules. Then there exists
an exact sequence
\[
\begin{tikzcd}
 	\cdots \arrow{r}
 	& \rH_{i+1}(N_\bullet) \arrow{r}{\rH_{i+1}(\beta)}
 	& \rH_{i+1}(P_\bullet) \arrow[out=-5, in=175]{dll} \\
 	\rH_{i}(M_\bullet) \arrow{r}{\rH_{i}(\alpha)}
 	& \rH_{i}(N_\bullet) \arrow{r}{\rH_{i}(\beta)}
 	& \rH_{i}(P_\bullet) \arrow[out=-5, in=175]{dll} \\
 	\rH_{i-1}(M_\bullet) \arrow{r}{\rH_{i-1}(\alpha)}
 	& \rH_{i-1}(N_\bullet) \arrow{r}
 	& \cdots 
\end{tikzcd}
\]
of $R$-modules.
\end{theorem}

The resulting exact sequence of homology modules is called the \emph{long exact sequence of homology} associated to the short exact sequence of chain complexes.

\begin{proof}[Proof of Theorem \ref{thm:long-exact-sequence}]
We omit the proof, which is a tedious diagram chase in the style of the proof of the Snake Lemma (Theorem \ref{thm:snake-lemma}). In fact, the Snake Lemma is a special case of the theorem, obtained by assuming the $M_i$, $N_i$ and $P_i$ vanish for $i\not\in \{0,1\}$.
\end{proof}

\section{The homotopy category}

\begin{definition}
Let $R$ be a ring and let $f\colon M_\bullet \to M'_\bullet$ and $g\colon M_\bullet \to M'_\bullet$ be morphisms of chain complexes of $R$-modules. A \emph{homotopy} from $f$ to $g$ consists of
a collection of $R$-linear maps $h_i\colon M_i \to M'_{i+1}$ indexed by $i\in \bZ$, such that for every $i\in \bZ$ the identity
\[
	g_i - f_i =  d'_{i+1} h_i + h_{i-1} d_i
\]
holds in $\Hom_R( M_i, M'_i )$. We say that $f$ and $g$ are \emph{homotopic}, and write $f\sim g$, if there exists a homotopy from $f$ to $g$. 
\end{definition}

It is convenient to keep track of these maps in a diagram:
\[
\begin{tikzcd}[row sep=large, column sep=large]
\cdots \arrow{r}
	& M_{2} \arrow{r}{d_{2}} 
		\arrow[transform canvas={xshift=-.5ex},swap]{d}{f_{2}} 
		\arrow[transform canvas={xshift=.5ex}]{d}{g_{2}} 
		\arrow[dashed]{dl}{}
	& M_{1} \arrow{r}{d_{1}}  
		\arrow[transform canvas={xshift=-.5ex},swap]{d}{f_{1}} 
		\arrow[transform canvas={xshift=.5ex}]{d}{g_{1}} 
		\arrow[dashed]{dl}[description]{h_1}
	& M_{0} \arrow{r} 
		\arrow[transform canvas={xshift=-.5ex},swap]{d}{f_{0}} 
		\arrow[transform canvas={xshift=.5ex}]{d}{g_{0}} 
		\arrow[dashed]{dl}[description]{h_0}
	& \cdots \arrow[dashed]{dl}{} \\
\cdots \arrow{r}
	& M'_{2} \arrow[swap]{r}{d'_{2}} 
	& M'_{1} \arrow[swap]{r}{d'_{1}} 
	& M'_{0} \arrow[swap]{r} 
	& \cdots 
\end{tikzcd}
\]
but note that in the definition of homotopy it is \emph{not} required that the $h_i$'s commute with the horizontal maps $d$ in any way.

The equation defining homotopy can be remembered as
\[
	g-f=dh+hd,
\]
omitting the indices which can be reinserted in only one meaningful way. 

\begin{proposition}\label{prop:homotopy-equivalence-relation}
Let $R$ be a ring and $M_\bullet$, $M'_\bullet$ chain complex of $R$-modules. Then homotopy is an equivalence relation on the set
 $\Hom_{{}_R\Ch}(M_\bullet,M'_\bullet)$. \qed
\end{proposition}



\begin{proposition}\label{prop:homotopy-composition}
Let $R$ be a ring and $f,g \colon M_\bullet \to M'_\bullet$ homotopic morphisms of chain complexes of $R$-modules. Then
\begin{enumerate}
\item for any morphism $s\colon M'_\bullet \to N_\bullet$ in ${}_R\Ch$, the compositions $sf$ and $sg$ are homotopic;
\item for any morphism $t\colon N_\bullet \to M_\bullet$ in ${}_R\Ch$, the compositions $ft$ and $gt$ are homotopic.
\end{enumerate}
\end{proposition}


\begin{proof}
Let $(h_i)_i$ be a homotopy from $f$ to $g$. In the first case, one  verifies that $(s_{i+1}h_i)_i$ is a homotopy from $sf$ to $sg$, and in the second case,  that $(h_it_i)_i$ is a homotopy from $ft$ to $gt$. 
\end{proof}

\begin{definition}
The \emph{homotopy category of chain complexes of $R$-modules}, denoted ${}_R\Ho$,  is the
category with
\begin{enumerate}
\item $\ob {}_R\Ho := \ob {}_R\Ch$
\item $\Hom_{{}_R\Ho}(M_\bullet,N_\bullet) := \Hom_{{}_R\Ch}(M_\bullet, N_\bullet) / \sim$
\end{enumerate}
where composition and identity maps are inherited from composition and identity maps in ${}_R\Ch$.
\end{definition}

Proposition  \ref{prop:homotopy-composition} guarantees that composition in ${}_R\Ho$ is well-defined. 

\begin{proposition}\label{prop:homotopic-maps-agree-on-homology}
Let $f,g\colon M_\bullet \to M'_\bullet$ be homotopic maps in $\Ch_R$, and let $i$ be an integer. Then $\rH_i(f)=\rH_i(g)$ as maps $\rH_i(M_\bullet) \to \rH_i(M'_\bullet)$.
\end{proposition}

\begin{proof}
By definition, an element of $\rH_i(M_\bullet)$ is a coset
\[
	 \bar{x} := x + \im(d_{i+1})
\]
for some  $x\in \ker(d_i)$. We have
\begin{align*}
	\rH_i(f)(\bar{x}) - \rH_i(g)(\bar{x}) &=  f(x) - g(x) + \im (d'_{i+1})  \\
	&= d'_{i+1}(h_i(x)) + h_{i-1}(d_i(x))  +  \im (d'_{i+1}).
\end{align*}
Since $d'_{i+1}(h_i(x))  \in \im(d'_{i+1})$, the first term vanishes in $\rH_i(M'_\bullet)$. Since $x\in \ker(d_i)$, also the second term vanishes, and  $\rH_i(f)=\rH_i(g)$.
\end{proof}

A consequence of Proposition \ref{prop:homotopic-maps-agree-on-homology} is that the functors $\rH_i$ on ${}_R\Ch$ induce  functors
\[
	\rH_i\colon {}_R\Ho \to {}_R\Mod,\, M_\bullet \mapsto \rH_i(M_\bullet)
\]
on the homotopy category of chain complexes.

\begin{remark}As the terminology suggests, there is a close relationship with the notions of homotopic maps and homology groups in algebraic topology. Homology groups of topological spaces are usually defined in terms of the functor
\[
	C\colon \Top \to {}_\bZ\Ch,\, X \mapsto C_\bullet(X)
\]
that maps a space $X$ to the chain complex of \emph{singular chains}. One then defines the
$i$-th homology group of $X$ by
\[
	\rH_i(X,\bZ) := \rH_i(C_\bullet(X))
\]
and obtains functors $\rH_i\colon \Top \to \Ab$. 

If $f,g\colon X\to Y$ are homotopic continuous maps, then their induced maps $C_\bullet(X) \to C_\bullet(Y)$ are homotopic maps of chain complexes, and hence their induced maps $\rH_i(X,\bZ) \to \rH_i(Y,\bZ)$ coincide.
\end{remark}

\begin{remark}
Chain complexes form a mathematical context in which three layers play a role: objects (chain complexes), morphisms (morphisms of chain complexes), and maps between morphisms (homotopies). See Remark \ref{rmk:2-cat} for two other such contexts: topological spaces (spaces, continuous maps, homotopies), and categories (categories, functors, morphisms of functors).
\end{remark}


\newpage
\section*{Exercises}

\begin{exercise}[Functoriality of homology]\label{exc:induced-morphism-on-homology}
Let $f\colon M_\bullet \to M'_\bullet$ be a morphism of chain complexes of $R$-modules. 
\begin{enumerate}
\item Show  $f_i(\ker d_i) \subset \ker d'_i$;
\item Show $f_i(\im d_{i+1}) \subset \im d'_{i+1}$;
\item Conclude that $f_i$ induces an $R$-linear map $\rH_i(M) \to \rH_i(M')$.
\end{enumerate}
\end{exercise}

\begin{exercise}
Let 
\[
	\ldots \to M_2 \to M_1 \to M_0 \to N \to 0
\]
be an exact sequence of $R$-modules. Show that the complex
\[
	M_\bullet = \big( \ldots \to M_2 \to M_1 \to M_0  \to 0 \to \cdots \big)
\]
satisfies $\rH_0(M_\bullet)\cong N$ and $\rH_n(M_\bullet)=0$ for all $n\neq 0$.
\end{exercise}


\begin{exercise}\label{exc:an-isomorphism-in-the-homotopy-category}
Let $R$ be a non-zero ring. Show that the  two chain complexes
\[
\begin{tikzcd}[row sep=small]
\cdots \arrow{r}
	& 0 \arrow{r}
	& 0 \arrow{r}
	& 0 \arrow{r}
	& 0 \arrow{r}
	&\cdots \\ 
\cdots \arrow{r}
	& 0 \arrow{r}
	& R \arrow{r}{\id}
	& R \arrow{r}
	& 0 \arrow{r}
	&\cdots
\end{tikzcd}
\]
are isomorphic in ${}_R\Ho$.
\end{exercise}

\begin{exercise}Let $n>1$ and let $f$ be the morphism in ${}_\bZ\Ch$ given by the diagram
\[
\begin{tikzcd}
\cdots \arrow{r}
	& 0 \arrow{r} \arrow{d}
	& \bZ \arrow{r}{n} \arrow{d}
	& \bZ \arrow{r} \arrow{d}{\pi}
	& 0 \arrow{r} \arrow{d}
	&\cdots \\ 
\cdots \arrow{r}
	& 0 \arrow{r}
	& 0 \arrow{r}
	& \bZ/n\bZ \arrow{r}
	& 0 \arrow{r}
	&\cdots
\end{tikzcd}
\]
with $\pi$ the canonical map. Show that $\rH_i(f)$ is an isomorphism for all $i$, but that $f$ is not an isomorphism in ${}_\bZ\Ho$.
\end{exercise}

\begin{exercise}Let $R$ and $S$ be rings. A functor $F\colon {}_R\Mod\to {}_S\Mod$ is
called \emph{additive} if for all $R$-modules $M$, $N$ the map
\[
	F\colon \Hom_R(M,N) \to \Hom_S(FM,FN)
\]
is a homomorphism of abelian groups.  Let $A$ be an $(R,S)$-bimodule. Show that the functor
\[
	{}_R\Mod \to {}_S\Mod,\, M \mapsto \Hom_R(A,M)
\]
is additive.
\end{exercise}

\begin{exercise}
Show that an additive functor $F\colon {}_R\Mod \to {}_S\Mod$ induces functors
${}_R\Ch\to {}_S\Ch$ and ${}_R\Ho\to {}_S\Ho$.
\end{exercise}

\begin{exercise}\label{exc:contravariant-hom-on-chain-complexes}
Let $R$ and $S$ be rings and $A$ an $(R,S)$-bimodule. For a chain complex $M_\bullet$ in ${}_R\Ch$ define a chain complex $M'_\bullet$ in $\Ch_S$
by
\begin{enumerate}
\item $M'_i := \Hom_R(M_{-i},A)$ 
\item $d_i\colon M'_i \to M'_{i-1}$ the map induced from $d_{1-i} \colon M_{1-i} \to M_{-i}$
\end{enumerate}
Verify that $M'_\bullet$ is a chain complex of right $S$-modules, and that the operation $M_\bullet \mapsto M'_\bullet$ defines functors ${}_R\Ch^\opp \to \Ch_S$ and ${}_R\Ho^\opp \to \Ho_S$.
\end{exercise}

\begin{exercise}[$\star$]
Let $G$ be a group. Consider the abelian groups $C_n(G) := \bZ^{(G^n)}$. To ease the notation, denote the basis vector $e_{(g_1,\ldots,g_n)}$ by $[g_1,\ldots,g_n] \in C_n(G)$. Consider the morphisms
\[
	d_n \colon C_n(G) \to C_{n-1}(G) 
\]
defined on the basis of $C_n(G)$ by
\begin{align*}
	[g_1,\ldots, g_n] \mapsto &[g_2, g_3, \ldots, g_n] \\
	&+ \sum_{i=1}^{n-1} (-1)^i [g_1,\ldots, g_{i-1}, g_i g_{i+1}, g_{i+2}, \ldots, g_n] \\
	& + (-1)^n [ g_1,\ldots, g_{n-1} ].
\end{align*}
Set $C_n(G)=0$ for $n<0$.
\begin{enumerate}
\item Show that $C_\bullet(G)$ is a chain complex of abelian groups.
\end{enumerate}
The homology groups
$\rH_n(C_\bullet(G))$ are called the \emph{homology groups} of $G$, and are denoted $\rH_n(G,\bZ)$.
\begin{enumerate}
\item[(2)] Show that $\rH_0(G,\bZ)\cong \bZ$;
\item[(3)] Show that $\rH_1(G,\bZ)\cong G^\ab$.
\end{enumerate} 
\end{exercise}


\begin{exercise}[$\star$]
Let $K$ be a field. Let $\prod_{i\in \bZ} \Vec_K$ be the category whose objects are sequences $(V_i)_{i\in \bZ}$ of $K$-vector spaces, and whose morphisms are sequences $(f_i)_{i\in \bZ}$ of $K$-linear maps. Show that the functor
\[
	\Ho_K \to \prod_{i\in \bZ} \Vec_K,\, V_\bullet \mapsto \left( \rH_i(V_\bullet) \right)_{i}
\]
is an equivalence of categories.
\end{exercise}


