
\chapter{Tensor product}


\section{Tensor product of a right and a left module}

\begin{definition}
Let $R$ be a  ring, $M$ a right $R$-module, and $N$ a left $R$-module. Let $A$ be an abelian group. A map
\[
	f\colon M\times N \to A
\]
is said to be $R$-bilinear if for all $x, x_1,x_2\in M$, $y, y_1,y_2\in N$ and $r\in R$ the following hold:
\begin{enumerate}
\item $f(x_1+x_2,y)=f(x_1,y) + f(x_2,y)$
\item $f(x,y_1+y_2) = f(x,y_1) + f(x,y_2)$
\item $f(xr,y)=f(x,ry)$ 
\end{enumerate}
\end{definition}

\begin{remark}
These conditions imply that moreover
\begin{enumerate}
\item [(4)] $f(x,0)=0$, and
\item[(5)] $f(0,y) =0$
\end{enumerate}
hold for all $x\in M$ and $y\in N$. See Exercise~\ref{exc:bilinear-zero}.
\end{remark}


Note that if $f\colon M\times N\to A$ is bilinear, and $h\colon A\to B$ is a homomorphism of abelian groups, then the composition $hf\colon M\times N \to B$ is also a bilinear map. The following theorem states that there is a `universal' bilinear map, from which all others can be obtained by a unique composition with a homomorphism of abelian groups.

\begin{theorem}\label{thm:existence-of-tensor-product}
Let $R$ be a ring, $M$ a right $R$-module and $N$ a left $R$-module. Then there exists an abelian group $T$ and an $R$-bilinear map
\[
	g\colon M\times N \to T
\]
such that for every abelian group $A$ and every $R$-bilinear map $f\colon M\times N \to A$ there is a unique group homomorphism
$h\colon T \to A$ with $f=hg$:
\[
\begin{tikzcd}
M\times N\arrow{r}{g} \arrow{d}{f} & T \arrow[dashed]{ld}{h} \\
A
\end{tikzcd}
\]
Moreover, the pair $(T,g)$ is unique up to unique isomorphism in the following sense: if both $(T_1,g_1)$ and $(T_2,g_2)$ satisfy the above property, then there is a unique isomorphism $h\colon T_1 \to T_2$ such that $g_2=hg_1$.
\end{theorem}


\begin{definition}We will call the abelian group $T$ (unique up to unique isomorphism) the \emph{tensor product} of $M$ and $N$, and denote it by $M\otimes_R N := T $. For $x\in M$ and $y\in N$ we denote the image of $(x,y)$ in $M\otimes_R N$ by $x\otimes y := g(x,y)$. 
\end{definition}

% TODO: verplaatsen naar na het bewijs, korte discussie over wat 'het tensor product is' (iedereen mag eigen keuze maken, bvb die uit het bewijs, zolang maar aan universele eigenschap is voldaan. Vergelijking met reele getallen: je hoeft niet te weten of iemand anders' reele getallen Dedekindsneden of cauchy-rij-klassen zijn.

\begin{proof}[Proof of Theorem \ref{thm:existence-of-tensor-product}]
\emph{Uniqueness}. This is purely formal:  everything defined by a universal property is unique up to unique isomorphism, by an argument  that is basically the same as the proof of Proposition \ref{prop:final-object-uniquely-unique}: 
Assume $(T_1,g_1)$ and $(T_2,g_2)$ both satisfy the required property. Since $g_1\colon M\times N \to T_1$ is bilinear there is a unique map $h_1\colon T_2 \to T_1$ with $g_1=h_1g_2$. Reversing the roles of $T_1$ and $T_2$, we get a unique map $h_2\colon T_1\to T_2$. Moreover, both the compositions $h_1h_2$ and $h_2h_1$ must be the identity, so we conclude that $h_1$ is an isomorphism.

\emph{Existence}. This part is certainly not formal! The proof is a bit messy, but in a way very natural: we just construct an abelian group $T$ (and a map $g$) with all the desired properties built-in. 

Let $F:=\bZ^{(M\times N)}$ be the free $\bZ$-module on the set $M\times N$. Given an element $(x,y) \in M\times N$, we denote by $e_{(x,y)} \in \bZ^{(M\times N)}$ the corresponding basis vector, see \ref{ex:free-module}. There is a canonical map of sets
\[
	M\times N \to F,\, (x,y) \mapsto e_{(x,y)}.
\]
This map has no reason to be bilinear. We will force it to become bilinear by dividing out the necessary relations. Let $G\subset F$ be the subgroup generated by the elements
\begin{gather*}
	e_{(x_1+x_2,y)}-e_{(x_1,y)}-e_{(x_2,y)},\\
	e_{(x,y_1+y_2)}-e_{(x,y_1)}-e_{(x,y_2)},\\
	e_{(xr,y)}-e_{(x,ry)},
\end{gather*}
for all $x_1,x_2,x\in M$ and $y_1,y_2,y\in N$ and $r\in R$. Let $T$ be the quotient group $F/G$, and consider the composition
\[
\begin{tikzcd}
M\times N \arrow{r} \arrow{rd}{g} & F \arrow{d} \\
& T
\end{tikzcd}
\]
Then the map $g$ is bilinear by construction.

We now show that $(T,g)$ is a tensor product. Let $f\colon M\times N \to A$ be a billinear map. Then there is a unique homomorphism
$f' \colon F \to A$, which sends the basis vector $e_{(x,y)}$ to $f(x,y)$, see Proposition \ref{prop:universal-property-free-module}.
Since $f$ is bilinear, we have that $f'$ vanishes on all the generators of $G$, and therefore that $f'(G)=0$. Hence $f'$ induces a homomorphism $h\colon T \to A$ with $f=hg$. To see that a map $h$ with this property is unique, note that $T$ is generated by the images of the elements $(x,y)\in M\times N$, and that $h$ must send the image of $(x,y)$ to $f(x,y)$.
\end{proof}


\begin{remark}
In practice it is often easier not to use the actual construction of the tensor product in the proof of Theorem \ref{thm:existence-of-tensor-product}, but only the defining universal property in the statement of Theorem  \ref{thm:existence-of-tensor-product}, together with the fact that the tensor product is generated by the elements of the form $x\otimes y$ with $x\in M$ and $y\in N$.
\end{remark}

\begin{remark}
Elements of $M\otimes_R N$ are finite sums of elements of the form $x\otimes y$, but these are not independent. In fact, 
the map
\[
	M\times N \to M\otimes_R N,\, (x,y) \mapsto x\otimes y
\]
is $R$-bilinear (by definition of the tensor product), so that for all  $x, x_1,x_2\in M$, $y,y_1,y_2\in N$ and $r\in R$  the identities
\begin{gather*}
	(x_1+x_2) \otimes y = (x_1 \otimes y) + (x_2 \otimes y) \\
	x\otimes (y_1+y_2) = (x \otimes y_1) + (x \otimes y_2) \\
	(xr) \otimes y = x \otimes (ry) \\
	x \otimes 0 = 0 \\
	0 \otimes y = 0 
\end{gather*}
hold in $M\otimes_R N$. 
\end{remark}


We end this section with a few examples of tensor products.

\begin{example}
Let $M$ be a right $R$-module, then we claim $M \otimes_R \{0\} \cong \{0\}$. Indeed, the tensor product is generated as an abelian group by the elements $x\otimes 0$, but these all are equal to $0$ in $M\otimes_R \{0\}$. 
\end{example}


\begin{example}\label{exa:tensor-with-R}
We claim that for any left $R$-module $M$ there is a unique isomorphism 
\[
	f\colon R\otimes_R M \isomto M
\]
satisfying
\[
	r\otimes x \mapsto rx.
\]
Indeed, the map $R\times M \to M,\, (r,x) \mapsto rx$
is $R$-bilinear, and hence induces an $R$-linear homomorphism
$f\colon R\otimes_R M \to M$ with $f(r\otimes x)=rx$. Conversely, the map
$M \to R\otimes_R M$ given by $x \mapsto 1\otimes x$
is $R$-linear, and is a two-sided inverse to the map $f$.

Of course, in the same way one can produce an isomorphism
\[
	 M\otimes_R R \isomto M,\, x \otimes r \mapsto xr
\]
for every \emph{right} $R$-module $M$.
\end{example}

\begin{example}
The tensor product of two non-zero modules can be zero. For example, we have
\[
	(\bZ/2\bZ) \otimes_\bZ (\bZ/3\bZ) = 0.
\]
Indeed, the tensor product is generated by elements of the form $x\otimes y$ with $x\in \bZ/2\bZ$ and $y\in \bZ/3\bZ$. But since
$3x=x$ for all $x\in \bZ/2\bZ$ we find 
\[
	x\otimes y = (3x) \otimes y  = x \otimes (3y) = x\otimes 0 =  0.
\]
Similarly we have $(\bZ/n\bZ)\otimes_\bZ (\bZ/m\bZ)=0$ whenever $n$ and $m$ are co-prime. See also Exercise \ref{exc:tensor-Z-mod-nZ}.
\end{example}

% --- updated towards noncommutative setting up to here

\section{Tensor products and bimodules}

In many cases, the tensor product of two modules is not just an abelian group, but itself again a module. 

\begin{definition}
Let $R$ and $S$ be rings.
An \emph{$(R,S)$-bimodule} is an abelian group $M$ equipped with operations
\[
	R\times M \to M,\, (r,x) \mapsto rx
\]
and 
\[
	M \times S \to M,\, (x,s) \mapsto xs
\]
such that 
\begin{enumerate}
\item[(B1)] the first operation makes $M$ into a left $R$-module 
\item[(B2)] the second operation makes $M$ into a right $S$-module
\item[(B3)] for all $r\in R$, $s\in S$ and $x\in M$ the identity $r(xs)=(rx)s$ holds in $M$.
\end{enumerate}
A map  $f\colon M \to N$  is called a \emph{morphism} of $(R,S)$-bimodules if it is both a morphism
of left $R$-modules and a morphism of right $S$-modules. We denote the category of $(R,S)$-bimodules by ${}_R\Mod_S$.
\end{definition}

Because of axiom (B3), we can simply write $rxs$ for $r(xs)=(rx)s$, and we will frequently describe the structure of an $(R,S)$-bimodule by the map
\[
	R\times M \times S \to M,\, (r,x,s) \mapsto rxs,
\]
simultaneously encoding the left and right module structures.

\begin{example}An abelian group $A$ has a unique structure of $(\bZ,\bZ)$-bimodule, which is given by 
$nam := (nm)a$ for all $n,m\in \bZ$ and $a\in A$. It follows that a $(\bZ,\bZ)$-bimodule is the same thing as an abelian group. Similarly, an $(R,\bZ)$-bimodule is the same as a left $R$-module, and a $(\bZ,R)$-bimodule is the same as a right $R$-module.
\end{example}

\begin{example}Let $R$ be a \emph{commutative} ring and $M$ an $R$-module. Then $M$ also is an $(R,R)$-bimodule by setting
\[
	rxs := rsx
\]
for all $r,s\in R$ and $x\in M$.
\end{example}

\begin{example}\label{exa:Rn-bimodule}
Let $R$ be a ring and $n$ a non-negative integer. Then $R^n$ becomes an $(R,R)$-bimodule with
\[
	r(x_1,\ldots, x_n)s := (rx_1s, \ldots, rx_ns).
\]
\end{example}

\begin{example}Let $R$ be a commutative ring and $n$ and $m$ non-negative integers. Then the additive group $M=\Mat_{m,n}(R)$ of $n$ by $m$ matrices is a $(\Mat_n(R),\Mat_m(R))$-bimodule where for $A\in \Mat_{n}(R)$, $B\in \Mat_{m}(R)$ and $X\in M$ the element $AXB$ of $M$ is defined by matrix multiplication.
\end{example}


\begin{example}\label{exa:Hom-bimodule}
Let $M$ be a left $S$-module and $N$ be a left $R$-module. Consider the group $\Hom(M,N)$ of homomorphisms of abelian groups. This group carries a natural structure of $(R,S)$-bimodule with
\[
	rfs := \big[ M \to N,\, x \mapsto rf(sx) \big]
\] 
for all $r\in R$, $f\in \Hom(M,N)$ and $s\in S$.
\end{example}

\begin{proposition}
Let $R$, $S$, and $T$ be rings. Let $M$ be an $(R,S)$-bimodule, and let $N$ be an $(S,T)$-bimodule. Then the tensor product $M\otimes_S N$ has a unique structure of an $(R,T)$-bimodule satisfying
\[
	r(x\otimes y)t = (rx) \otimes (yt)
\]
for all $r\in R$, $x\in M$, $y\in N$ and  $t\in T$. 
\end{proposition}

If $R$ is a commutative ring then any $R$-module is canonically an $(R,R)$-bimodule, and we find that
the tensor product of two $R$-modules over $R$ is naturally an $R$-module. When dealing with commutative rings, there is no essential distinction between left and right modules, and we will usually treat the tensor product as an operation that produces a left $R$-module out of two left $R$-modules.





\section{Tensor product as a functor}

Let $R$ be a ring and let $f\colon M_1\to M_2$ be a morphism of right $R$-modules, and let $g\colon N_1\to N_2$ be a morphism of left $R$-modules. Then the map
\[
	M_1\times N_1 \to M_2\otimes_R N_2,\, (x,y) \mapsto f(x)\otimes g(y)
\]
is $R$-bilinear. Hence, by the universal property of the tensor product, there exists a unique homomorphism of abelian groups
\[
	f\otimes g\colon M_1\otimes_R N_1 \to M_2\otimes_R N_2
\]
that maps an element $x\otimes y$ to $f(x)\otimes g(y)$. This upgrades  the tensor product from being just a construction on pairs of modules to a functor
\[
	-\otimes_R-\colon \Mod_R \times {}_R\Mod \to \Ab.
\]
Similarly, if $R$, $S$ and $T$ are rings, then we have a functor
\[
	-\otimes_S- \colon {}_R\Mod_S \times {}_S\Mod_T \to {}_R\Mod_T.
\]
	
\begin{proposition}\label{prop:tensor-right-exact}
Let $R$ be a ring and let $N$ be a left $R$-module. If 
\begin{equation}\label{eq:pre-tensor-right-exact}
	M_1 \overset{f}{\longto} M_2 \overset{g}{\longto} M_3 \longto 0
\end{equation}
is an exact sequence of right $R$-modules, then the induced sequence
\begin{equation}\label{eq:tensor-right-exact}
	M_1\otimes_R N \overset{f\otimes \id}{\longto}
	M_2\otimes_R N \overset{g\otimes \id}{\longto}
	M_3\otimes_R N \longto 0
\end{equation}
of abelian groups is exact. Similarly, if
\[
	N_1 \longto N_2 \longto N_3 \longto 0
\]
is an exact sequence of left $R$-modules, and $M$ a right $R$-module, then the induced sequence
\[
	M \otimes_R N_1 \longto M \otimes_R N_2 \longto M \otimes_R N_3 \longto 0
\]
is exact.
\end{proposition}

The functor $-\otimes_R N$ in general does \emph{not} preserve short exact sequences: even if
\[
	0\to M_1 \to M_2 \to M_3 \to 0
\]
is a short exact sequence, then we only obtain a partial exact sequence
\[
	 M_1\otimes_R N \to M_2\otimes_R N \to M_3\otimes_R N \to 0.
\]
See Exercise \ref{exc:tensor-not-exact}.

\begin{proof}[Proof of Proposition \ref{prop:tensor-right-exact}]
For all $x\in M_1$ and $y\in N$ we have $gf(x)\otimes y=0$, hence $\im (f\otimes \id) \subset \ker (g\otimes \id)$. In particular, we have an induced map
\begin{equation}\label{eq:tensor-exact-induced}
	\Phi\colon \frac{M_2\otimes_R N}{\im (f\otimes \id)} \longto M_3 \otimes_R N.
\end{equation}
To show that the sequence (\ref{eq:tensor-right-exact}) is exact, it suffices to show that the above map is an isomorphism.  We will verify this by constructing an inverse map.

For every $x\in M_3$ choose an $x'\in M_2$ with $g(x')=x$ (note that $g$ is surjective). We define a map
\[
	  M_3 \times N \to \frac{M_2\otimes_R N}{\im (f\otimes \id)},\,  (x,y) \mapsto x'\otimes y.
\]
This is well-defined, since if $x''$ is another element with $g(x'')=x$, then
\[
	(x''\otimes y) - (x'\otimes y) = (x''-x') \otimes y \in \im (f\otimes \id),
\]
using the exactness of the original sequence (\ref{eq:pre-tensor-right-exact}).
Moreover, the map is bilinear, so it induces a homomorphism 
\[
	M_3\otimes_R N \to \frac{M_2\otimes_R N}{\im (f\otimes \id)}
\]
and one verifies that this is a two-sided inverse to the map $\Phi$.
\end{proof}

\begin{example}
Proposition \ref{prop:tensor-right-exact} can be a powerful tool in computing tensor products. As an example, let us use it to compute $M\otimes_\bZ (\bZ/n\bZ)$ for any $\bZ$-module $M$. We have an exact sequence
\[
	\bZ \overset{n}{\longto} \bZ \longto \bZ/n\bZ \longto 0
\]
of $\bZ$-modules, which induces an exact sequence
\[
	\bZ\otimes_\bZ M \overset{n\otimes \id}{\longto} \bZ \otimes_\bZ M
	 \longto \bZ/n\bZ \otimes_\bZ M \longto 0.
\]
We have $\bZ\otimes_\bZ M \cong M$ (see Example \ref{exa:tensor-with-R}), and the above exact sequence is isomorphic with the exact sequence
\[
	M \overset{n}{\longto} M \longto \bZ/n\bZ \otimes_\bZ M \longto 0,
\]
from which we conclude that $\bZ/n\bZ \otimes_\bZ M$ is isomorphic with $M/nM$.
\end{example}


\section{The adjunction}

Let $R$ and $S$ be rings. If $N$ is an $(R,S)$-bimodule, and $P$ a right $S$-module, then $\Hom_S(N,P)$ is naturally a right $R$-module, with the action of $R$ defined by:
\[
	fr\colon N \to P,\, x \mapsto f(rx).
\]
See Exercise \ref{exc:hom-bimodule}.


\begin{theorem}[Tensor-Hom adjunction]\label{thm:tensor-hom-adjunction}
Let $R$ and $S$ be rings. Let $M$ be a right $R$-module, $N$ an $(R,S)$-bimodule, and $P$ be a right $S$-module. Then the map of abelian groups
\[
	\Hom_S(M\otimes_R N, P) \to \Hom_R(M,\Hom_S(N,P))
\]
given by
\[
	f \mapsto \big( x \mapsto ( y \mapsto f(x\otimes y) )\big)
\]
is an isomorphism.
\end{theorem}

\begin{remark}
The $\Hom_R$ and $\Hom_S$ in the theorem denote the set of homomorphisms in the categories of \emph{right} $R$- and $S$-modules. 
\end{remark}


\begin{proof}[Proof of Theorem \ref{thm:tensor-hom-adjunction}]
Given an $R$-linear map $f\colon M\to \Hom_S(N,P)$ we obtain a map
\[
	M\times N \to P,\,  (x,y) \mapsto f(x)(y)
\]
which is $R$-bilinear, hence it defines a homomorphism
\[
	f'\colon M\otimes_R N \to P
\]
with the property that it maps $x\otimes y$ to $f(x)(y)$. This map is $S$-linear. This construction defines a homomorphism 
\[
	\Hom_R(M,\Hom_S(N,P)) \to \Hom_S(M\otimes_R N, P),\, f\mapsto f'.
\]
We leave it to the reader to verify that this is a two-sided inverse to the map in the theorem.
\end{proof}

\begin{remark}\label{thm:alternate-adjunction}
There is completely analogous theorem about tensoring on the left with a fixed $(S,R)$-bimodule $N$. It gives an isomorphism
\[
	\Hom_S(N\otimes_R M, P) \to \Hom_R(M, \Hom_S(N,P))
\]
where $M$ is a left $R$-module and $P$ a left $S$-module. In this version, $\Hom_S$ and $\Hom_R$ denote the sets of homomorphisms in the categories of \emph{left} $S$- and $R$-modules.
\end{remark}

%\section{Restriction and extension of scalars}\label{sec:restriction-extension-of-scalars}
%
%Let $f\colon R\to S$ be a morphism of rings.
%If $M$ is an $S$-module, then $M$ also has the structure of an $R$-module, by $rx := f(r)x$. We obtain a functor
%\[
%	{}_S\Mod \to {}_R\Mod,\, M \mapsto M
%\]
%which is called \emph{restriction of scalars}. 
%
%\begin{example}
%If $f\colon \bR \to \bC$ is the inclusion, then the corresponding restriction of scalars functor maps an $n$-dimensional complex vector space $V$, to the $2n$-dimensional real vector space $V$.
%\end{example}
%
%\begin{example}
%If $f\colon \bZ\to R$ is the canonical ring morphism from $\bZ$ to a commutative ring, then the restriction of scalars functor coincides with the forgetful functor
%\[
%	{}_R\Mod \to \Ab,\, M \mapsto M
%\]
%which maps a module to the underlying abelian group.
%\end{example}
%
%There is also a construction in the other direction.  Let $M$ be a (left) $R$-module. Note that $S$ is an $(S,R)$-bimodule where $s\in S$ acts by left multiplication on $S$, and $r\in R$ by right multiplication with $f(r)$. Hence the tensor product $S\otimes_R M$ is a left $S$-module and we obtain a functor
%\[
%	{}_R\Mod \to {}_S\Mod,\, M \mapsto S\otimes_R M
%\]
%called \emph{extension of scalars}. 
%
%\begin{example}
%If $M$ is free with basis $x_1,\ldots, x_n$, then $S\otimes_R M$ is free with basis $1\otimes x_1,\ldots,1\otimes x_n$. That is, every element $y\in S\otimes_R M$ can be uniquely written as
%\[
%	y = s_1\otimes x_1 + \cdots + s_n \otimes x_n
%\]
%with $s_i \in S$. For example, if $f\colon \bR \to \bC$ is the inclusion and $V=\bR^n$ then we have $\bC\otimes_\bR V \cong \bC^n$ as complex vector spaces.
%\end{example}
%
%
%
%\begin{proposition}[Universal property of extension of scalars]\label{prop:universal-property-extension-of-scalars}
%Let $f\colon R\to S$ be a ring homomorphism. Let $M$ be a left $R$-module and let $N$ be a left $S$-module. Let $\varphi\colon M\to N$ be an $R$-linear map. Then there exists a unique $S$-linear map $\tilde\varphi\colon S\otimes_R M \to N$ such that the triangle
%\[
%\begin{tikzcd}
%S\otimes_R M \arrow[dashed]{r}{\tilde\varphi} & N \\
%M \arrow{u}{x\mapsto 1\otimes x} \arrow[swap]{ru}{\varphi}
%\end{tikzcd}
%\]
%commutes.
%\end{proposition}
%
%\begin{proof}We first prove uniqueness. For all $x\in M$ we must have
%\[
%	\tilde\varphi(s\otimes x) = s\tilde\varphi(1\otimes x) = s\varphi(x),
%\]
%and since $S\otimes_R M$ is generated by elements $s\otimes x$ we see that there can be at most one $\tilde\varphi$. To show existence, note that
%\[
%	S\times M \to N,\, (s,x) \mapsto s\varphi(x)
%\]
%is $R$-bilinear and hence defines a map
%\[
%	\tilde\varphi\colon S\otimes_R M \to N
%\]
%satisfying $\tilde\varphi(s\otimes x) = s\varphi(x)$. For all $s,s'\in S$ and $x\in M$ we have
%\[
%	\tilde\varphi(s'(s\otimes x)) = \tilde\varphi(s's\otimes x) = s's\varphi(x) = s'(s\varphi(x)),
%\]
%which shows that $\tilde\varphi$ is indeed $S$-linear. 
%\end{proof}
%
%A useful reformulation of the above proposition is the following corollary, which can also be seen as a special case of Theorem \ref{thm:tensor-hom-adjunction}, using the canonical isomorphism $\Hom_S(S,N)=N$.
%
%\begin{corollary}[Adjunction between extension and restriction of scalars]
%Let $f\colon R\to S$ be a ring homomorphism. Let $M$ be an $R$-module and let $N$ be an $S$-module. Then the map
%\[
%	\Hom_S(S\otimes_R M,N) \to \Hom_R(M, N),\,
%	f \mapsto \left( x \mapsto f(1\otimes x) \right)
%\]
%is a bijection.\qed
%\end{corollary}
%
%


\newpage
\section*{Exercises}

% TODO: if R commutative, show that bilinear map to A implies A is R-module, and is bilinear in the usual sense.  FALSE! Can postcompose with any additive map A-->A'.

\begin{exercise}\label{exc:bilinear-zero}
Prove that if $f\colon M\times N \to A$ is $R$-bilinear, then $f(x,0)=0$ for all $x\in M$ and $f(0,y)=0$ for all $y\in N$.
\end{exercise}


\begin{exercise}Let $R$ be a ring, let $M$ be a left $R$-module and let $n$ be a non-negative integer.  
\begin{enumerate}
\item Show that the map $R^n \times M \to M^n$ given by 
\[((r_1,\ldots,r_n),x) \mapsto (r_1x,\ldots, r_nx)\]
is $R$-bilinear.
\item Show that the above map is universal, and conclude that $R^n \otimes_R M \cong M^n$.
\end{enumerate}
\end{exercise}
%
%
%\begin{exercise}Let $R$ be a commutative ring. Show that $R^n \otimes_R R^m$ is isomorphic to $R^{nm}$ as $R$-module.
%\end{exercise}
%
\begin{exercise} \label{exc:tensor-Z-mod-nZ}
Let $n$ and $m$ be positive integers with greatest common divisor $d$. Show that 
\begin{enumerate}
\item $(\bZ/n\bZ) \otimes_\bZ (\bZ/m\bZ) \cong \bZ/d\bZ$;
\item $\bQ\otimes_\bZ (\bZ/m\bZ) \cong 0$;
\item $\bQ \otimes_\bZ \bQ \cong \bQ$.
\end{enumerate}
\end{exercise}

\begin{exercise}
Let $K$ be a field and $R:=\Mat_n(K)$ the ring of $n$ by $n$ matrices. Let $M=K^n$ be the right $R$-module of length $n$ row vectors, and $N=K^n$ the left $R$-module of length $n$ column vectors. Show that the abelian group $M\otimes_R N$ is isomorphic to $K$.
\end{exercise}


\begin{exercise}
Let $R$ be a commutative ring and $f,g\in R$. Show that
\[
	R/(f) \otimes_R R/(g) \cong R/(f,g)
\]
as $R$-modules.
\end{exercise}

\begin{exercise}
Let $R$ be a ring, let $I\subset R$ be a (two-sided) ideal and let $M$ be a left $R$-module. Show that 
$(R/I) \otimes_R M$ is isomorphic to $M/IM$.
\end{exercise}
\begin{exercise}
Let $R$ be a commutative ring. Show that the functors ${}_R\Mod \times {}_R\Mod \to {}_R\Mod$ given by $(M,N) \mapsto M\otimes_R N$ and $(M,N) \mapsto (N\otimes_R M)$ are isomorphic. 
\end{exercise}



\begin{exercise}Verify that the bimodule in Example \ref{exa:Hom-bimodule} indeed satisfies the bimodule axioms.
\end{exercise}

\begin{exercise}\label{exc:hom-bimodule}
Let $R$ and $S$ be rings. Let $M$ be a left $R$-module and $N$ an $(R,S)$-bimodule. Show that $\Hom_R(M,N)$ is naturally a right $S$-module and that $\Hom_R(N,M)$ is naturally a left $S$-module.
\end{exercise}


\begin{exercise}
Let $R$ be a ring and let $M_1$, $M_2$ be right $R$-modules and $N$ a left $R$-module. Show that there is a unique isomorphism
\[
	 (M_1\oplus M_2) \otimes_R N \to (M_1\otimes_R N) \oplus (M_2 \otimes_R N)
\]
such that 
\[
	(x,y)\otimes z \mapsto (x\otimes z, y\otimes z)
\]
for all $x\in M_1$, $y\in M_2$ and $z\in N$.
\end{exercise}


\begin{exercise}
Let $R$ and $S$ be rings, let $M_1$ be a right $R$-module, $M_2$ an $(R,S)$-bimodule, and $M_3$ a left $S$-module. Show that there is a unique isomorphism
\[
	 (M_1\otimes_R M_2) \otimes_S M_3 \to M_1 \otimes_R (M_2 \otimes_S M_3)
\]
such that 
\[
	(x\otimes y)\otimes z \mapsto x\otimes(y\otimes z)
\]
for all $x\in M_1$, $y\in M_2$ and $z\in M_3$.
\end{exercise}



\begin{exercise}
Let $R$ be an integral domain with fraction field $K$. Let $M$  be an $R$-module. Show that every element of $K\otimes_R M$ is of the form $\lambda\otimes x$ with $\lambda \in K$ and $x\in M$.
\end{exercise}


\begin{exercise}
Let $R$ be a ring, $M$ a right and $N$ a left $R$-module. Show that there is a surjective morphism of 
abelian groups
\[
	M\otimes_\bZ N \to M\otimes_R N
\]
which maps $x\otimes y$ to $x\otimes y$. 
\end{exercise}

\begin{exercise}
Let $R$, $S$, $T$ be rings. Let $M$ be an $(R,S)$-bimodule and $N$ an $(S,T)$-bimodule. Let $P$ be an $(R,T)$-bimodule and let $f\colon M\times N\to P$ be an $S$-bilinear map. Let $h\colon M\otimes_S N \to P$ be the unique additive map such that $h(x\otimes y)=f(x,y)$ for all $x\in M$, $y\in N$. 
\begin{enumerate}
\item show that $h$ is a homomorphism of $(R,T)$-bimodules if and only if the bilinear map $f$ satisfies $f(rx,y) = rf(x,y)$ and $f(x,yt) = f(x,y)t$ for all $x\in M$, $y\in N$, $r\in R$ and $t \in T$. 
\item conclude that the map $M\times N \to M\otimes_S N$ is universal amongst bilinear maps $f\colon M\times N \to P$ to $(R,T)$-bimodules satisfying the identities in (1). 
\end{enumerate}
\end{exercise}


\begin{exercise}\label{exc:tensor-not-exact}
Consider the short exact sequence of $\bZ$-modules
\[
	0 \longto \bZ \overset{2}{\longto} \bZ \longto \bZ/2\bZ \longto 0.
\]
Show that the sequence obtained by applying the functor $-\otimes_\bZ \bZ/2\bZ$ is \emph{not} exact.
\end{exercise}



\begin{exercise}\label{exc:tensor-with-free-is-exact}
Let $R$ be a ring and let $0\to M_1\to M_2 \to M_3\to 0$ be a short exact sequence of right $R$-modules. Let $N$ be a \emph{free} left $R$-module. Show that the induced sequence
\[
	0 \to M_1\otimes_R N \to M_2 \otimes_R N \to M_3 \otimes_R N \to 0
\]
is a short exact sequence.
\end{exercise}


\begin{exercise}
Let $R$ be a ring and let $0\to M_1\to M_2 \to M_3\to 0$ be a split short exact sequence of right $R$-modules. Let $N$ be a left $R$-module. Show that the induced sequence
\[
	0 \to M_1\otimes_R N \to M_2 \otimes_R N \to M_3 \otimes_R N \to 0
\]
is a short exact sequence.
\end{exercise}

\begin{exercise}Let $R$ be a principal ideal domain and let
\[
	M \cong R^m \oplus R/p_1^{e_1}R \oplus \cdots \oplus R/p_n^{e_n}R
\]
 be a finitely generated $R$-module (with the notation of Corollary \ref{cor:structure-fg-mod-over-PID}). 
 \begin{enumerate}
 \item Let $K$ be the fraction field of $R$. Compute $\dim_K(K\otimes_R M)$.
 \item Let $p\in R$ be irreducible. Compute $\dim_{R/pR}(R/pR \otimes_R M)$.
 \end{enumerate}
\end{exercise}

%
%\begin{exercise}
%Deduce the universal property of extension of scalars (Proposition \ref{prop:universal-property-extension-of-scalars}) from the tensor-hom adjunction (in the form stated in Remark \ref{thm:alternate-adjunction}).
%\end{exercise}
%
%% TODO: add hint, be careful with left and right
%
%\begin{exercise}[Extension of scalars preserves finite generation]
%Let $f\colon R\to S$ be a morphism of rings. Let $M$ be a finitely generated $R$-module. Show that $S\otimes_R M$ is a finitely generated $S$-module.
%\end{exercise}
%
%\begin{exercise}[Restriction of scalars does not preserve finite generation]
%Give an example of a morphism $f\colon R\to S$ of commutative rings, and a finitely generated $S$-module $M$ such that $M$ is not finitely generated as an $R$-module.
%\end{exercise}
%
%
%\begin{exercise}
%Let $f\colon R\to R'$ and $g\colon R'\to R''$ be morphisms of rings, and consider the extension of scalar functors
%\[
%	{}_R\Mod \to {}_{R'}\Mod \to {}_{R''}\Mod.
%\]
%Show that the composite of these functors is isomorphic to the extension of scalar functor
%\[
%	{}_R\Mod \to {}_{R''}\Mod
%\]
%for the ring morphism $gf\colon R \to R''$.
%\end{exercise}
%
%\begin{exercise}
%Let $f\colon R\to S$ be a \emph{surjective} morphism of rings. Show that the restriction of scalars functor
%\[
%	{}_S\Mod \to {}_R\Mod,\, M \mapsto M
%\]
%is full and faithful. Give an example to show that it need not be essentially surjective. 
%\end{exercise}



%%%%%%%%%%%%%%%%%%%%%%%%%
% ADJOINT PAIRS OF FUNCTORS
%%%%%%%%%%%%%%%%%%%%%%%%%

