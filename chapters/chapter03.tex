
\chapter{Finitely generated modules over a PID}\label{ch:modules-over-PID}

\section{Introduction}



The classification of finite abelian groups states that for every finite abelian group $A$ there are prime numbers $p_i$ (not necessarily distinct) and exponents $e_i\geq 1$ such that
\[
	A \cong (\bZ/p_1^{e_1}\bZ) \times \cdots \times (\bZ/p_n^{e_n}\bZ).
 \]
 
 The existence of Jordan normal forms states that for every square matrix $P$ over $\bC$ there exist
 complex numbers $\lambda_i$ (not necessarily distinct) and integers $e_i \geq 1$ so that
 $P$ is conjugate to a block diagonal matrix with blocks
\[
\left(\begin{matrix} \lambda_i & 0 &   \cdots & 0 & 0 \\ 
	1 & \lambda_i &  \cdots & 0 & 0 \\
	\vdots & \vdots &  & \vdots & \vdots \\ 
	0 & 0 &  \cdots & \lambda_i & 0 \\
	0 & 0 &   \cdots & 1& \lambda_i \end{matrix} \right)
\] 
of size $e_i$.
 
In this chapter, we will see that these two theorems are just two instances of one and the same theorem about finitely generated modules over a principal  ideal domain. In the first case, the PID will be $\bZ$, in the second case it will be $\bC[X]$.



\section{Review of principal ideal domains}

Let $R$ be a PID (principal ideal domain). Recall that this means that $R$ is an integral domain (a nonzero commutative ring without zero divisors), and that for every ideal $I\subset R$ there is an $r\in R$ with $I=(r)=Rr$. For our purposes, the most important examples are $R=\bZ$ and $R=K[X]$ with $K$ a field. Other important examples are the ring of power series $K[[X]]$, the ring of $p$-adic integers $\bZ_p$, and the ring of Gaussian integers $\bZ[i]$. Also a field $K$ is a PID, but of a rather trivial kind.

PID's are unique factorization domains. This means that every non-zero element $r$ in a principal ideal domain $R$ can be written as
\[
	r = u p_1^{e_1} \cdots p_n^{e_n}
\]
with $u \in R^\times$, with the $p_i \in R$ irreducible, and with $e_i$ non-negative integers. Moreover, such factorization is unique up to multiplying $u$ and the $p_i$'s by units, and up to permuting the factors.

Using this prime factorization, we can define greatest common divisors  $\gcd(r,s)$ of two non-zero elements of $R$. They are uniquely determined up to units (for example both $2$ and $-2$ are a gcd of $4$ and $6$ in $\bZ$). Similarly, we can define gcds of any sequence $r_1,\ldots,r_n$ of elements of $R$ which is not identically zero (ignoring the zeroes in the sequence).

An element $d\in R$ is a gcd of $r_1$, \ldots, $r_n$ if and only if $d$ generates the ideal $(r_1,\ldots, r_n)$ of $R$. In particular, there are $a_1,\ldots, a_n$ in $R$ with $d=a_1r_1+\cdots+a_nr_n$. For example, if $r$ and $s$ are coprime (have no common prime factor), then there are $a$ and $b$ in $R$ with $ar+bs=1$.


\section{Free modules of finite rank over a PID}



\begin{proposition}\label{prop:submodule-of-free-module-over-PID}
Let $R$ be a PID and $M\subset R^n$ a submodule. Then $M\cong R^k$ for some $k\leq n$.
\end{proposition}

Hence over a PID a submodule of a free module of finite rank is itself free of finite rank.   
The condition that $R$ be a PID cannot be dropped from the proposition, see Exercise \ref{exc:non-free-submodule}. 

% TODO: true without finite generation? Useful for Ext^n later...

\begin{proof}[Proof of Proposition \ref{prop:submodule-of-free-module-over-PID}]
We use induction on $n$. For $n=0$ we have $M=R^n=0$, and $M$ is indeed free of rank $0$. Assume that the proposition has been shown to hold for submodules of $R^{n-1}$. Consider the projection
\[
	\pi\colon R^n \to R,\, (r_1,\ldots, r_n) \to r_n
\]
with kernel $R^{n-1}\times \{0\}$. Then we have a short exact sequence
\[
	0 \longto M \cap \ker \pi \longto M \longto \pi(M) \longto 0.
\] 
By the induction hypothesis, we have that the submodule $M\cap \ker \pi$ of $R^{n-1}\times \{0\}$ is free of rank $k\leq n-1$. If $\pi(M)=0$, then we are done. If $\pi(M)\neq 0$, then $\pi(M)$ is an ideal in $R$,
hence a principal ideal, hence $\pi(M)\cong R$ as $R$-module. By Exercise \ref{exc:free-module-split-short-exact-sequence}
any short exact sequence of $R$-modules of the form
\[
	0 \longto N \longto M \longto R \longto 0
\]
splits, and we find $M \cong R \oplus (M \cap \ker \pi) \cong R^{k+1}$ with $k+1\leq n$.
\end{proof}


\begin{corollary}\label{cor:free-presentation}
Let $R$ be a PID and $M$ a finitely generated $R$-module. Then there exists an exact sequence 
\[
	0 \longto F_1 \longto F_2 \longto M \longto 0
\]
with $F_1$ and $F_2$ free $R$-modules of finite rank.
\end{corollary}

\begin{proof}
Since $M$ is finitely generated, by Proposition \ref{prop:universal-property-free-module} there is a surjection $F_2 \to M$ with $F_2$ a free module of finite rank. The kernel $F_1\subset F_2$ is also free of finite rank, thanks to Proposition \ref{prop:submodule-of-free-module-over-PID}.
\end{proof}


\section{Structure of finitely generated modules over a PID}

The main theorem of this chapter is the following structure theorem.

\begin{theorem}\label{thm:structure-fg-mod-over-PID}
Let $R$ be a principal ideal domain and $M$  a finitely generated $R$-module. Then there exists an integer $n$ and non-zero ideals $I_1$,  \ldots,  $I_k$ of $R$ such that
\[
	M \cong R^n \oplus R/I_1 \oplus \cdots \oplus R/I_k
\]
as $R$-modules.
\end{theorem}

For the proof we need the notion of \emph{content} of an element of a module.
Let $R$ be a commutative ring and let $M$ be an $R$-module. An element $x\in M$ determines a map
\begin{equation}\label{eq:content-evaluation}
	\Hom_R(M,R) \to R,\,f \mapsto f(x).
\end{equation}
This map is $R$-linear, hence the image is an ideal. We denote it by $c_M(x)$, and call it the \emph{content} of $x$.

\begin{lemma}\label{lemma:non-zero-content}
If $M$ is free of finite rank, and $x\in M$ is non-zero, then $c_M(x)$ is a non-zero ideal in $R$.
\end{lemma}

\begin{proof}
Without loss of generality, we may assume $M=R^n$ and $x=(x_1,\ldots, x_n)$. Since $x$ is non-zero, there exists an $i$ with $x_i\neq 0$. Consider the map
\[
	\pr_i \colon R^n \to R,\, (y_1,\ldots, y_n) \mapsto y_i.
\]
We have $\pr_i\in \Hom_R(M,R)$ and $\pr_i(x)\neq 0$, hence $c_M(x)\neq 0$.
\end{proof}

Note that the coordinates $x_i$ of $x\in M \cong R^n$ depend on the choice of basis of $M$, but that the content ideal $c_M(x)$ is independent of such choice!

\begin{lemma}\label{lemma:content-division}
Let $R$ be a principal ideal domain, $M$ a free $R$-module of finite rank, and $x\in M$ a non-zero element.  Then
there is a surjective $R$-linear map $f\colon M\to R$ such that $f(x)\in R$ generates the ideal $c_M(x)\subset R$.
\end{lemma}

\begin{proof}
Since $R$ is a principal ideal domain, there exists an $f\in \Hom_R(M,R)$ such that $f(x)$ generates the ideal $c_M(x)$. Moreover, by Lemma \ref{lemma:non-zero-content} we have $f(x)\neq 0$.

Consider the image of $f\colon M\to R$. This is an ideal $I\subset R$, and it suffices to show that $I=R$. As $R$ is a PID, the ideal $I$ is generated by some element $r\in R$. For every $y \in M$ we have $f(y)\in I$ and (since $R$ is an integral domain) there exists a unique $g(y)\in R$ such that
\[
	f(y) = g(y) \cdot r.
\]
This defines an $R$-linear map $g\colon M\to R$. Now consider the element $g(x)\in R$. By definition of $c_M(x)$ we have $g(x)\in c_M(x)$. Since $f(x)$ generates $c_M(x)$ there is an $s\in R$ with $g(x)=sf(x)$. But we also have $f(x)=rg(x)$, hence $rs=1$, hence $I=R$ and $f$ is surjective.
\end{proof}


\begin{proof}[Proof of Theorem \ref{thm:structure-fg-mod-over-PID}]
By Corollary \ref{cor:free-presentation} every finitely generated $R$-module $M$ sits in a short exact sequence
\[
	0 \longto F_1 \longto F_2 \longto M \longto 0
\]
with the $F_i$ free $R$-modules of finite rank. We will interpret $F_1$ as a submodule of $F_2$.

The proof goes by induction on the rank of $F_1$. If $F_1$ is zero, then $M\cong F_2\cong R^n$ for some $n$, and the theorem holds.

Otherwise, let  $x\in F_1$  be a non-zero element whose content $I := c_{F_2}(x)$ with respect to $F_2$ is maximal (amongst all the ideals in $R$ of the form $c_{F_2}(x)$ with $x\in F_1$). 

By Lemma \ref{lemma:content-division} there is a surjective map $f\colon F_2 \to R$ such that $f(x)$ generates $I$. 

\emph{Claim}. $f(F_1) = I \subset R$. Indeed, since $f(x)$ generates $I$ we certainly have $f(F_1)\supset I$. Conversely, for $z\in F_1$, let $d$ be a gcd of $f(z)$ and $f(x)$. Then there exists $r, s\in R$ with $rf(z)+sf(x)=d$, and hence $f(rz+sx)=d$. By the maximality of the content of $x$ we have that $d$ must equal $f(x)$ up to a unit, hence $f(z)$ must be divisible by $f(x)$ and hence $f(z)\in I$.

Denote the kernel of $f\colon F_2 \to R$ by $F_2'$ (note that by Proposition \ref{prop:submodule-of-free-module-over-PID}, this is also a free module). We have a short exact sequence
\[
	0 \longto F_2' \longto F_2 \overset{f}\longto R \longto 0.
\]
Similarly, let $F_1'$ be the kernel of the restriction $f\colon F_1 \to R$. We find a short exact `sub-sequence'
\[
	0 \longto F_1' \longto F_1 \overset{f}\longto I \longto 0.
\]
Since $f(x)$ generates $I = c_{F_2}(x)$ there is a $y\in F_2$ with $f(x)y = x$. Now 
the first sequence splits by the section $R\to F_2,\, 1\mapsto y$, and this section restricts to a section
$I\to F_1$ in the second short exact sequence.

We find $M\cong M' \oplus R/I$ with $M'$ given by the short exact sequence
\[
	0 \longto F_1' \longto F_2' \longto M' \longto 0.
\]
Since $F_1'$ has lower rank, the induction hypothesis guarantees
\[
	M' \cong R^{n'} \oplus R/I_1 \oplus \cdots \oplus R/I_k,
\]
which finishes the proof.
\end{proof}


\section{Application to Jordan normal form}


Theorem \ref{thm:structure-fg-mod-over-PID} has the following corollary.

\begin{corollary}\label{cor:structure-fg-mod-over-PID}
Let $R$ be a PID and $M$  a finitely generated $R$-module. Then there exists an integer $n$, irreducible elements $p_1,\ldots,p_k$ of $R$ and positive integers $e_1$, \ldots, $e_k$ such that
\[
	M \cong R^n \oplus R/p_1^{e_1}R \oplus \cdots \oplus R/p_k^{e_k}R
\]
as $R$-modules.
\end{corollary}

\begin{proof}
By Theorem \ref{thm:structure-fg-mod-over-PID} it suffices to show that for every non-zero ideal $I$  the $R$-module $R/I$ can be written in the desired form. Let $x$ be a generator of $I$, and consider its prime factorization
\[	
	x = u p_1^{e_1} \cdots p_k^{e_k}
\]
with $p_i$ pairwise non-associated primes. Then by the Chinese Remainder Theorem, we have
\[
	R/I \cong R/p_1^{e_1}R \oplus \cdots \oplus R/p_k^{e_k}R,
\]
as we had to show.
\end{proof}

We now consider two special cases of this corollary. The first one is a structure theorem for finitely generated and finite abelian groups.

\begin{theorem}[Classification of finitely generated abelian groups]
Let $A$ be a finitely generated abelian group. Then there exists an integer $n$, prime numbers 
$p_1$, \ldots, $p_k$, and positive integers $e_1$, \ldots, $e_k$ such that 
\[
	A \cong \bZ^{n} \times \bZ/{p_1^{e_1}}\bZ \times \cdots \times \bZ/{p_k^{e_k}}\bZ.
\]
If $A$ is a finite abelian group then 
\[
	A \cong \bZ/{p_1^{e_1}}\bZ \times \cdots \times \bZ/{p_k^{e_k}}\bZ.
\]
\end{theorem}

\begin{proof}
Apply Corollary \ref{cor:structure-fg-mod-over-PID} to the case $R=\bZ$.
\end{proof}

The second special case is a structure theorem for endomorphisms of finite-dimensional vector spaces over $\bC$ (or over an algebraically closed field). 

\begin{theorem}Let $K$ be an algebraically closed field. Let $V$ be a finite-dimensional vector space over $K$. Let $\alpha \colon V\to V$ be an endomorphism. Then there exist $\lambda_1$, \ldots, $\lambda_k$ in $K$, positive integers $e_1,\ldots, e_k$, and a decomposition
\[
	V = V_1 \oplus \cdots \oplus V_k
\]
such that $\alpha(V_i)\subset V_i$, and such that each $V_i$ has a basis on which $\alpha$ is expressed
as the standard Jordan matrix
\[
\left(\begin{matrix} \lambda_i & 0 &   \cdots & 0 & 0 \\ 
	1 & \lambda_i &  \cdots & 0 & 0 \\
	\vdots & \vdots &  & \vdots & \vdots \\ 
	0 & 0 &  \cdots & \lambda_i & 0 \\
	0 & 0 &   \cdots & 1& \lambda_i \end{matrix} \right)
\] 
of size $e_i$.
\end{theorem}

\begin{proof}
As in Example \ref{exa:vect-with-endo}, we turn the vector space $V$ into a $K[X]$-module, with $X$ acting via the endomorphism $\alpha$. Since $V$ is finite-dimensional, it is finitely generated as a $K$-module, hence a fortiori also as a $K[X]$-module.

Since $K$ is algebraically closed, the irreducible elements of $K[X]$ are (up to units) the linear polynomials
$X-\lambda$ with $\lambda \in K$. By Corollary \ref{cor:structure-fg-mod-over-PID}, we have
\[
	V\cong K[X]^n \oplus \frac{K[X]}{(X-\lambda_1)^{e_1}K[X]} \oplus \cdots \oplus
	\frac{K[X]}{(X-\lambda_k)^{e_k} K[X]}.
\]
Note that $K[X]$ is infinite-dimensional as a $K$-vector space, so we necessarily must have $n=0$.

Without loss of generality, we may assume 
\[
	V = K[X]/(X-\lambda)^{e}K[X]
\]
for some $\lambda \in K$ and $e > 0$. Consider the elements
\[
	v_j := (\bar{X}-\lambda)^j \in V \quad j \in \{0,\ldots, e-1\}.
\]
Note that the $v_j$ form a $K$-basis of $V$. The action of $\alpha$ on this basis is given by
\[
	\alpha(v_j) = X \cdot (\bar{X}-\lambda)^j
	= (\bar{X}-\lambda)^{j+1} + \lambda (\bar{X}-\lambda)^j
\]
hence
\[
	\alpha(v_j) = \begin{cases}
	 v_{j+1} + \lambda v_j & j<e-1 \\
	 \lambda v_j & j = e-1.
	 \end{cases}
\]
and we see that the matrix of $\alpha$ with respect to this basis is the standard Jordan block of eigenvalue $\lambda$ and size $e$.
\end{proof}


\newpage
\section*{Exercises}

\begin{exercise}Let $R$ be a commutative ring and $\fm \subset R$ a maximal ideal. 
Let $M$ be an $R$-module. Show that $M/\fm M$ is a vector space over $R/\fm$. Show that if $M$ is generated by $x_1,\ldots, x_n$ then $M/\fm M$ has dimension at most $n$ over $R/\fm$.
\end{exercise}

\begin{exercise}\label{ex:torsion-not-submodule}
Let $R$ be a commutative ring and $M$ an $R$-module. An element $x\in M$ is called a \emph{torsion} element if there exists a non-zero $r\in R$ with $rx=0$. 

Assume that $R$ is an integral domain. Show that the torsion elements of an $R$-module $M$ form a submodule of $M$. 

Give an example to show that the condition that $R$ is an integral domain cannot be dropped.
\end{exercise}

\begin{exercise}\label{exc:non-free-submodule}
Let $R$ be an integral domain. Show that the following are equivalent:
\begin{enumerate}
\item every submodule of a free $R$-module of finite rank is free of finite rank,
\item $R$ is a principal ideal domain.
\end{enumerate}
\end{exercise}


\begin{exercise}
Let $R$ be a PID and let $M$ be an $R$-module which can be generated by $n$ elements. Show that every submodule $N\subset M$ can be generated by $n$ elements. Show that the condition that $R$ is a PID cannot be dropped.
\end{exercise}

\begin{exercise}
Let $R$ be a PID and let $M$ be a finitely generated $R$-module such that for all $r\in R$ and $x\in M$ we have that $rx=0$ implies $r=0$ or $x=0$. Show that $M$ is free of finite rank.
\end{exercise}

\begin{exercise}
Let $R$ be a ring and let 
\[
	0 \longto M {\longto} E {\longto} N \longto 0
\]
be a short exact sequence of $R$-modules. Assume that $M$ can be generated by $m$ elements, and that $N$ can be generated by $n$ elements. Show that $E$ can be generated by $m+n$ elements.
\end{exercise}

\begin{exercise}
Let $R$ be a PID, and let $p_1$ and $p_2$ be irreducible elements with $(p_1) \neq (p_2)$. Let $e_1,e_2$ be non-negative integers. Show that the only $R$-module homomorphism
\[
	R/p_1^{e_1}R \to R/p_2^{e_2}R
\]
is the zero homomorphism. 
\end{exercise}

\begin{exercise}
Let $R$ be a PID and let $p\in R$ be irreducible. Let $e_1,e_2$ be non-negative integers. Show that there is an isomorphism of
 $R$-modules
\[
	\Hom_R(R/p^{e_1}R,R/p^{e_2}R) \cong R/p^e R
\]
with $e=\min(e_1,e_2)$.
\end{exercise}

\begin{exercise}
Describe all $\bZ[i]$-modules with at most $10$ elements, up to isomorphism.
\end{exercise}

\begin{exercise}
Let $R$ be a PID and let $p \in R$ be an irreducible element. Let $E$ be an $R$-module contained in a short exact sequence
\[
	0 \longto R/pR \overset{f}{\longto} E \overset{g}{\longto} R/pR \longto 0.
\]
Show that either $E \cong R/pR \oplus R/pR$ or $E\cong R/p^2R$, that both options occur, and that 
$R/pR \oplus R/pR \not\cong R/p^2R$.
\end{exercise}

\begin{exercise}[$\star$]
Let
\[
	0 \longto A_1 \longto A_2 \longto A_3 \longto \cdots \longto A_n \longto 0
\]
be an exact sequence of finite abelian groups. Show that the equality
\[
	\prod_i |A_i|^{(-1)^{i}} = 1
\]
holds.
\end{exercise}

\begin{exercise}
Let $K$ be a field and let $f_1,\ldots, f_n \in K[X]$ be monic polynomials. Consider the $K[X]$-module
\[
	V := K[X]/(f_1) \oplus \cdots \oplus K[X]/(f_n).
\]
Show that the characteristic polynomial of the endomorphism $v\mapsto X\cdot v$ of the $K$-vector space $V$ equals $\prod_i f_i$.
\end{exercise}

\begin{exercise}
Let $K$ be a field and let $V$ be a finite-dimensional vector space over $K$. Let $\alpha$ be an endomorphism of $V$. 
Assume that the characteristic polynomial of $\alpha$ is irreducible. Show that there is no proper non-zero subspace $W\subset V$ with $\alpha(W) \subset W$.
\end{exercise}

\begin{exercise}
Let $K$ be a field and let $V$ be a finite-dimensional vector space over $K$. Let $\alpha$ be an endomorphism of $V$. Assume that the characteristic polynomial of $\alpha$ is separable. Show that there are subspaces $W_1,\ldots, W_n$ of $V$, such that the following hold:
\begin{enumerate}
\item  $V = W_1 \oplus W_2 \oplus \cdots \oplus W_n$,
\item $\alpha(W_i) \subset W_i$ for every $i$,
\item the characteristic polynomial of $\alpha_{|W_i}$ is irreducible for every~$i$.
\end{enumerate}
Give an example to show that the condition that the characteristic polynomial is separable cannot be dropped.
\end{exercise}


\begin{exercise}[$\star\star$]
Let $R$ be a PID. An $R$-module $M$ is called \emph{torsion} if for every $x\in M$ there is a non-zero $r\in R$ with $rx=0$. 
For a finitely generated torsion $R$-module
\[
	M\cong R/I_1 \oplus \cdots \oplus R/I_k
\]
denote by $F(M)$ the ideal $I_1I_2\cdots I_k$. Show that $F(M)$ is independent of the chosen decomposition. Show that if 
\[
	0 \longto M_1 \longto M_2 \longto M_3 \longto 0
\]
is a short exact sequence of $R$-modules, and if $M_1$ and $M_3$ are finitely generated torsion $R$-modules, then $M_2$ is a finitely generated torsion $R$-module and $F(M_2)=F(M_1)F(M_3)$.
\end{exercise}


%%%%%%%%%%%%%%%%%%%%%%%%%%%%%%%%%%%%%%%%%%
% CHAPTER: CATEGORIES %
%%%%%%%%%%%%%%%%%%%%%%%%%%%%%%%%%%%%%%%%%%
