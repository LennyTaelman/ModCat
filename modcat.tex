\documentclass[11pt]{amsbook}

\setlength{\textwidth}{11.8cm} % or whatever length you want
\setlength{\textheight}{18.5cm} 
\calclayout

\usepackage{amsgen}
\usepackage{hyperref}
\usepackage{tikz-cd}
\usepackage{chngcntr}
\usepackage{wrapfig}
\usepackage{etoolbox}

\usepackage[T2A,T1]{fontenc}
\DeclareSymbolFont{cyrillic}{T2A}{cmr}{m}{n}
\DeclareMathSymbol{\Sha}{\mathalpha}{cyrillic}{216}


\makeatletter
\newcommand*\rel@kern[1]{\kern#1\dimexpr\macc@kerna}
\newcommand*\widebar[1]{%
  \begingroup
  \def\mathaccent##1##2{%
    \rel@kern{0.8}%
    \overline{\rel@kern{-0.8}\macc@nucleus\rel@kern{0.2}}%
    \rel@kern{-0.2}%
  }%
  \macc@depth\@ne
  \let\math@bgroup\@empty \let\math@egroup\macc@set@skewchar
  \mathsurround\z@ \frozen@everymath{\mathgroup\macc@group\relax}%
  \macc@set@skewchar\relax
  \let\mathaccentV\macc@nested@a
  \macc@nested@a\relax111{#1}%
  \endgroup
}

\newcommand*\cocolon{%
        \nobreak
        \mskip6mu plus1mu
        \mathpunct{}%
        \nonscript
        \mkern-\thinmuskip
        {:}%
        \mskip2mu
        \relax
}

\makeatother


\usepackage{xstring}
\renewcommand{\tocappendix}[3]{\indentlabel{\IfStrEq{#3}{Bibliography}{Bibliography.}{\IfStrEq{#3}{Index}{Index.}{Appendix #2.\quad{#3}}}}}

\renewcommand{\chaptername}{Chapter}

\DeclareRobustCommand{\gobblefive}[5]{}
\newcommand*{\SkipTocEntry}{\addtocontents{toc}{\gobblefive}}

\usepackage{amssymb}
\usepackage{amscd}
\usepackage[all,cmtip]{xy}

\newcommand{\medwedge}{{\scalebox{1.25}{$\wedge$}}}

\newcommand{\isomto}{\overset{\sim}{\to}}
\newcommand{\longisomto}{\overset{\sim}{\longrightarrow}}

\newcommand{\isomfrom}{\overset{\sim}{\longleftarrow}}
\newcommand{\surjto}{\twoheadrightarrow}
\newcommand{\longsurjto}{\relbar\joinrel\twoheadrightarrow}
\newcommand{\surjfrom}{\twoheadleftarrow}
\newcommand{\injto}{\hookrightarrow}
\newcommand{\longinjto}{\lhook\joinrel\longrightarrow}
\newcommand{\injfrom}{\hookleftarrow}
\newcommand{\longto}{\longrightarrow}
\newcommand{\longfrom}{\longleftarrow}

\newcommand{\plim}{\mathop{\varprojlim}\limits}

\renewcommand{\ast}{\star}

\DeclareMathOperator{\Frac}{\mathrm{Frac}}
\DeclareMathOperator{\Frob}{\mathrm{Frob}}
\DeclareMathOperator{\Spm}{\mathrm{Spm}}
\DeclareMathOperator{\Sym}{\mathrm{Sym}}
\DeclareMathOperator{\Aut}{\mathrm{Aut}}
\DeclareMathOperator{\Tot}{\mathrm{Tot}}
\DeclareMathOperator{\rk}{\mathrm{rk}}
\DeclareMathOperator{\End}{\mathrm{End}}
\DeclareMathOperator{\Reg}{\mathrm{Reg}}
\DeclareMathOperator\Hom{Hom}
\DeclareMathOperator\Map{Map}
\DeclareMathOperator\Spec{Spec}
\DeclareMathOperator\GL{GL}
\DeclareMathOperator\Gal{Gal}
\DeclareMathOperator\QCoh{{\bf{QCoh}}}
\DeclareMathOperator\Coh{{\bf{Coh}}}
\DeclareMathOperator\Proj{Proj}
\DeclareMathOperator\Crys{{\bf{Crys}}}
\DeclareMathOperator\Nil{{\bf{Nil}}}
\DeclareMathOperator\Mod{{\bf{Mod}}}
\renewcommand{\Vec}{\operatorname{{\bf{Vec}}}}
\DeclareMathOperator\Cl{Cl}
\DeclareMathOperator\im{im}
\DeclareMathOperator\coker{coker}
\DeclareMathOperator\Pic{Pic}
\DeclareMathOperator\Mat{Mat}
\DeclareMathOperator\Ind{Ind}
\DeclareMathOperator\Res{Res}
\DeclareMathOperator\Lie{Lie}
\DeclareMathOperator\Det{det}
\DeclareMathOperator\tr{tr}
\DeclareMathOperator\Nm{Nm}
\DeclareMathOperator\pr{pr}
\DeclareMathOperator\TS{TS}
\DeclareMathOperator\Ext{Ext}
\DeclareMathOperator\ext{ext}
\DeclareMathOperator\Tor{Tor}
\DeclareMathOperator\Ann{Ann}

\DeclareMathOperator\Kosz{Kosz}
\DeclareMathOperator\gr{gr}
\DeclareMathOperator\ob{ob}

\DeclareMathOperator\colim{colim}


\def\bQ{{\mathbf{Q}}} \def\bZ{{\mathbf{Z}}} \def\N{{\mathbf{N}}}
\def\bF{{\mathbf{F}}} \def\bG{\mathbf{G}} \def\bR{{\mathbf{R}}}
\def\bC{{\mathbf{C}}} \def\bP{{\mathbf{P}}} \def\bA{{\mathbf{A}}}
\def\bN{{\mathbf{N}}}
\def\bOne{{\mathbf{1}}}

\def\cO{{\mathcal{O}}} \def\cV{{\mathcal{V}}} \def\cM{{\mathcal{M}}}
\def\cU{{\mathcal{U}}} \def\cC{{\mathcal{C}}} \def\cL{{\mathcal{L}}}
\def\cF{{\mathcal{F}}} \def\cD{{\mathcal{D}}} \def\cI{{\mathcal{I}}}
\def\cE{{\mathcal{E}}} \def\cG{{\mathcal{G}}} \def\cH{{\mathcal{H}}}
\def\cN{{\mathcal{N}}} \def\cA{{\mathcal{A}}} \def\cB{{\mathcal{B}}}
\def\cQ{{\mathcal{Q}}} \def\cP{{\mathcal{P}}} \def\cS{{\mathcal{S}}}
\def\cR{{\mathcal{R}}} \def\cZ{{\mathcal{Z}}}

\def\rH{{\mathrm H}} \def\rK{{\mathrm K}} \def\rR{{\mathrm R}}
\def\rmd{{\mathrm d}} \def\rmT{{\mathrm T}} \def\rmk{{\mathrm k}}
\def\rL{{\mathrm L}} \def\rD{{\mathrm D}}\def\rB{{\mathrm B}}

\def\fp{{\mathfrak p}} \def\fm{{\mathfrak{m}}} \def\fS{{\mathfrak S}}
\def\fU{{\mathfrak U}} \def\fC{{\mathfrak{C}}} \def\fV{{\mathfrak{V}}}


\def\id{{\rm id}}
\def\et{\mathrm{et}}
\def\ab{\mathrm{ab}}
\def\opp{\mathrm{op}}
\def\sep{\mathrm{sep}}
\def\tors{\mathrm{tors}}

\def\Set{\mathbf{Set}}
\def\Ab{\mathbf{Ab}}
\def\Top{\mathbf{Top}}
\def\hTop{\mathbf{hTop}}
\def\Grp{\mathbf{Grp}}
\def\Ring{\mathbf{Ring}}
\def\CRing{\mathbf{CRing}}
\def\Alg{\mathbf{Alg}}
\def\CAlg{\mathbf{CAlg}}
\def\Mod{\mathbf{Mod}}
\def\FVec{\mathbf{FVec}}
\def\Fun{\mathbf{Fun}}
\def\Ch{\mathbf{Ch}}
\def\Ho{\mathbf{Ho}}

\newcommand{\comment}[1]{}


\makeatletter
\patchcmd{\@thm}{\thm@headfont{\scshape}}{\thm@headfont{\scshape\bfseries}}{}{}
%\patchcmd{\@thm}{\thm@notefont{\fontseries\mddefault\upshape}}{}{}{}
\makeatother
\theoremstyle{plain}
\newtheorem{theorem}{Theorem}
\newtheorem{corollary}[theorem]{Corollary}
\newtheorem{lemma}[theorem]{Lemma}
\newtheorem{proposition}[theorem]{Proposition}
\theoremstyle{definition}
\newtheorem{definition}[theorem]{Definition}
\newtheorem{example}[theorem]{Example}
\newtheorem{examples}[theorem]{Examples}
\newtheorem{remark}[theorem]{Remark}
\newtheorem{exercise}{Exercise}

\counterwithin{theorem}{chapter}
\counterwithin{exercise}{chapter}


\renewcommand*{\thefootnote}{\fnsymbol{footnote}}

\makeindex
\begin{document}

\title
 {Modules and Categories\footnote{Preliminary version, \today}}


\author
 {Lenny Taelman \\\medskip\bigskip\bigskip\medskip\bigskip\bigskip\medskip\bigskip\bigskip\medskip\bigskip\bigskip\medskip\bigskip\bigskip\medskip\bigskip\bigskip\medskip\bigskip\bigskip\medskip\bigskip\bigskip\medskip\bigskip\bigskip\medskip\bigskip\bigskip\medskip\bigskip\bigskip
 \includegraphics[height=2cm]{uva}}
 
 \date
 {\today}

%\begin{abstract} 
%\end{abstract}


\maketitle


\tableofcontents


\chapter*{Foreword}

These are course notes for a one-semester third-year course on categories and modules taught at the University of Amsterdam. 

\section*{Prerequisites}

Linear algebra, groups, elementary point-set topology, and the basics of rings and fields. Several exercises and  examples in the parts about categories refer to basic concepts in algebraic topology, Galois theory, or representation theory. Although in principle these can be skipped, developing a rich vocabulary of natural  examples is probably the most important aspect of becoming acquainted with categories.


\section*{Other sources}

Excellent alternative sources for much of the material in this course are the concise but clear and well-written Atiyah \& MacDonald \cite{AtiyahMacDonald}, the more lengthy Lang \cite{Lang}, the course notes by Moerdijk \cite{Moerdijk} and (in Dutch) the Algebra 2 course notes by Stevenhagen \cite{Stevenhagen}. A good introduction in the language of categories and functors containing much more than these notes is \cite{Leinster}.


\section*{Acknowledgements}

Many thanks to  Wessel Bindt, David de Boer, Jeroen Dekker, Koen van Duin, Eline Filius, Pepijn Hoefgeest, Tycho van Hoof, Juultje Kok, Christopher Spelt, Tijn de Vos, for their many corrections and suggestions to the first drafts of these course notes. 

% Max Blans, Reinier Kramer


%%%%%%%%%%%%%%%%%%%%%%%%%%%%%%%%%%%%%%%%%%
% CHAPTER: MODULES OVER A RING %
%%%%%%%%%%%%%%%%%%%%%%%%%%%%%%%%%%%%%%%%%%

\chapter{Modules over a ring}
\label{chapter:modules}


\section{Left and right modules}

Let $R$ be a ring. Recall that this means $R$ is a set equipped with an addition $(s,t)\mapsto s+t$, multiplication $(s,t)\mapsto st$, and distinguished elements $0\in R$ and $1\in R$ satisfying 
\begin{enumerate}
\item[(R1)] $(R,+,0)$ is an abelian group,
\item[(R2)] for all $r,s,t \in R$ we have $(rs)t = r(st)$,
\item[(R3)] for all $r,s,t \in R$ we have $r(s+t) = rs+rt$ and $(r+s)t=rt+st$,
\item[(R4)] for all $r\in R$ we have $1r = r1=r$.
\end{enumerate}


\begin{definition}A \emph{left module} over a ring $R$ is an abelian group $M$ equipped with an operation
\[
	R\times M \to M, (r,x) \mapsto rx
\]
satisfying for all $r,s\in R$ and $x,y \in M$ the following identities:
\begin{enumerate}
\item[(M1)] $r(x+y)=rx+ry$,
\item[(M2)] $(r+s)x=rx+sx$,
\item[(M3)] $(rs)x=r(sx)$,
\item[(M4)] $1x=x$.
\end{enumerate}
\end{definition}

One also says that `$R$ acts on $M$', so that axiom (M3) for example expresses that acting by $rs$ is the same as first acting by $s$, and then by $r$.

We use the same symbol $0$ to denote the elements $0\in M$ and $0\in R$. This should not lead to confusion, see Exercise \ref{exc:zeroes}.

A \emph{right module} over $R$ is defined similarly: the action is written on the right: $M\times R\to M, (x,r) \mapsto xr$, and must satisfy
\begin{enumerate}
\item[(M1')] $(x+y)r=xr+yr$,
\item[(M2')] $x(r+s)=xr+xs$,
\item[(M3')] $x(rs)=(xr)s$,
\item[(M4')] $x1=x$.
\end{enumerate}

 If $R$ is commutative, then the difference between a left and a right module is purely a matter of notation, but over a non-commutative ring the axioms (M3) and (M3') give  genuinely different conditions.

We will mostly work with left modules, and simply call them \emph{modules over $R$} or \emph{$R$-modules}. 

There is another way to describe (left) modules, using the endomorphism ring of an abelian group.
Let $A$ be an abelian group. Denote by $\End(A)$ the set of group homomorphisms $A\to A$. This forms a ring with addition and multiplication of $f,g\in \End(A)$ defined by pointwise addition
\[
	f+g\colon A\to A, \,a \mapsto f(a)+g(a)
\]
and composition
\[
	fg\colon A\to A,\, a\mapsto f(g(a)).
\]
The zero element of this ring is the constant map $0\colon a \mapsto 0$, and the unit element is the identity map $\id_A\colon a\mapsto a$.


\begin{lemma}\label{lemma:left-module-via-end}
Let $M$ be an $R$-module. Then the map
\[
	R \to \End(M), \, r \mapsto \left( x \mapsto rx \right)
\]
is a ring homomorphism. Conversely, let $M$ be an abelian group and $\phi\colon R\to \End(M)$ be a ring homomorphism. Then the operation
\[
	R\times M \to M, (r, x) \mapsto rx:= \phi(r)(x)
\]
gives $M$ the structure of an $R$-module.
\end{lemma}

\begin{proof}See Exercise \ref{exc:left-module-via-end}.
\end{proof}

In other words, a (left) $R$-module is the same as an abelian group $M$ together with a ring homomorphism $R\to \End(M)$. 




\section{First examples}

\begin{example} Let $R$ be a ring. Then the trivial group $\{0\}$ is an $R$-module with $r0:=0$ for all $r\in R$. We denote this module by $0$, and call it the \emph{zero module}.
\end{example}

\begin{example} Let $R$ be a ring and $n\geq 0$. Then $M:=R^n$ is an $R$-module with addition
\[
	(x_1,\ldots, x_n) + (y_1,\ldots, y_n) := (x_1+y_1, \ldots, x_n+y_n)
\]
and $R$-action
\[
	r\cdot (x_1,\ldots, x_n) := (rx_1,\ldots, rx_n).
\]
For $n=0$ we obtain the zero module $R^0=0$ and for $n=1$ we obtain the $R$-module $R$.
\end{example}

\begin{example} For every abelian group $A$ there is a unique ring homomorphism 
$\bZ \to \End(A)$. It follows that a $\bZ$-module is the same as an abelian group.
For $r\in \bZ$ and $x\in A$ we have
\[
	r \cdot x = \begin{cases} x + \cdots + x\quad \text{ ($r$ terms) } & r \geq 0 \\
	-(x+ \cdots + x ) \quad\text{($-r$ terms) } & r \leq 0 \end{cases}
\]
\end{example}

\begin{example} Let $K$ be a field. Then a $K$-module is the same as a $K$-vector space.
\end{example}

\begin{example} Let $K$ be a field and $n\geq 0$. Let $\Mat_n(K)$ be the ring of $n$ by $n$ matrices over $K$. Then $K^n$ is a left $\Mat_n(K)$-module via
\[
	\Mat_n(K) \times K^n \mapsto K^n, (A, v) \mapsto A\cdot v,
\]
where we interpret vectors $v\in K^n$ as column matrices.
\end{example}


\begin{example} Let $R$ be a ring and $I\subset R$ an ideal. Then $I$ is an $R$-module.
\end{example}


\begin{example}\label{exa:vect-with-endo}
Let $K$ be a field. Let $V$ be a $K$-vector space, and $\alpha\colon V\to V$ be a $K$-linear endomorphism of $V$. Then there is a unique ring homomorphism
\[
	\rho\colon K[X] \to \End(V)
\]
such that 
\begin{enumerate}
\item $\rho(\lambda)(v) = \lambda v$ for all $\lambda \in K$ and $v\in V$\!, 
\item $\rho(X)=\alpha$.	
\end{enumerate}
This homomorphism is given by
\begin{equation}\label{eq:vect-with-endo}	
	\rho\colon \sum \lambda_i X^i \mapsto \left( v \mapsto \sum \lambda_i \alpha^i(v) \right),
\end{equation}
where $\alpha^i$ denotes the iterated composition $\alpha \circ \cdots \circ \alpha$. In particular, $V$ obtains the structure of a $K[X]$-module.

Conversely, given a $K[X]$-module $V$, the restriction of the action of $K[X]$ to $K$ makes $V$ into a $K$-vector space, and the map
\[
	\alpha\colon V \to V,\, v \mapsto X\cdot v
\]
is $K$-linear. We conclude that a $K[X]$-module is the same as a $K$-vector space equipped with an endomorphism (namely the action of $X$).

We will see in Chapter \ref{ch:modules-over-PID} that the `Jordan normal form' of complex square matrices ($\bC$-linear endomorphisms of $\bC^n$), is really a theorem about the structure of $\bC[X]$-modules, and that it is most naturally explained in  terms of ideals in $\bC[X]$.
\end{example}


\begin{example}[The group algebra and representations]\label{exa:group-algebra}
Let $K$ be a field and $G$ a group. Let $K[G]$ be the group algebra of $G$ over $K$. Elements of $K[G]$ are formal expressions 
\[
	\sum_{g\in G} a_g g \quad (a_g \in K)
\]
with $a_g=0$ for all but  finitely many $g$ (this is automatic if $G$ is a finite group).
Addition is defined in the obvious way. Multiplication is defined by extending the multiplication in $G$. We have
\[
	\left(\sum_{g\in G} a_g g \right) \cdot \left(\sum_{h\in G} b_h h \right)
	= \sum_{t \in G} c_t t
\]
with
\[
	c_t = \sum_{gh=t} a_gb_h.
\]
A $K[G]$-module is the same as a $K$-vector space $V$, together with a group homomorphism
\[
	G \to \GL_K(V) = \End_K(V)^\times.
\]
In other words, a $K[G]$-module is a $K$-linear representation of $G$.
\end{example}


\section{Homomorphisms, submodules and quotient modules}

\begin{definition} Let $M$ and $N$ be $R$-modules. An \emph{$R$-module homomorphism} from $M$ to $N$ is a map $f\colon M\to N$ such that for all $r\in R$ and $x,y\in M$ we have
\[
	f(x+y)=f(x)+f(y)
\]
and
\[
	f(rx) = r f(x).
\]
We also say that the map $f\colon M\to N$ is \emph{$R$-linear}. The set of $R$-module homomorphisms from $M$ to $N$ is denoted $\Hom_R(M,N)$. An \emph{isomorphism} of $R$-modules is a bijective $R$-module homomorphism. Two $R$-modules are called \emph{isomorphic} if there exists an  isomorphism between them.
\end{definition}




A \emph{submodule} of an $R$-module $M$ is a subgroup $N\subset M$ such that for all $r\in R$ and $x\in N$ we have $rx\in N$. A submodule of an $R$-module is itself an $R$-module. If $N\subset M$ is a submodule, then the abelian group $M/N$ has the structure of an $R$-module, via
\[
	r(x+N) := rx + N.
\]
We call $M/N$ the \emph{quotient module}.

To a module homomorphism $f\colon M \to N$ are associated three important modules. The \emph{kernel}
\[
	\ker f := \{ x\in M \mid f(x)=0 \} \subset M,
\]
which is a submodule of $M$, the \emph{image}
\[
	\im f := f(M) \subset N,
\]
which is a submodule of $N$, and the \emph{cokernel}
\[
	\coker f := N/(\im f),
\]
which is a quotient module of $N$. A homomorphism $f$ is injective if and only if $\ker f$ is trivial, and it is surjective if and only if $\coker f$ is trivial.

As with groups or vector spaces, we have the natural isomorphism
\[
	M/(\ker f) \isomto \im f,\, \bar{x} \mapsto f(x).
\]

\section{Products, direct sums and free modules}\label{sec:products-and-direct-sums}

Let $R$ be a ring and let $M$ and $N$ be $R$-modules. The cartesian product $M\times N$ is naturally an $R$-module with $(x_1,x_2)+(y_1,y_2)=(x_1+y_1,x_2+y_2)$ and $r(x_1,x_2)=(rx_1,rx_2)$. We call this $R$-module the \emph{product} or \emph{direct product} of the $R$-modules $M$ and $N$.  More generally, if $(M_i)_{i\in I}$ is a family of $R$-modules indexed by a set $I$, then the product $\prod_{i\in I} M_i$ is naturally an $R$-module with
\[
	(x_i)_{i\in I} + (y_i)_{i\in I} = (x_i+y_i)_{i\in I},\quad r(x_i)_{i\in I} = (rx_i)_{i \in I},
\]
for all $(x_i)_{i\in I}$,  $(y_i)_{i\in I}$ in $\prod_{i\in I} M_i$ and $r\in R$. The empty product gives the zero module.  

The \emph{direct sum} of a collection $(M_i)_{i\in I}$ of $R$-modules indexed by a set $I$, denoted $\bigoplus_{i\in I} M_i$ is the $R$-submodule of $\prod_{i\in I} M_i$ defined as
\[
	\bigoplus_{i\in I} M_i := \Big\{ (x_i)_{i\in I} \in \prod_{i\in I} M_i \,\mid\,
	\{i\in I \colon x_i\neq 0\} \text{ is finite}\Big\}.
\]
Note that this is indeed a submodule: if $(x_i)_{i\in I}$ and $(y_i)_{i\in I}$  have only finitely many non-zero terms, then so do $(x_i+y_i)_{i\in I}$ and $(rx_i)_{i\in I}$.  

One sometimes phrases the finiteness condition as  `$x_i$ is zero for all but finitely many $i$' or `$(x_i)_{i\in I}$ has finite support'. Of course, if $I$ is finite then the condition is vacuous and we have $\bigoplus_{i\in I} M_i = \prod_{i\in I} M_i$.

The direct sum comes equipped with `inclusion' maps
\[
	\iota_j\colon M_j \to \bigoplus_{i\in I} M_i,\, x \mapsto \iota_j(x),\quad
	\iota_j(x)_i = \begin{cases} x & i=j \\ 0 & i\neq j \end{cases}
\]
and the product comes equipped with `projection' maps
\[
	\pi_j\colon  \prod_{i\in I} M_i \to M_j,\, (x_i)_{i\in I} \mapsto x_j.
\]


If we have $M_i=M$ for all $i$, then we write $M^I := \prod_{i\in I} M$ and $M^{(I)} := \bigoplus_{i\in I} M$,
so that we have
\[
	M^{I} = \{ (x_i )_{i\in I} \,\mid\, x_i \in M \}
\]
and
\[
	M^{(I)} = \big\{ (x_i)_{i\in I} \,\mid\, x_i \in M, \text{ and $x_i=0$ for all but  finitely many $i$}\, \big\}.
\]
If $I$ is a finite set of cardinality $n$, then we have $M^{I} = M^{(I)} \cong M^n$.


\bigskip

Let $(x_i)_{i \in I}$ be a family of elements of an $R$-module $M$. Then we call the intersection of all submodules $N\subset M$ that  contain all $x_i$ the \emph{submodule generated by $(x_i)_{i\in I}$}. We denote it by $\langle x_i \rangle_{i \in I}$.  It is the smallest submodule of $M$ containing all the $x_i$. It consists of all finite $R$-linear combinations of the $x_i$.

We say that $M$ is  \emph{finitely generated} if there exists a finite family $(x_i)_{i\in I}$ with $M=\langle x_i \rangle_{i\in I}$.

\begin{example}\label{ex:free-module}
Let $R$ be a ring and $I$ a set. Consider the module
\[
	R^{(I)}  = \Big\{ (r_i)_{i\in I} \in R^I \mid  \text{  $r_i=0$ for all but finitely many $i\in I$} \Big\}.
\]
For an index $i\in I$ we denote by $e_i \in R^{(I)}$ the element $e_i := \iota_i(1)$. One may think of $e_i$ as the `standard basis' element
\[
	e_i = ( \ldots, 0, 0, 1, 0, 0, \ldots  )
\]
with a $1$ at position $i$.  Clearly, every element of $R^{(I)}$ is a finite $R$-linear combination of $e_i$'s, so the family $(e_i)_{i \in I}$ generates $R^{(I)}$. Note that if $I$ is infinite, then the $e_i$ do not generate the direct product $R^{I}$. 
\end{example}

\begin{proposition}\label{prop:universal-property-free-module}
Let $R$ be a ring, $M$ an $R$-module, and $(x_i)_{i \in I}$ a family of elements of $M$. Then there exists a unique 
$R$-linear map $\varphi\colon R^{(I)} \to M$ with $\varphi(e_i)=x_i$ for every $i\in I$. 

Moreover,  $\varphi$ is surjective if and only if $M$ is generated by $(x_i)_{i\in I}$.
\end{proposition}

This generalises the basic fact from linear algebra that giving a linear map $\bR^n\to V$ is the same as giving the images of the standard basis vectors.

\begin{proof}[Proof of Proposition \ref{prop:universal-property-free-module}]
The map $\varphi$ is given by
\[
	\varphi\colon R^{(I)} \to M,\, (r_i)_{i\in I} \mapsto \sum_{i\in I} r_i x_i
\]
(note that in the sum only finitely many terms are non-zero). This map is surjective if and only if every
element of $M$ can be written as a finite $R$-linear combination of elements $x_i$.
\end{proof}

\begin{definition}
Let $M$ be an $R$-module and $(x_i)_{i \in I}$ a family of elements of $M$. Let $\varphi\colon R^{(I)} \to M$ be the unique $R$-linear map with
$\varphi(e_i)=x_i$ for all $i\in I$. We say that $(x_i)_{i\in I}$ is a \emph{basis} of $M$ if $\varphi$ is an isomorphism. We say that an $R$-module $M$ is \emph{free} if it has a basis. If it has a basis of cardinality $n$, then we say that $M$ is \emph{free of rank $n$}.
\end{definition}

In particular, $M$ is free of rank $n$ if and only if $M\cong R^n$.

% TODO: maybe define free of rank n as isomorphic with R^n?

In contrast with the case of vector spaces (modules over a field), a finitely generated $R$-module need not have a basis. For example: for $m> 1$ the $\bZ$-module $\bZ/m\bZ$ does not have a basis, and hence is not free.

\begin{proposition}\label{prop:equivalent-def-basis}
Let $M$ be an $R$-module and $(x_i)_{i \in I}$ a family of elements of $M$. Then $(x_i)_{i\in I}$ is a basis of $M$ if and only if for every $x\in M$ there 
is a unique family $(r_i)_{i\in I}$ of elements in $R$ with
\begin{enumerate}
\item $r_i=0$ for all but finitely many $i$, and
\item   $x=\sum_{i\in I} r_i x_i$.
\end{enumerate}
\end{proposition}

Note that  the condition in (1) guarantees that only finitely many terms in the sum in (2) are non-zero.

\begin{proof}[Proof of Proposition \ref{prop:equivalent-def-basis}]
This is a direct translation of the definition: existence of $(r_i)_{i\in I}$ is equivalent with $x$ being in the image of $\varphi\colon R^{(I)}\to M$, and uniqueness is equivalent with $\varphi\colon R^{(I)} \to M$ being injective.
\end{proof}

\begin{example}
Let $U$ be an open subset of $\bR^n$. Then the space of $1$-forms $\Omega^1(U)$ on $U$ forms a module over the ring $\cC^\infty(U)$ of $\cC^\infty$ functions on $U$. This module is free of rank $n$, with basis ${\rm d} x_1, \ldots, {\rm d}x_n$.
\end{example}


\begin{proposition}\label{prop:invariant-basis-number}
Let $R$ be a commutative ring with $0\neq 1$. If $R^n$ and $R^m$ are isomorphic $R$-modules, then $n=m$.
\end{proposition}

In other words: an $R$-module over a non-zero commutative ring which is free of finite rank has a well-defined rank. We already know this if $R$ is a field (any two bases of a vector space have the same cardinality), and the proof of the proposition will be by reduction to the case of a field.

\begin{proof}[Proof of Proposition \ref{prop:invariant-basis-number}]
Let $M=R^m$, $N=R^n$ and let
\[
	\varphi\colon M \to N
\]
be an isomorphism. Let $I \subset R$ be a maximal ideal (which exists in every non-zero commutative ring). Then $\varphi$ induces an isomorphism
\[
	\bar\varphi\colon M/IM \to N/IN
\]
of $R/I$-modules (see also Exercise \ref{exc:module-by-ideal}). We have $M/IM = (R/I)^m$ and $N/IN = (R/I)^n$. But $R/I$ is a field, so using
the fact that a vector space has a well-defined dimension we find
\[
	m = \dim_{R/I} M/IM = \dim_{R/I} N/IN = n,
\]
which is what we had to prove.
\end{proof}

\newpage
\section*{Exercises}


\begin{exercise}\label{exc:zeroes}
Let $M$ be an $R$-module. Show that for every $r\in R$ and $x\in M$ the following identities in $M$ hold:
\begin{enumerate}
\item $r0=0$,
\item $0x=0$,
\item  $(-r)x=r(-x)=-(rx)$.
\end{enumerate}
\end{exercise}

\begin{exercise}
Let $R = \{0\}$ be the zero ring. Show that every $R$-module is the zero module.
\end{exercise}

\begin{exercise}\label{exc:left-module-via-end}
Prove Lemma \ref{lemma:left-module-via-end}.
\end{exercise}


\begin{exercise}Let $R=(R,0,1,+,\cdot)$ be a ring. Consider the opposite ring
$R^\opp=(R,0,1,+,\cdot^\opp)$ where multiplication is defined by
\[
	r \cdot^\opp s := s\cdot r.
\]
Show that  a \emph{right module} over $R$ is the same as an abelian group $M$ equipped with a ring homomorphism $R^\opp \to \End(M)$.
\end{exercise}

\begin{exercise}
Let $M$ be an $R$-module, and let $x_1,\ldots, x_n$ be elements of $M$. Verify that the map
\[
	R^n \to M,\, (r_1,\ldots, r_n) \mapsto r_1x_1 + \cdots + r_n x_n
\]
is a homomorphism of $R$-modules.
\end{exercise}

\begin{exercise}\label{exc:principal-ideal-in-domain}
Let $R$ be an integral domain and $I\subset R$ a non-zero principal ideal. Show that $I$, as an $R$-module, is isomorphic to the $R$-module $R$. 
\end{exercise}

\begin{exercise}Let $R$ be a ring, and let $M$ and $N$ be $R$-modules. Show that point-wise addition makes $\Hom_R(M,N)$ into an abelian group. Show that if $R$ is commutative, then $\Hom_R(M,N)$ has  a natural structure of $R$-module.
\end{exercise}

\begin{exercise}
Let $R$ be a ring. Show $\Hom_R(R,M) \cong M$ as abelian groups (and if $R$ is commutative, as $R$-modules).
\end{exercise}

\begin{exercise}\label{ex:vect-with-endo} Verify that the map (\ref{eq:vect-with-endo}) in Example \ref{exa:vect-with-endo} is indeed a ring homomorphism.
\end{exercise}

\begin{exercise}\label{exc:modules-over-multivariate-polynomial-ring}
Let $K$ be a field and $n$ a positive integer. Let $V$ be a $K$-vector space, and let $\alpha_1$, $\alpha_2$, \ldots, $\alpha_n$ be pairwise \emph{commuting} linear endomorphisms of $V$. Show that there is a unique ring homomorphism
\[
	\rho \colon K[X_1,\ldots,X_n] \to \End(V)
\]
such that for all $\lambda \in K$ and $v\in V$ we have $\rho(\lambda)(v)=\lambda v$ and for all $i$ we have $\rho(X_i)(v) = \alpha_i(v)$.

Convince yourself that a $K[X_1,\ldots,X_n]$-module is the same thing as a vector space together with $n$ pairwise commuting linear endomorphisms.
\end{exercise}



\begin{exercise}\label{exc:module-by-ideal}
Let $R$ be a ring and $I\subset R$ a (two-sided) ideal. Let $M$ be an $R$-module. Show that
\[
	IM := \{ \sum r_ix_i \mid r_i\in I, x_i\in M \}
\]
is a sub-$R$-module of $M$, and show that $M/IM$ is an $R/I$-module. Show that if $M$ is free of rank $n$ as $R$-module, then $M/IM$ is free of rank $n$ as $R/I$-module.
\end{exercise}

\begin{exercise}\label{exc:annihilator}
Let $R$ be a ring and let $M$ be a left $R$-module. Show that
\[
	\Ann_R( M ) := \{ r \in R \mid \text{ $rx=0$ for all $x\in M$ } \}
\]
is a (two-sided) ideal in $R$.
\end{exercise}


\begin{exercise}
Show that $\bQ$ is not a finitely generated $\bZ$-module.
\end{exercise}

\begin{exercise}
Let $K$ be a field and $I$ a countably infinite set. Show that the $K$-module $K^{I}$ does not have a countable generating set.
\end{exercise}


\begin{exercise}
Show that $\bQ$ is not a free $\bZ$-module.
\end{exercise}

\begin{exercise}Prove the following generalisation of Proposition \ref{prop:universal-property-free-module}: Let $R$ be a ring and let $(M_i)_{i\in I}$ be a collection of $R$-modules indexed by a set $I$. Let $N$ be an $R$-module, and let $(f_i\colon M_i \to N)_i$ be a collection of $R$-linear maps. Then there exists a unique $R$-linear map
\[
	f\colon \bigoplus_{i\in I} M_i \to N
\]
such that for all $i\in I$ and $x\in M_i$ we have $f(\iota_i(x)) = f_i(x)$.
\end{exercise}


\begin{exercise}\label{exc:universal-property-direct-sum}
Let $R$ be a ring, $(M_i)_{i\in I}$ a family of $R$-modules, and $N$ an $R$-module. Show that there are
isomorphisms
\[
	\Hom_R(\bigoplus_{i\in I} M_i, N) \longisomto \prod_{i\in I} \Hom_R(M_i,N)
\]
and
\[
	\Hom_R(N, \prod_{i\in I} M_i) \longisomto \prod_{i\in I} \Hom_R(N,M_i)
\]
of abelian groups.
\end{exercise}

\begin{exercise}[$\star$]\label{exc:no-well-defined-rank}
Let $K$ be a field. Let $R$ be the set of $\infty$ by $\infty$ matrices 
\[
\left(\begin{array}{ccc}a_{11} & a_{12} & \cdots \\a_{21} & a_{22} & \cdots \\\vdots & \vdots & \end{array}\right)
\]
with $a_{ij} \in K$, and with the property that every column contains only finitely many non-zero elements. Verify that $R$ (with the usual rule for matrix multiplication) is a ring. Show that the left $R$-modules $R$ and $R\oplus R$ are isomorphic. Conclude that the condition that $R$ is commutative cannot be dropped from Proposition \ref{prop:invariant-basis-number}.
\end{exercise}


%%%%%%%%%%%%%%%%%%
% EXACT SEQUENCES
%%%%%%%%%%%%%%%%%%

\chapter{Exact sequences}

Exact sequences form a useful and extensively used notational tool in algebra. They allow to replace tedious and verbose arguments involving kernels and quotients by quick and intuitive `diagram chases'. 


\section{Exact sequences}


If $f\colon M_1\to M_2$ and $g\colon M_2\to M_3$ are $R$-module homomorphisms, then we say that the sequence
\[
	M_1 \overset{f}{\longto} M_2 \overset{g}{\longto} M_3
\]
is \emph{exact} if and only if the image of $f$ is the kernel of $g$, as submodules of $M_2$. For example: the sequence
\[
	0 \longto M \longto N
\]
is exact if and only if $M\to N$ is injective, and the sequence
\[
	M \longto N \longto 0
\]
is exact if and only if $M\to N$ is surjective. 

A general sequence
\[
	\cdots \longto M_{i-1} \overset{f_{i-1}}{\longto} M_i \overset{f_{i}}{\longto}  M_{i+1} \longto \cdots
\]
is called exact if for every $i$ we have $\ker f_i = \im f_{i-1}$ as submodules of $M_i$.  


An exact sequence of the form
\[
	0 \longto M_1 \overset{f}{\longto} M_2 \overset{g}{\longto} M_3 \longto 0
\]
is called a \emph{short exact sequence}. Note that $f$ induces an isomorphism
\[
	M_1 \cong \ker g
\]
and $g$ induces an isomorphism
\[
	\coker f  \cong M_3.
\]
We will often interpret the injective map $f$ as the inclusion of a submodule $M_1$ into $M_2$, and $M_3$ as the quotient of $M_2$ by the submodule $M_1$, so that we can think of any short exact sequence as 
a sequence of the type
\[
	0 \longto M_1 \longto M_2 \longto M_2/M_1 \longto 0.
\]

\section{The Five Lemma and the Snake Lemma}

The Five Lemma and Snake Lemma are powerful and often-used lemmas about modules that are hard to state (and even harder to prove) without the language of commutative diagrams and exact sequences. The proofs are classic examples of `diagram chasing'. Such arguments are often fairly easy to verify by tracing elements around the diagram on the blackboard or a piece of paper, but are sometimes headache-provokingly resistant to being rendered or read in prose. 


\begin{theorem}[Five Lemma]\label{thm:five-lemma}
Let $R$ be a ring. Consider a commutative diagram of $R$-modules
\[
\begin{tikzcd}
	M_1 \arrow{r} \arrow{d}{f_1}
	& M_2 \arrow{r} \arrow{d}{f_2}
	& M_3 \arrow{r} \arrow{d}{f_3}
	& M_4 \arrow{r} \arrow{d}{f_4}
	& M_5  \arrow{d}{f_5} \\
	N_1 \arrow{r}
	& N_2 \arrow{r}
	& N_3 \arrow{r}
	& N_4 \arrow{r}
	& N_5 
\end{tikzcd}
\]
with exact rows. If $f_1$, $f_2$, $f_4$, $f_5$ are isomorphisms, then so is $f_3$.
\end{theorem}

In fact, the proof will show that it suffices to assume that $f_1$ is surjective, $f_5$ is injective, and $f_2$ and $f_4$ are isomorphisms.


\begin{proof}The proof consists of two parts, one showing that $f_3$ is injective, the other that it is surjective. Both parts require only part of the hypotheses in the theorem.

\emph{Claim}. If $f_1$ is surjective, and $f_2$ and $f_4$ are injective, then $f_3$ is injective.

Indeed, assume $f_3(x)=0$ for some $x\in M_3$. We need to show that $x=0$. Let $x'\in M_4$ be the image of $x$. By the commutativity of the diagram, $f_4(x')=0$. But $f_4$ was injective, hence $x'=0$. It follows that $x\in M_3$ is the image of some element $y\in M_2$.  By commutativity, $f_2(y)$ maps to zero in $N_3$, hence $f_2(y)$ is the image of some element $z \in N_1$. By the assumption on $f_1$, there is a $\tilde{z} \in M_1$ with $f_1(\tilde{z})=z$. 

Consider the image $y'$ of $\tilde{z}$ in $M_2$. By commutativity, we have $f_2(y')=f_2(y)$, but since $f_2$ is injective, this implies $y'=y$. We see that $x\in M_3$ is the image of some element $\tilde{z}$ in $M_1$, and hence by exactness we conclude $x=0$.

\emph{Claim}. If $f_5$ is injective, and $f_2$ and $f_4$ are surjective, then $f_3$ is surjective.
The proof of this second claim is left to the reader, see Exercise \ref{exc:five-lemma-part-2}.

The theorem follows immediately from the above two claims.
\end{proof}


\begin{theorem}[Snake Lemma]\label{thm:snake-lemma}
Let $R$ be a ring. Let
\[
\begin{tikzcd}
 0 \arrow{r} 
 	& M_1 \arrow{r}{\alpha} \arrow{d}{f_1}
 	& M_2 \arrow{r}{\beta} \arrow{d}{f_2}
 	& M_3 \arrow{r} \arrow{d}{f_3}
	& 0 \\
0 \arrow{r}
	& N_1 \arrow{r}{\alpha'}
	& N_2 \arrow{r}{\beta'}
	& N_3 \arrow{r}
	& 0
\end{tikzcd}
\]
be a commutative diagram in which both horizontal rows are short exact sequences. Then there is 
a commutative diagram
\[
\begin{tikzcd}
0 \arrow{r} 
 	& \ker f_1 \arrow{d} \arrow{r}
 	& \ker f_2 \arrow{d} \arrow{r}
 	& \ker f_3 \arrow{d} \arrow[out=-5, in=175]{dddll}
	&  \\
 0 \arrow{r} 
 	& M_1 \arrow{r}{\alpha} \arrow{d}{f_1}
 	& M_2 \arrow{r}{\beta} \arrow{d}{f_2}
 	& M_3 \arrow{r} \arrow{d}{f_3}
	& 0 \\
0 \arrow{r}
	& N_1 \arrow{r}{\alpha'} \arrow{d}
	& N_2 \arrow{r}{\beta'} \arrow{d}
	& N_3 \arrow{r} \arrow{d}
	& 0 \\
	& \coker f_1 \arrow{r}
	& \coker f_2 \arrow{r}
	& \coker f_3 \arrow{r} & 0
\end{tikzcd}
\]
of $R$-modules in which the maps $\ker f_i\to M_i$ and $N_i \to \coker f_i$ are the natural inclusions and projections, and
in which the sequence
\[
	0 \to \ker f_1 \to \ker f_2 \to \ker f_3 \to \coker f_1 \to \coker f_2 \to \coker f_3 \to 0
\]
is exact.
\end{theorem}



\begin{proof}[Sketch of proof]
We only give the most interesting part of the proof: the construction of the `snake' map
\[
	d\colon \ker f_3 \to \coker f_1.
\]

Let $x\in \ker f_3 \subset M_3$.  Since the map $\beta\colon M_2\to M_3$ is surjective, there is a $y\in M_2$ with $\beta(y)=x$. By the commutativity of the right square, we have
\[
	\beta'(f_2(y)) = f_3(\beta(y)) = f_3(x) = 0.
\]
So $f_2(y) \in \ker \beta' = \im \alpha'$, hence there is a $z\in N_1$ with $\alpha'(z)=f_2(y)$. Note that $z$ is unique, as the map $\alpha'$ is injective. We define $d(x)$ as the element $\bar{z} \in \coker f_1 = N_1/f_1(M_1)$.

We must check that this is well-defined, since our construction depended on the choice of $y\in M_2$ with $\beta(y)=x$. Let $y'\in M_2$ be another element with $\beta(y')=x$, leading to a $z'\in N_1$ as above.
Since $\beta(y'-y)=x-x=0$ there is a unique $\delta \in M_1$  with $y' -y = \alpha(\delta)$. Now the commutativity of the diagram shows $z'-z=f_1(\delta)$ in $N_1$, and hence $\bar{z}' = \bar{z}$ in $N_1/f_1(M_1)$, as we had to show.
\end{proof}

\section{Split short exact sequences}

Let $M$ and $N$ be $R$-modules. Then their direct sum fits into a short exact sequence
\[
	0 \longto M \overset{i}{\longto} M\oplus N \overset{p}{\longto} N \longto 0
\]
where the maps are the natural inclusion $i\colon x\mapsto (x,0)$, and projection $p\colon (x,y)\mapsto y$. It is often convenient to be able to recognize if a given short exact sequence is of the above special form.

\begin{theorem}[Splitting lemma]\label{thm:splitting-lemma}
Let
\[
	0 \longto M_1 \overset{f}{\longto} M_2 \overset{g}{\longto} M_3 \longto 0
\]
be an exact sequence of $R$-modules. Then the following are equivalent:
\begin{enumerate}
\item there is a homomorphism $h\colon M_2 \to M_1$ such that $hf=\id_{M_1}$
\item there is a homomorphism $s\colon M_3 \to M_2$ such that $gs=\id_{M_3}$
\item there is an isomorphism $\varphi\colon M_2 \isomto M_1 \oplus M_3$ such that the diagram
\[
\begin{tikzcd}
0 \arrow{r} & M_1 \arrow{r}{f} \arrow{d}{\id} 
	& M_2 \arrow{r}{g} \arrow{d}{\varphi} 
	& M_3 \arrow{r} \arrow{d}{\id} & 0 \\
0 \arrow{r} & M_1 \arrow{r}{i} & M_1 \oplus M_3 \arrow{r}{p} & M_3 \arrow{r} & 0
\end{tikzcd}
\]
commutes (where $i(x)=(x,0)$ and $p(x,y)=y$.)
\end{enumerate}
\end{theorem}

If these conditions hold, we say that the sequence is \emph{split} or \emph{split exact}. The map $h$ is called a \emph{retraction} of $f$, and the map $s$ a \emph{section} of $g$. See Exercise \ref{exc:nonsplit-examples} for examples of short exact sequences that are not split.

\begin{proof}[Proof of Theorem \ref{thm:splitting-lemma}]
We first show that (3) implies (2). Indeed, the map
\[
	M_3 \to M_2,\, y \mapsto \varphi^{-1}(0,y)
\]
is a section of $g$.

Next, we show that (2) implies (1), so assume that $s\colon M_3\to M_2$ is a section. We will construct a retraction $h\colon M_2 \to M_1$. Let $x\in M_2$. Consider the element
\[
	y := x - s(g(x)) \in M_2.
\]
Then, since $gs=\id$ we have
\[
	g(y) = g(x) - g(s(g(x))) = g(x) - g(x) = 0.
\]
So $y \in \ker g = \im f$, and since $f$ is injective, there is a {unique} $z\in M_1$ with $f(z)=y$. Define $h(x) := z$. One checks that $h$ is indeed a retraction.

Finally, to show that (1) implies (3), assume that $h\colon M_2 \to M_1$ is a retraction. Then consider the map
\[
	\varphi\colon M_2 \to M_1 \oplus M_3,\,
	x \mapsto (h(x), g(x)).
\]
Note that this map is an $R$-module homomorphism. We verify that it makes the diagram commute. We start with the left square. Take an $x\in M_1$ in the left-top corner of this square. Going down and then right, it gets mapped to $i(\id (x)) = (x,0) \in M_1 \oplus M_3$. Following the other path, we end up with
$ \varphi(f(x))= (h(f(x)),g(f(x)))$. But now, $h(f(x))=x$ because $h$ is a retraction, and $g(f(x))=0$ because $\im f = \ker g$ by the hypothesis that the sequence is exact. We conclude that $\varphi(f(x))=(x,0)$ and that the left square indeed commutes. For the right-hand square, take $x\in M_2$. Then one path yields $\id(g(x)) = g(x)$, and following the other path, we obtain $p(\varphi(x))=p(h(x),g(x))=g(x)$. These agree, so we conclude that the diagram indeed commutes. Finally, by Exercise \ref{exc:morphism-of-extensions-is-isomorphism} we see that the map $\varphi$ is automatically an isomorphism, which shows that (3) indeed follows from (1).
\end{proof}





% SECTION: EXERCISES

\newpage
\section*{Exercises}




\begin{exercise}Show that
\[
	0 \longto M \longto 0
\]
is exact if and only if $M$ is the zero module.
\end{exercise}

\begin{exercise}Let $f\colon M\to  N$ be an $R$-module homomorphism. Show that there is an exact sequence
\[
	0 \longto \ker f \longto M \overset{f}{\longto} N \longto \coker f \longto 0
\]
of $R$-modules.
\end{exercise}

\begin{exercise}\label{exc:five-lemma-part-2}
Complete the proof of the Five Lemma (Theorem \ref{thm:five-lemma}): show that if $f_2$ and $f_4$ are surjective, and if $f_5$ is injective, then $f_3$ is surjective. 
\end{exercise}

\begin{exercise}
Let
\[
\begin{tikzcd}
	M_1 \arrow{r} \arrow{d}{f_1}
	& M_2  \arrow{r} \arrow{d}{f_2}
	& M_3 \arrow{r} \arrow[dashed]{d}{}  & 0 \\
 N_1 \arrow{r} & N_2 \arrow{r} & N_3 \arrow{r} & 0
\end{tikzcd}
\]
be a commutative diagram of $R$-modules with exact rows. Show that there exists a unique $R$-linear map $f_3\colon M_3 \to N_3$ making the resulting diagram commute.
\end{exercise}

\begin{exercise}\label{exc:morphism-of-extensions-is-isomorphism}
Consider a commutative diagram of $R$-modules
\[
\begin{tikzcd}
0 \arrow{r}
	& M \arrow{r}{\alpha} \arrow{d}{\id_M}
	& E \arrow{r} \arrow{d}{f}
	& N \arrow{r} \arrow{d}{\id_N} & 0 \\
0 \arrow{r} & M \arrow{r} & E' \arrow{r} & N \arrow{r} & 0
\end{tikzcd}
\]
in which both rows are short exact sequences. Deduce from the snake or five lemma that $f$ must be an isomorphism. Show that $f$ is  an isomorphism without using the snake or five lemma. (Hint for surjectivity: given $y\in E'$ choose an $x\in E$ with same image as $y$ in $N$. Show that there is an $z\in M$ with $f(\alpha(z)+x)=y$.)
\end{exercise}

\begin{exercise}
Give an example of a diagram as in Theorem \ref{thm:snake-lemma}, for which the `snake map' $d\colon \ker f_3 \to \coker f_1$ is non-zero.
\end{exercise}


\begin{exercise}\label{exc:square-snake}
Let $R$ be a ring and let
\[
\begin{tikzcd}
M_1 \arrow[hook]{r} \arrow{d}{\alpha_1} & M_2 \arrow{d}{\alpha_2} \\
N_1 \arrow[hook]{r} & N_2 
\end{tikzcd}
\]
be a commutative diagram of $R$-modules, in which the two horizontal maps are injective. Show that there exists an $R$-module $E$ and an exact sequence
\[
	0 \longto \ker \alpha_1 \longto \ker \alpha_2 \longto E \longto \coker \alpha_1 \longto \coker \alpha_2
\]
of $R$-modules.
\end{exercise}


\begin{exercise} \label{exc:covariant-short-exact-hom}
Let $R$ be a ring,  let
\begin{equation}\label{eq:exc-short-exact-hom}
	0 \longto M_1 \longto M_2 \longto M_3 
\end{equation}
be an exact sequence of $R$-modules, and let $N$ be an $R$-module. Show that there 
is an exact sequence of abelian groups
\[
	0 \longto \Hom_R(N,M_1) \longto \Hom_R(N,M_2) \longto \Hom_R(N,M_3).
\]
Give an example to show that the exactness of $0\to M_1 \to M_2\to M_3\to 0$ need not imply
that the map $\Hom_R(N,M_2)\to \Hom_R(N,M_3)$ is surjective.
\end{exercise}

\begin{exercise}\label{exc:contravariant-short-exact-hom}
Let $R$ be a ring,  let
\[
	 M_1 \longto M_2 \longto M_3 \longto 0
\]
be an exact sequence of $R$-modules, and let $N$ be an $R$-module. Show that there 
is an exact sequence of abelian groups
\[
	0 \longto \Hom_R(M_3,N) \longto \Hom_R(M_2,N) \longto \Hom_R(M_1,N).
\]
Give an example to show that the exactness of $0\to M_1 \to M_2\to M_3\to 0$ need not imply
that the map $\Hom_R(M_2,N)\to \Hom_R(M_1,N)$ is surjective.
\end{exercise}


\begin{exercise}
Let $I$ and $J$ be left ideals in a ring $R$. Show that there are exact sequences
\[
	0 \to I \cap J \to I \oplus J \to I + J \to 0
\]
and
\[
	0 \to R/(I\cap J) \to R/I \oplus R/J \to R/(I+J) \to 0
\]
of $R$-modules. 
\end{exercise}



\begin{exercise}
Let $R$ be a commutative ring and let
\[
	0 \longto M \longto E \longto N \longto 0
\]
be a short exact sequence of $R$-modules. Let $I:=\Ann_R M$ and $J := \Ann_R N$ (see Exercise \ref{exc:annihilator}). Show that
\[
	IJ \subset \Ann_R E \subset I \cap J
\]
as ideals in $R$.
\end{exercise}



\begin{exercise}\label{exc:nonsplit-examples}
Show that the short exact sequences of $\bZ$-modules
\[
	0 \longto \bZ \overset{2}{\longto} \bZ \longto \bZ/2\bZ \longto 0
\]
and
\[
	0 \longto \bZ/2\bZ \overset{2}{\longto} \bZ/4\bZ \longto \bZ/2\bZ \longto 0
\]
are \emph{not} split. (Here the `$2$' above the arrows are shorthand for the maps $x\mapsto 2x$ and $\bar{x}\mapsto \overline{2x}$.)
\end{exercise}


\begin{exercise}Let $K$ be a field. Show that every short exact sequence of $K$-modules is split exact.
\end{exercise}

\begin{exercise}
Let $G$ be a finite group, and $K$ a field of characteristic zero. \emph{Maschke's theorem} asserts that for every representation $V$ of $G$ over $K$, and for every $G$-stable subspace $W\subset V$ there exists a $G$-stable complement $U\subset V$. 
Show that every short exact sequence of $K[G]$-modules is split. 
\end{exercise}

\begin{exercise}[$\star$]
Let $G$ be the cyclic group of $2$ elements. Consider the group ring $R:=\bF_2[G]$. Give an example of a non-split short exact sequence of $R$-modules. Show that not every representation of the group $G$ over the field $\bF_2$ is isomorphic to a direct sum of irreducible representations. 
\end{exercise}


\begin{exercise}Let $K$ be a field. Consider the subring
\[
	R := \left\{ \left(\begin{matrix} a & b \\ 0 & c \end{matrix}\right) \mid a,b,c\in K \right \}
\]
of the ring $\Mat(2,K)$ of two-by-two matrices over $K$.
Let $M$ be the module of column vectors $x \choose {y}$ on which $R$ acts by the usual matrix multiplication:
\[
\left(\begin{matrix} a & b \\ 0 & c \end{matrix}\right)
\left(\begin{matrix} x  \\ y \end{matrix}\right) =
\left(\begin{matrix} ax + by  \\ cy \end{matrix}\right)
\]
Show that $N:= \{ {x \choose 0} \mid x \in K \} \subset M$ is a sub-$R$-module, and that the short exact sequence
of $R$-modules
\[
	0 \longto N \longto M \longto M/N \longto 0
\]
does not split.
\end{exercise}

\begin{exercise}\label{exc:free-module-split-short-exact-sequence}
Let $R$ be a ring and let $M$ and $N$ be $R$-modules. Show that any short exact sequence of the form
\[
	0 \longto M \longto N \longto R^n \longto 0
\]
is split. 
\end{exercise}

\begin{exercise}
Let $R$ be a ring and let $I\subset R$ be a left ideal. Show that a short exact sequence
of left $R$-modules of the form
\[
	0 \longto M \longto N \overset{\pi}{\longto} R/I \longto 0
\]
splits if and only if there exists an $x\in N$ with $\pi(x)=1+I$ and $rx=0$ for all $r\in I$.
\end{exercise}



\begin{exercise}\label{exc:hom-of-split-exact-seq}
Let
\[
	0\longto M_1 \longto M_2 \longto M_3 \longto 0
\]
be a split short exact sequence of $R$-modules, and let $N$ be an $R$-module. Show that the induced sequences
\[
	0 \longto \Hom_R(N,M_1) \longto \Hom_R(N,M_2) \longto \Hom_R(N,M_3) \longto 0
\]
and
\[
	0 \longto \Hom_R(M_3,N) \longto \Hom_R(M_2,N) \longto \Hom_R(M_1,N) \longto 0
\]
are exact.
\end{exercise}


\begin{exercise}
Let $R$ be a ring and let 
\[
	0 \longto M_1 \longto M_2 \longto M_3 \longto 0
\]
be a short exact sequence of $R$-modules. Show that if $M_1$ is free of rank $n_1$ and $M_3$ is free of rank $n_3$, then $M_2$ is free of rank $n_1+n_3$.
\end{exercise}
%
%\begin{exercise}[$\star$]
%Let $R$ be a non-zero commutative ring, and let
%\[
%	0 \longto M_1 \longto M_2  \longto \cdots \longto M_n \longto 0
%\]
%be an exact sequence of $R$-modules. Assume that $M_i$ is free of rank $n_i$. Show that
%the equality 
%\[	
%	\sum_i (-1)^i n_i = 0
%\]
%holds.
%\end{exercise}
%


%%%%%%%%%%%%%%%%%%%%%%%%%%%%%%%%%%%%%%%%%%
% CHAPTER: FINITELY GENERATED MODULES OVER A PID %
%%%%%%%%%%%%%%%%%%%%%%%%%%%%%%%%%%%%%%%%%%

\chapter{Finitely generated modules over a PID}\label{ch:modules-over-PID}

\section{Introduction}



The classification of finite abelian groups states that for every finite abelian group $A$ there are prime numbers $p_i$ (not necessarily distinct) and exponents $e_i\geq 1$ such that
\[
	A \cong (\bZ/p_1^{e_1}\bZ) \times \cdots \times (\bZ/p_n^{e_n}\bZ).
 \]
 
 The existence of Jordan normal forms states that for every square matrix $P$ over $\bC$ there exist
 complex numbers $\lambda_i$ (not necessarily distinct) and integers $e_i \geq 1$ so that
 $P$ is conjugate to a block diagonal matrix with blocks
\[
\left(\begin{matrix} \lambda_i & 0 &   \cdots & 0 & 0 \\ 
	1 & \lambda_i &  \cdots & 0 & 0 \\
	\vdots & \vdots &  & \vdots & \vdots \\ 
	0 & 0 &  \cdots & \lambda_i & 0 \\
	0 & 0 &   \cdots & 1& \lambda_i \end{matrix} \right)
\] 
of size $e_i$.
 
In this chapter, we will see that these two theorems are just two instances of one and the same theorem about finitely generated modules over a principal  ideal domain. In the first case, the PID will be $\bZ$, in the second case it will be $\bC[X]$.



\section{Review of principal ideal domains}

Let $R$ be a PID (principal ideal domain). Recall that this means that $R$ is an integral domain (a nonzero commutative ring without zero divisors), and that for every ideal $I\subset R$ there is an $r\in R$ with $I=(r)=Rr$. For our purposes, the most important examples are $R=\bZ$ and $R=K[X]$ with $K$ a field. Other important examples are the ring of power series $K[[X]]$, the ring of $p$-adic integers $\bZ_p$, and the ring of Gaussian integers $\bZ[i]$. Also a field $K$ is a PID, but of a rather trivial kind.

PID's are unique factorization domains. This means that every non-zero element $r$ in a principal ideal domain $R$ can be written as
\[
	r = u p_1^{e_1} \cdots p_n^{e_n}
\]
with $u \in R^\times$, with the $p_i \in R$ irreducible, and with $e_i$ non-negative integers. Moreover, such factorization is unique up to multiplying $u$ and the $p_i$'s by units, and up to permuting the factors.

Using this prime factorization, we can define greatest common divisors  $\gcd(r,s)$ of two non-zero elements of $R$. They are uniquely determined up to units (for example both $2$ and $-2$ are a gcd of $4$ and $6$ in $\bZ$). Similarly, we can define gcds of any sequence $r_1,\ldots,r_n$ of elements of $R$ which is not identically zero (ignoring the zeroes in the sequence).

An element $d\in R$ is a gcd of $r_1$, \ldots, $r_n$ if and only if $d$ generates the ideal $(r_1,\ldots, r_n)$ of $R$. In particular, there are $a_1,\ldots, a_n$ in $R$ with $d=a_1r_1+\cdots+a_nr_n$. For example, if $r$ and $s$ are coprime (have no common prime factor), then there are $a$ and $b$ in $R$ with $ar+bs=1$.


\section{Free modules of finite rank over a PID}



\begin{proposition}\label{prop:submodule-of-free-module-over-PID}
Let $R$ be a PID and $M\subset R^n$ a submodule. Then $M\cong R^k$ for some $k\leq n$.
\end{proposition}

Hence over a PID a submodule of a free module of finite rank is itself free of finite rank.   
The condition that $R$ be a PID cannot be dropped from the proposition, see Exercise \ref{exc:non-free-submodule}. 

% TODO: true without finite generation? Useful for Ext^n later...

\begin{proof}[Proof of Proposition \ref{prop:submodule-of-free-module-over-PID}]
We use induction on $n$. For $n=0$ we have $M=R^n=0$, and $M$ is indeed free of rank $0$. Assume that the proposition has been shown to hold for submodules of $R^{n-1}$. Consider the projection
\[
	\pi\colon R^n \to R,\, (r_1,\ldots, r_n) \to r_n
\]
with kernel $R^{n-1}\times \{0\}$. Then we have a short exact sequence
\[
	0 \longto M \cap \ker \pi \longto M \longto \pi(M) \longto 0.
\] 
By the induction hypothesis, we have that the submodule $M\cap \ker \pi$ of $R^{n-1}\times \{0\}$ is free of rank $k\leq n-1$. If $\pi(M)=0$, then we are done. If $\pi(M)\neq 0$, then $\pi(M)$ is an ideal in $R$,
hence a principal ideal, hence $\pi(M)\cong R$ as $R$-module. By Exercise \ref{exc:free-module-split-short-exact-sequence}
any short exact sequence of $R$-modules of the form
\[
	0 \longto N \longto M \longto R \longto 0
\]
splits, and we find $M \cong R \oplus (M \cap \ker \pi) \cong R^{k+1}$ with $k+1\leq n$.
\end{proof}


\begin{corollary}\label{cor:free-presentation}
Let $R$ be a PID and $M$ a finitely generated $R$-module. Then there exists an exact sequence 
\[
	0 \longto F_1 \longto F_2 \longto M \longto 0
\]
with $F_1$ and $F_2$ free $R$-modules of finite rank.
\end{corollary}

\begin{proof}
Since $M$ is finitely generated, by Proposition \ref{prop:universal-property-free-module} there is a surjection $F_2 \to M$ with $F_2$ a free module of finite rank. The kernel $F_1\subset F_2$ is also free of finite rank, thanks to Proposition \ref{prop:submodule-of-free-module-over-PID}.
\end{proof}


\section{Structure of finitely generated modules over a PID}

The main theorem of this chapter is the following structure theorem.

\begin{theorem}\label{thm:structure-fg-mod-over-PID}
Let $R$ be a principal ideal domain and $M$  a finitely generated $R$-module. Then there exists an integer $n$ and non-zero ideals $I_1$,  \ldots,  $I_k$ of $R$ such that
\[
	M \cong R^n \oplus R/I_1 \oplus \cdots \oplus R/I_k
\]
as $R$-modules.
\end{theorem}

For the proof we need the notion of \emph{content} of an element of a module.
Let $R$ be a commutative ring and let $M$ be an $R$-module. An element $x\in M$ determines a map
\begin{equation}\label{eq:content-evaluation}
	\Hom_R(M,R) \to R,\,f \mapsto f(x).
\end{equation}
This map is $R$-linear, hence the image is an ideal. We denote it by $c_M(x)$, and call it the \emph{content} of $x$.

\begin{lemma}\label{lemma:non-zero-content}
If $M$ is free of finite rank, and $x\in M$ is non-zero, then $c_M(x)$ is a non-zero ideal in $R$.
\end{lemma}

\begin{proof}
Without loss of generality, we may assume $M=R^n$ and $x=(x_1,\ldots, x_n)$. Since $x$ is non-zero, there exists an $i$ with $x_i\neq 0$. Consider the map
\[
	\pr_i \colon R^n \to R,\, (y_1,\ldots, y_n) \mapsto y_i.
\]
We have $\pr_i\in \Hom_R(M,R)$ and $\pr_i(x)\neq 0$, hence $c_M(x)\neq 0$.
\end{proof}

Note that the coordinates $x_i$ of $x\in M \cong R^n$ depend on the choice of basis of $M$, but that the content ideal $c_M(x)$ is independent of such choice!

\begin{lemma}\label{lemma:content-division}
Let $R$ be a principal ideal domain, $M$ a free $R$-module of finite rank, and $x\in M$ a non-zero element.  Then
there is a surjective $R$-linear map $f\colon M\to R$ such that $f(x)\in R$ generates the ideal $c_M(x)\subset R$.
\end{lemma}

\begin{proof}
Since $R$ is a principal ideal domain, there exists an $f\in \Hom_R(M,R)$ such that $f(x)$ generates the ideal $c_M(x)$. Moreover, by Lemma \ref{lemma:non-zero-content} we have $f(x)\neq 0$.

Consider the image of $f\colon M\to R$. This is an ideal $I\subset R$, and it suffices to show that $I=R$. As $R$ is a PID, the ideal $I$ is generated by some element $r\in R$. For every $y \in M$ we have $f(y)\in I$ and (since $R$ is an integral domain) there exists a unique $g(y)\in R$ such that
\[
	f(y) = g(y) \cdot r.
\]
This defines an $R$-linear map $g\colon M\to R$. Now consider the element $g(x)\in R$. By definition of $c_M(x)$ we have $g(x)\in c_M(x)$. Since $f(x)$ generates $c_M(x)$ there is an $s\in R$ with $g(x)=sf(x)$. But we also have $f(x)=rg(x)$, hence $rs=1$, hence $I=R$ and $f$ is surjective.
\end{proof}


\begin{proof}[Proof of Theorem \ref{thm:structure-fg-mod-over-PID}]
By Corollary \ref{cor:free-presentation} every finitely generated $R$-module $M$ sits in a short exact sequence
\[
	0 \longto F_1 \longto F_2 \longto M \longto 0
\]
with the $F_i$ free $R$-modules of finite rank. We will interpret $F_1$ as a submodule of $F_2$.

The proof goes by induction on the rank of $F_1$. If $F_1$ is zero, then $M\cong F_2\cong R^n$ for some $n$, and the theorem holds.

Otherwise, let  $x\in F_1$  be a non-zero element whose content $I := c_{F_2}(x)$ with respect to $F_2$ is maximal (amongst all the ideals in $R$ of the form $c_{F_2}(x)$ with $x\in F_1$). 

By Lemma \ref{lemma:content-division} there is a surjective map $f\colon F_2 \to R$ such that $f(x)$ generates $I$. 

\emph{Claim}. $f(F_1) = I \subset R$. Indeed, since $f(x)$ generates $I$ we certainly have $f(F_1)\supset I$. Conversely, for $z\in F_1$, let $d$ be a gcd of $f(z)$ and $f(x)$. Then there exists $r, s\in R$ with $rf(z)+sf(x)=d$, and hence $f(rz+sx)=d$. By the maximality of the content of $x$ we have that $d$ must equal $f(x)$ up to a unit, hence $f(z)$ must be divisible by $f(x)$ and hence $f(z)\in I$.

Denote the kernel of $f\colon F_2 \to R$ by $F_2'$ (note that by Proposition \ref{prop:submodule-of-free-module-over-PID}, this is also a free module). We have a short exact sequence
\[
	0 \longto F_2' \longto F_2 \overset{f}\longto R \longto 0.
\]
Similarly, let $F_1'$ be the kernel of the restriction $f\colon F_1 \to R$. We find a short exact `sub-sequence'
\[
	0 \longto F_1' \longto F_1 \overset{f}\longto I \longto 0.
\]
Since $f(x)$ generates $I = c_{F_2}(x)$ there is a $y\in F_2$ with $f(x)y = x$. Now 
the first sequence splits by the section $R\to F_2,\, 1\mapsto y$, and this section restricts to a section
$I\to F_1$ in the second short exact sequence.

We find $M\cong M' \oplus R/I$ with $M'$ given by the short exact sequence
\[
	0 \longto F_1' \longto F_2' \longto M' \longto 0.
\]
Since $F_1'$ has lower rank, the induction hypothesis guarantees
\[
	M' \cong R^{n'} \oplus R/I_1 \oplus \cdots \oplus R/I_k,
\]
which finishes the proof.
\end{proof}


\section{Application to Jordan normal form}


Theorem \ref{thm:structure-fg-mod-over-PID} has the following corollary.

\begin{corollary}\label{cor:structure-fg-mod-over-PID}
Let $R$ be a PID and $M$  a finitely generated $R$-module. Then there exists an integer $n$, irreducible elements $p_1,\ldots,p_k$ of $R$ and positive integers $e_1$, \ldots, $e_k$ such that
\[
	M \cong R^n \oplus R/p_1^{e_1}R \oplus \cdots \oplus R/p_k^{e_k}R
\]
as $R$-modules.
\end{corollary}

\begin{proof}
By Theorem \ref{thm:structure-fg-mod-over-PID} it suffices to show that for every non-zero ideal $I$  the $R$-module $R/I$ can be written in the desired form. Let $x$ be a generator of $I$, and consider its prime factorization
\[	
	x = u p_1^{e_1} \cdots p_k^{e_k}
\]
with $p_i$ pairwise non-associated primes. Then by the Chinese Remainder Theorem, we have
\[
	R/I \cong R/p_1^{e_1}R \oplus \cdots \oplus R/p_k^{e_k}R,
\]
as we had to show.
\end{proof}

We now consider two special cases of this corollary. The first one is a structure theorem for finitely generated and finite abelian groups.

\begin{theorem}[Classification of finitely generated abelian groups]
Let $A$ be a finitely generated abelian group. Then there exists an integer $n$, prime numbers 
$p_1$, \ldots, $p_k$, and positive integers $e_1$, \ldots, $e_k$ such that 
\[
	A \cong \bZ^{n} \times \bZ/{p_1^{e_1}}\bZ \times \cdots \times \bZ/{p_k^{e_k}}\bZ.
\]
If $A$ is a finite abelian group then 
\[
	A \cong \bZ/{p_1^{e_1}}\bZ \times \cdots \times \bZ/{p_k^{e_k}}\bZ.
\]
\end{theorem}

\begin{proof}
Apply Corollary \ref{cor:structure-fg-mod-over-PID} to the case $R=\bZ$.
\end{proof}

The second special case is a structure theorem for endomorphisms of finite-dimensional vector spaces over $\bC$ (or over an algebraically closed field). 

\begin{theorem}Let $K$ be an algebraically closed field. Let $V$ be a finite-dimensional vector space over $K$. Let $\alpha \colon V\to V$ be an endomorphism. Then there exist $\lambda_1$, \ldots, $\lambda_k$ in $K$, positive integers $e_1,\ldots, e_k$, and a decomposition
\[
	V = V_1 \oplus \cdots \oplus V_k
\]
such that $\alpha(V_i)\subset V_i$, and such that each $V_i$ has a basis on which $\alpha$ is expressed
as the standard Jordan matrix
\[
\left(\begin{matrix} \lambda_i & 0 &   \cdots & 0 & 0 \\ 
	1 & \lambda_i &  \cdots & 0 & 0 \\
	\vdots & \vdots &  & \vdots & \vdots \\ 
	0 & 0 &  \cdots & \lambda_i & 0 \\
	0 & 0 &   \cdots & 1& \lambda_i \end{matrix} \right)
\] 
of size $e_i$.
\end{theorem}

\begin{proof}
As in Example \ref{exa:vect-with-endo}, we turn the vector space $V$ into a $K[X]$-module, with $X$ acting via the endomorphism $\alpha$. Since $V$ is finite-dimensional, it is finitely generated as a $K$-module, hence a fortiori also as a $K[X]$-module.

Since $K$ is algebraically closed, the irreducible elements of $K[X]$ are (up to units) the linear polynomials
$X-\lambda$ with $\lambda \in K$. By Corollary \ref{cor:structure-fg-mod-over-PID}, we have
\[
	V\cong K[X]^n \oplus \frac{K[X]}{(X-\lambda_1)^{e_1}K[X]} \oplus \cdots \oplus
	\frac{K[X]}{(X-\lambda_k)^{e_k} K[X]}.
\]
Note that $K[X]$ is infinite-dimensional as a $K$-vector space, so we necessarily must have $n=0$.

Without loss of generality, we may assume 
\[
	V = K[X]/(X-\lambda)^{e}K[X]
\]
for some $\lambda \in K$ and $e > 0$. Consider the elements
\[
	v_j := (\bar{X}-\lambda)^j \in V \quad j \in \{0,\ldots, e-1\}.
\]
Note that the $v_j$ form a $K$-basis of $V$. The action of $\alpha$ on this basis is given by
\[
	\alpha(v_j) = X \cdot (\bar{X}-\lambda)^j
	= (\bar{X}-\lambda)^{j+1} + \lambda (\bar{X}-\lambda)^j
\]
hence
\[
	\alpha(v_j) = \begin{cases}
	 v_{j+1} + \lambda v_j & j<e-1 \\
	 \lambda v_j & j = e-1.
	 \end{cases}
\]
and we see that the matrix of $\alpha$ with respect to this basis is the standard Jordan block of eigenvalue $\lambda$ and size $e$.
\end{proof}


\newpage
\section*{Exercises}

\begin{exercise}Let $R$ be a commutative ring and $\fm \subset R$ a maximal ideal. 
Let $M$ be an $R$-module. Show that $M/\fm M$ is a vector space over $R/\fm$. Show that if $M$ is generated by $x_1,\ldots, x_n$ then $M/\fm M$ has dimension at most $n$ over $R/\fm$.
\end{exercise}

\begin{exercise}\label{ex:torsion-not-submodule}
Let $R$ be a commutative ring and $M$ an $R$-module. An element $x\in M$ is called a \emph{torsion} element if there exists a non-zero $r\in R$ with $rx=0$. 

Assume that $R$ is an integral domain. Show that the torsion elements of an $R$-module $M$ form a submodule of $M$. 

Give an example to show that the condition that $R$ is an integral domain cannot be dropped.
\end{exercise}

\begin{exercise}\label{exc:non-free-submodule}
Let $R$ be an integral domain. Show that the following are equivalent:
\begin{enumerate}
\item every submodule of a free $R$-module of finite rank is free of finite rank,
\item $R$ is a principal ideal domain.
\end{enumerate}
\end{exercise}


\begin{exercise}
Let $R$ be a PID and let $M$ be an $R$-module which can be generated by $n$ elements. Show that every submodule $N\subset M$ can be generated by $n$ elements. Show that the condition that $R$ is a PID cannot be dropped.
\end{exercise}

\begin{exercise}
Let $R$ be a PID and let $M$ be a finitely generated $R$-module such that for all $r\in R$ and $x\in M$ we have that $rx=0$ implies $r=0$ or $x=0$. Show that $M$ is free of finite rank.
\end{exercise}

\begin{exercise}
Let $R$ be a ring and let 
\[
	0 \longto M {\longto} E {\longto} N \longto 0
\]
be a short exact sequence of $R$-modules. Assume that $M$ can be generated by $m$ elements, and that $N$ can be generated by $n$ elements. Show that $E$ can be generated by $m+n$ elements.
\end{exercise}

\begin{exercise}
Let $R$ be a PID, and let $p_1$ and $p_2$ be irreducible elements with $(p_1) \neq (p_2)$. Let $e_1,e_2$ be non-negative integers. Show that the only $R$-module homomorphism
\[
	R/p_1^{e_1}R \to R/p_2^{e_2}R
\]
is the zero homomorphism. 
\end{exercise}

\begin{exercise}
Let $R$ be a PID and let $p\in R$ be irreducible. Let $e_1,e_2$ be non-negative integers. Show that there is an isomorphism of
 $R$-modules
\[
	\Hom_R(R/p^{e_1}R,R/p^{e_2}R) \cong R/p^e R
\]
with $e=\min(e_1,e_2)$.
\end{exercise}

\begin{exercise}
Describe all $\bZ[i]$-modules with at most $10$ elements, up to isomorphism.
\end{exercise}

\begin{exercise}
Let $R$ be a PID and let $p \in R$ be an irreducible element. Let $E$ be an $R$-module contained in a short exact sequence
\[
	0 \longto R/pR \overset{f}{\longto} E \overset{g}{\longto} R/pR \longto 0.
\]
Show that either $E \cong R/pR \oplus R/pR$ or $E\cong R/p^2R$, that both options occur, and that 
$R/pR \oplus R/pR \not\cong R/p^2R$.
\end{exercise}

\begin{exercise}[$\star$]
Let
\[
	0 \longto A_1 \longto A_2 \longto A_3 \longto \cdots \longto A_n \longto 0
\]
be an exact sequence of finite abelian groups. Show that the equality
\[
	\prod_i |A_i|^{(-1)^{i}} = 1
\]
holds.
\end{exercise}

\begin{exercise}
Let $K$ be a field and let $f_1,\ldots, f_n \in K[X]$ be monic polynomials. Consider the $K[X]$-module
\[
	V := K[X]/(f_1) \oplus \cdots \oplus K[X]/(f_n).
\]
Show that the characteristic polynomial of the endomorphism $v\mapsto X\cdot v$ of the $K$-vector space $V$ equals $\prod_i f_i$.
\end{exercise}

\begin{exercise}
Let $K$ be a field and let $V$ be a finite-dimensional vector space over $K$. Let $\alpha$ be an endomorphism of $V$. 
Assume that the characteristic polynomial of $\alpha$ is irreducible. Show that there is no proper non-zero subspace $W\subset V$ with $\alpha(W) \subset W$.
\end{exercise}

\begin{exercise}
Let $K$ be a field and let $V$ be a finite-dimensional vector space over $K$. Let $\alpha$ be an endomorphism of $V$. Assume that the characteristic polynomial of $\alpha$ is separable. Show that there are subspaces $W_1,\ldots, W_n$ of $V$, such that the following hold:
\begin{enumerate}
\item  $V = W_1 \oplus W_2 \oplus \cdots \oplus W_n$,
\item $\alpha(W_i) \subset W_i$ for every $i$,
\item the characteristic polynomial of $\alpha_{|W_i}$ is irreducible for every~$i$.
\end{enumerate}
Give an example to show that the condition that the characteristic polynomial is separable cannot be dropped.
\end{exercise}


\begin{exercise}[$\star\star$]
Let $R$ be a PID. An $R$-module $M$ is called \emph{torsion} if for every $x\in M$ there is a non-zero $r\in R$ with $rx=0$. 
For a finitely generated torsion $R$-module
\[
	M\cong R/I_1 \oplus \cdots \oplus R/I_k
\]
denote by $F(M)$ the ideal $I_1I_2\cdots I_k$. Show that $F(M)$ is independent of the chosen decomposition. Show that if 
\[
	0 \longto M_1 \longto M_2 \longto M_3 \longto 0
\]
is a short exact sequence of $R$-modules, and if $M_1$ and $M_3$ are finitely generated torsion $R$-modules, then $M_2$ is a finitely generated torsion $R$-module and $F(M_2)=F(M_1)F(M_3)$.
\end{exercise}


%%%%%%%%%%%%%%%%%%%%%%%%%%%%%%%%%%%%%%%%%%
% CHAPTER: CATEGORIES %
%%%%%%%%%%%%%%%%%%%%%%%%%%%%%%%%%%%%%%%%%%

\chapter{Categories}
\label{chapter:categories}


\section{Definition}

\begin{definition}A \emph{category} $\cC$ consists of the data of
\begin{enumerate}
\item a class of \emph{objects} $\ob \cC$, 
\item for every $X,Y\in \ob \cC$ a class $\Hom_\cC(X,Y)$,
\item for every $X\in \ob\cC$ an element $\id_X \in \Hom_\cC(X,X)$,
\item for every $X,Y,Z \in \ob \cC$ a map
\[
	  \Hom_\cC(Y,Z) \times \Hom_\cC(X,Y) \to \Hom_\cC(X,Z),\,
	(g,f) \mapsto gf,
\]
\end{enumerate}
called \emph{composition}, subject to the conditions
\begin{enumerate}
\item[(C1)] for every $X,Y,Z,T \in \ob \cC$ and for every $f \in \Hom_\cC(X,Y)$, $g\in \Hom_\cC(Y,Z)$ and $h \in \Hom_\cC(Z,T)$ the identity $h(gf) = (hg)f$ holds,
\item[(C2)] for every $X,Y\in \ob \cC$ and for every $f\in \Hom_\cC(X,Y)$ the identities $f \id_X = f$ and $\id_Y f = f$ hold.
\end{enumerate}
\end{definition}

We often write $f\colon X\to Y$ instead of $f\in \Hom_\cC(X,Y)$, and think of $f$ as being a `map from $X$ to $Y$'. However, the objects $X$ and $Y$ need not be sets, and expressions such as `$x\in X$' or `$f(x)\in Y$' can be completely meaningless. 

\begin{remark}
The use of the word `class' is to avoid set-theoretical problems. We want to consider examples such as the category of all sets, but must be careful to avoid paradoxes. The class of all sets does not form a set itself, otherwise one could consider the subset of all the sets that are not contained in itself, leading to Russell's paradox. 

This subtlety in the definition of a category is mostly harmless, and in almost all applications of categories one can safely pretend the class of objects form a set. In fact, one can often restrict the objects to a suitable chosen sub-\emph{set} of the given class without losing much.
\end{remark}

\section{Big examples}\label{sec:big-examples}

The notion of a category is modeled on the properties of the collection of all objects of a certain kind (sets, rings, spaces) together with the collection of all structure-preserving maps (functions, ring homomorphisms, continuous maps) between them. The most important examples are of this kind.

\begin{example}[The category of sets] The category $\Set$ with $\ob \Set$ the class of all sets,  $\Hom_{\Set}(X,Y)$ the set of all maps from $X$ to $Y$, $\id_X$ the identity map and the usual composition $gf:= g \circ f$ forms a category.
\end{example}

\begin{example}[The category of topological spaces] The category $\Top$ with $\ob \Top$ the class of all topological spaces, $\Hom_\Top(X,Y)$ the set of continuous maps from $X$ to $Y$ and the usual identity and composition form a category. (Note that the composition of two continuous maps is continuous!).
\end{example}

\begin{example}[The categories of left and right $R$-modules] If $R$ is a ring, we denote by ${}_R\Mod$ the category whose objects are the left  $R$-modules, and whose morphisms are the $R$-module homomorphisms. So for $M,N \in \ob {}_R\Mod$ we have
\[
	\Hom_{{}_R\Mod}( M, N ) := \Hom_R( M, N ).
\]
Similarly, we denote by $\Mod_R$ the category of right $R$-modules.
\end{example}


In the same style as the above examples, we have the category $\Ring$ of rings and ring homomorphisms, the category $\CRing$ of commutative rings and ring homomorphisms, the category $\Grp$ of groups and group homomorphisms, the category $\Ab$ of abelian groups and group homomorphisms, etcetera. Note that in all these examples the objects of the categories are sets equipped with some extra structure, and the morphisms are  functions that are compatible with the structure.  This is the case for many, but certainly not all, commonly used categories. 


\section{Small examples}\label{sec:small-examples}

A category is also a mathematical object in its own right, and one can write down explicit examples by specifying the objects and maps, in the same way one can specify say a ring by giving its elements, the addition, and multiplication. 

\begin{example}[One arrow]\label{exa:one-arrow}
Consider the category $\cC$ consisting of precisely two objects, $X$ and $Y$, and with precisely three maps: $\id_X$, $\id_Y$, and a map $f\colon X\to Y$. We can render this category in a picture:
\[
\begin{tikzpicture}
 \node (X) [label=below:{$X$}] at (-.8,0) {};
 \fill (X) circle[radius=2pt];
 \node (Y) [label=below:{$Y$}] at (.8,0) {};
 \fill (Y) circle[radius=2pt];
  \path (X) edge [loop left, out=-150,in=150,min distance=7mm, ->] node {$\id_X$} (X);
  \path (X) edge [->] node[above] {$f$} (Y);
  \path (Y) edge  [loop right, out=30,in=-30,min distance=7mm, ->] node {$\id_Y$} (Y);
\end{tikzpicture}
\]
\end{example}

\begin{example}[A group as a category] \label{exa:BG}
Let $G$ be a group. Consider the category $\rB G$ with one object
$\star$ (so $\ob\rB G = \{ \star\}$), and with $\Hom_{\rB G}(\star,\star) := G$, $\id_\star := 1 \in G$, and where composition is defined by multiplication in $G$. Note that axiom (C1) follows from the associativity of the group operation, and axiom (C2) from the axiom for the neutral element $1\in G$.
\end{example}

\begin{example}[Discrete category]\label{exa:discrete-cat}
If $S$ is a set, then $S$ defines a category $\cC$ with $\ob \cC := S$ and with
\[
	\Hom_\cC(x,y) := \begin{cases}  \{ \id \} & \text{if $x=y$} \\ \emptyset &\text{if $x\neq y$} \end{cases}
\]
for all $x,y\in S$.
\end{example}

\begin{example}[Pre-ordered set]\label{exa:pre-ordered}
A \emph{pre-ordered set} is a set $S$ equipped with a relation $\leq$ that is reflexive ($x\leq x$)  and transitive ($x\leq y$ and $y\leq z$ implies $x\leq z$). A pre-ordered set defines a category $\cC$ with $\ob \cC := S$ and with
\[
	\Hom_\cC(x,y) := \begin{cases}  \{  \star \} & \text{if $x\leq y$} \\ \emptyset &\text{if $x\not\leq y$} \end{cases}
\]
where $\{\star\}$ denotes any singleton. 
\end{example}

\begin{example}[The category of matrices] \label{exa:category-of-matrices}
Let $R$ be a ring. Then we can form a category $\cC$  with $\ob \cC :=\{ 0, 1, 2, \ldots \}$,  with
\[
	\Hom_\cC(n,m) := \Mat_{m,n}(R) := \{ \text{$m\times n$ matrices over $R$} \},
\]
and with composition being matrix multiplication
\[
	 \Mat_{m,\ell} \times \Mat_{\ell,k}  \to \Mat_{m,k}, (B,A) \mapsto BA.
\]
The identity element $\id_n\colon n \to n$ is the $n\times n$ identity matrix. Axiom (C1) corresponds to the associativity of matrix multiplication.
\end{example}

\begin{definition}\label{def:small-category}
A category is called \emph{locally small} if for every $X$ and $Y$ in $\ob \cC$ the class $\Hom_{\cC}(X,Y)$ is a set. It is called \emph{small} if moreover the class of objects $\ob \cC$ is itself a set.
\end{definition}

\begin{examples}
All the examples in sections~\ref{sec:big-examples} and \ref{sec:small-examples} are locally small. The examples in section~\ref{sec:small-examples} are in fact small. However, the examples in section~\ref{sec:big-examples} are not small, essentially because the class of all sets is not a set.
\end{examples}


Finally, we discuss two ways of constructing new categories out of old ones. The first is the opposite category, which is obtained by formally reversing all the arrows in a given category:
 
\begin{definition}\label{def:opposite-category}
The \emph{opposite} or \emph{dual} of a category $\cC$ is the category $\cC^\opp$ with $\ob \cC^\opp := \ob\cC$ and with
\[
	\Hom_{\cC^\opp}(X,Y) := \Hom_\cC(Y,X).
\]
Composition is done `the other way around': 
\[
	\Hom_{\cC^\opp}(Y,Z) \times \Hom_{\cC^\opp}(X,Y)   \to \Hom_{\cC^\opp}(X,Z),\,
	(g,f) \mapsto fg
\]
where $f\colon Y \to X$ and $g\colon Z\to Y$ and $fg\colon Z \to X$ are maps in $\cC$.
\end{definition}

\begin{definition}\label{def:product-category}
If $\cC$ and $\cD$ are categories, then the \emph{product category}
$\cC\times \cD$ is the category whose objects are pairs $(X,Y)$ with $X\in \ob \cC$ and $Y\in \ob \cD$, and whose morphisms are pairs of morphisms:
\[
	\Hom_{\cC\times \cD}((X,Y),(X',Y')) := \Hom_\cC(X,X') \times \Hom_\cD(Y,Y').
\]
\end{definition}

In general it makes no sense to ask if a morphism $f\colon X\to Y$ in a category $\cC$ is bijective, injective, or surjective. The objects $X$ and $Y$ are just elements of some class $\ob \cC$, and the morphism $f$ is just an element of some class $\Hom_\cC(X,Y)$.  It makes no sense to talk about elements of $X$ or $Y$. However, one can define what it means for $f$ to be an \emph{isomorphism}.


\begin{definition}Let $f\colon X\to Y$ be a morphism in a category $\cC$. We say that $f$ is an \emph{isomorphism} if there exists a morphism $g\colon Y \to X$ such that $fg=\id_Y$ and $gf=\id_X$.
\end{definition}

\begin{example}An isomorphism in $\Set$ is a bijection. An isomorphism in $\Grp$ is a group isomorphism. An isomorphism in $\Top$ is a homeomorphism (note that this is stronger than being a bijective continuous map!). An isomorphism in the category of matrices (Example \ref{exa:category-of-matrices}) is an invertible square matrix.
\end{example}

\begin{definition}We say that objects $X$ and $Y$ in a category are \emph{isomorphic} if there exists an isomorphism $f\colon X\to Y$.
\end{definition}

%\begin{definition}A morphism $f\colon X\to Y$ in a category $\cC$ is called a \emph{monomorphism} if
%for every object $T$ and for every pair of maps $g_1,g_2\colon T\to X$ with $fg_1=fg_2$ we have $g_1=g_2$.
%\end{definition}
%
%\begin{example}We claim that the monomorphisms in $\Set$ are precisely the injective functions. Indeed, if $f$ is injective, then clearly $fg_1 = fg_2$ implies $g_1=g_2$. Conversely, let $f\colon X\to Y$ be a monomorphism. Let $x_1,x_2\in X$ with $f(x_1)=f(x_2)$. Consider the one-element set $T=\{\star\}$ and the maps $g_i\colon T\to X,\, \star \mapsto x_i$. Then $fg_1=fg_2$, hence $g_1=g_2$, and therefore $x_1=x_2$.
%\end{example}
%
%\begin{example}By a similar argument one shows that the monomorphisms in $\Grp$ are precisely the injective group homomorphisms, see Exercise \ref{exc:monomorphisms-in-Grp}.
%\end{example}
%
%\begin{definition}A morphism $f\colon X\to Y$ in a category $\cC$ is called an \emph{epimorphism} if
%for every $T\in \ob \cC$ and for every pair of maps $g_1,g_2$ from $Y$ to $T$ with $g_1f=g_2f$ we have $g_1=g_2$.
%\end{definition}
%
%\begin{example}We claim that the epimorphisms in $\Set$ are precisely the surjective functions. Indeed, if $f$ is surjective and $g_1f=g_2f$, then certainly $g_1=g_2$. Conversely, let $f\colon X\to Y$ be an epimorphism. Assume that $f$ is not surjective, and pick an element $z\in Y$ not in the image of $f$. Consider the set $T:=\{0,1\}$ and maps 
%\[
%	g_1\colon Y \to T,\, y \mapsto 0
%\]
%and
%\[
%	g_2\colon Y \to T,\, y \mapsto \begin{cases} 0 & y \neq z \\ 1 & y = z \end{cases}
%\]
%Then $g_1f=g_2f$, but $g_1\neq g_2$, a contradiction.
%\end{example}


\section{Final and cofinal objects}

Some categories have special objects, called final and cofinal (also known as initial) objects. 

\begin{definition}
Let $\cC$ be a category. An object $X\in \ob \cC$ is called \emph{final} in $\cC$ if for every $Y\in \ob\cC$ there is a 
unique morphism $Y \to X$.
\end{definition}

\begin{definition}
Let $\cC$ be a category. An object $X\in \ob \cC$ is called \emph{initial} or \emph{cofinal} in $\cC$ if for every $Y\in \ob\cC$ there is a 
unique morphism $X\to Y$.
\end{definition}

\begin{examples}
In $\Top$ the empty space $\emptyset$ is a cofinal object, and a one-point space $\{\star\}$ is final. In ${}_R\Mod$ the zero module $0$ is both cofinal and final. In $\Ring$ the ring $\bZ$ is cofinal and the zero ring $0$ is final. In $\Grp$ the trivial group $\{1\}$ is both cofinal and final.
\end{examples}

\begin{proposition}\label{prop:final-object-uniquely-unique}
If it exists, a final object in a category $\cC$ is unique up to unique isomorphism.
\end{proposition}

\begin{proof}
Assume $X_1$ and $X_2$ are final. We need to show that there exists a unique isomorphism $f\colon X_1\to X_2$.

Since $X_2$ is final, there exists a map $f\colon X_1\to X_2$. Since $X_1$ is final, there exists a map $g\colon X_2\to X_1$. Consider the composite $fg\colon X_2 \to X_2$. Since $X_2$ is final, there is a unique map $X_2\to X_2$, so $fg=\id_{X_2}$. Similarly $gf=\id_{X_1}$. So we see that $f$ is an isomorphism $X_1\to X_2$.

Now assume that there are \emph{two} isomorphisms $f_1,f_2\colon X_1 \to X_2$. Then again, because $X_2$ is final, we must have $f_1=f_2$, which shows that $f$ is unique. 
\end{proof}

Like so many statements about categories, this proposition has a co-proposition:

\begin{proposition}
If it exists, a cofinal object in a category $\cC$ is unique up to unique isomorphism.
\end{proposition}

\begin{proof}Reverse the arrows in the proof of Proposition \ref{prop:final-object-uniquely-unique}. Alternatively, apply 
Proposition~\ref{prop:final-object-uniquely-unique} to the opposite category $\cC^\opp$.
\end{proof}

\newpage
\section*{Exercises}

\begin{exercise}[Automorphism group of an object]
Let $\cC$ be a (locally small) category and $X$ an object in $\cC$. Show that
\[
	\Aut_\cC(X) := \{ f\colon X\to X \mid \text{$f$ is an isomorphism} \}
\]
forms a group under composition.
\end{exercise}

\begin{exercise}
Let $\cC$ be a (locally small) category and $X$ and $Y$ objects in $\cC$. Assume that $X$ and $Y$ are isomorphic. Show that $\Aut_\cC(X)$ and $\Aut_\cC(Y)$ are isomorphic groups.
\end{exercise}

\begin{exercise}
Let $\cC$ be a category. Show that the relation `$X$ and $Y$ are isomorphic' forms an equivalence relation on $\ob \cC$.
\end{exercise}

\begin{exercise}
Let $f\colon X\to Y$ be a morphism in a category $\cC$. Show that if $f$ has both a left and a right inverse, then these must agree. In particular, $f$ has at most one two-sided inverse.
\end{exercise}

\begin{exercise}\label{exc:iso-category}
Let $\cC$ be a category. Define $\ob \cC^\times := \ob \cC$ and
\[
	\Hom_{\cC^\times}(X,Y) := \{ f \in \Hom_\cC(X,Y) \mid \text{$f$ is an isomorphism} \}.
\]
Show that $\cC^\times$ (with composition and identity maps inherited from $\cC$) is a category.
\end{exercise}

\begin{exercise}Verify that Examples~\ref{exa:one-arrow}  and~\ref{exa:discrete-cat} are special cases of Example~\ref{exa:pre-ordered}.
\end{exercise}

%\begin{exercise}\label{exc:monomorphisms-in-Grp}
%Show that the monomorphisms in $\Grp$ are precisely the injective group homomorphisms. (Hint: given $x\in G$ there is a group homomorphism $\bZ\to G$ which maps $1$ to $x$).
%\end{exercise}
%
%\begin{exercise}Let $R$ be a ring. Show that the monomorphisms in ${}_R\Mod$ are precisely the injective $R$-module homomorphisms.
%\end{exercise}
%
%\begin{exercise}Show that the epimorphisms in ${}_R\Mod$ are precisely the surjective  homomorphisms. (Hint: given $f\colon M\to N$ non-surjective, consider the module $\coker f$.)
%\end{exercise}

%
%\begin{exercise}
%Show that the map $\bZ\to\bQ$ in the category of rings is an epimorphism. Conclude that there exist morphisms that are both mono and epi, but not isomorphisms.
%\end{exercise}
%
%\begin{exercise}Let $\mathrm{\mathbf{Haus}}$ be the category of Hausdorff topological spaces and continuous maps. Let $f\colon X \to Y$ be a continuous map of Hausdorff spaces with dense image, show that $f$ is an epimorphism. (Hint: Let $g,h\colon Y \to T$ be continuous maps of topological spaces, with $T$ Hausdorff, show that $\{ x \in Y \mid g(x) = h(x) \}$ is closed). 
%\end{exercise}
%
%\begin{exercise}[$\star$] Show that the converse holds: if $f\colon X\to Y$ in $\mathrm{\mathbf{Haus}}$ is an epimorphism, then the image of $f$ is dense in $Y$.
%\end{exercise}
%
%
%\begin{exercise}[$\star\star$]Show that the epimorphisms in $\Grp$ are precisely the surjective group homomorphisms.
%\end{exercise}
%
%\begin{exercise}
%What are the isomorphisms, monomorphisms and epimorphisms in the matrix category over a field $K$? (See Example \ref{exa:category-of-matrices}).
%\end{exercise}

\begin{exercise}Show that the category of fields has neither final nor cofinal object. Show that the category of fields of a given fixed characteristic does have a cofinal object.
\end{exercise}

%\begin{exercise}[Split monomorphisms]\label{exc:split-mono}
%Let $f\colon X\to Y$ be a morphism in $\cC$. Assume that there exists a morphism $g\colon Y\to X$ with $gf=\id_X$. Show that $f$ is a monomorphism. Such $f$ is called a \emph{split monomorphism}. Give an example of a monomorphism that is not a split monomorphism.
%\end{exercise}
%
%\begin{exercise}[Split epimorphisms]\label{exc:split-epi}
%Let $f\colon X\to Y$ be a morphism in $\cC$. Assume that there exists a morphism $g\colon Y\to X$ with $fg=\id_Y$. Show that $f$ is an epimorphism. Such $f$ is called a \emph{split epimorphism}. Give an example of an epimorphism that is not a split epimorphism.
%\end{exercise}


\begin{exercise}[The category of $G$-sets] Let $G$ be a group. A $G$-set is a set $S$ together with a left action of $G$ on $S$, that is, a map $G\times S \to S,\, (g,s)\mapsto gs$ satisfying $1s=s$ and $g(hs)=(gh)s$ for all $g,h \in G$ and $s\in S$. A morphism of $G$-sets is a map $f\colon S\to T$ satisfying $f(gs)=gf(s)$ for every $s\in S$ and $g\in G$. The category of $G$-sets is denoted $_{G}\Set$. 

What are the final and cofinal objects of $_{G}\Set$? 
\end{exercise}

\begin{exercise}
Let $\cC$ be a category and $X$ and $Y$ objectsin $\cC$. Assume that $X$ is final, and that $X$ and $Y$ are isomorphic. Show that $Y$ is also final.
\end{exercise}


\begin{exercise}[Homotopy category]
Let $\hTop$ be the \emph{homotopy category} of topological spaces. Its objects are 
topological spaces, and its morphisms are \emph{homotopy classes} of continuous maps:
\[
	\Hom_{\hTop}(X,Y) := \{ f\colon X\to Y \mid \text{$f$ continuous}\}/\sim
\]
(Note that composition is compatible with homotopy, so that composition in $\hTop$ is well-defined). Show that two topological spaces $X$ and $Y$ are homotopy-equivalent if and only if $X$ and $Y$ are isomorphic in the category $\hTop$. 
\end{exercise}

\begin{exercise}\label{exc:yoneda-isomorphism-test}
Let $f\colon X\to Y$ be a morphism in $\cC$. Show that $f$ is an isomorphism if and only if for all objects $T$ in $\cC$ the map
\[
	\Hom_\cC(T,X) \to \Hom_\cC(T,Y),\, h \mapsto fh
\]
is a bijection. (Hint, for the hard direction: use $T=Y$ to find a $g\colon Y\to X$ with $fg=\id_Y$, and use $T=X$ to deduce that $gf=\id_X$.)
\end{exercise}

\begin{exercise}Formulate and prove the co-Exercise of Exercise~\ref{exc:yoneda-isomorphism-test}.
\end{exercise}

%%%%%%%%%%%%%%%%%%%%%%
% FUNCTORS AND EQUIVALENCES
%%%%%%%%%%%%%%%%%%%%%%%


\chapter{Functors}

If in order to study groups, modules or topological spaces, one should study the homomorphisms or continuous maps between them, then to study categories one should study `morphisms of categories'. Such morphisms of categories are called \emph{functors}. 

\section{Definition of a functor}

\begin{definition} Let $\cC$ and $\cD$ be categories. A \emph{functor} $F$ from $\cC$ to $\cD$, consists of the data of
\begin{enumerate}
\item for every object $X$ in $\cC$ an object $F(X)$ in $\cD$,
\item for every morphism $f\colon X\to Y$ in $\cC$ a morphism $F(f)\colon F(X) \to F(Y)$ in $\cD$
\end{enumerate}
subject to the conditions
\begin{enumerate}
\item[(F1)] for every $X$ in $\cC$ we have $F(\id_X) = \id_{F(X)}$
\item[(F2)] for every $f\colon X\to Y$ and $g\colon Y\to Z$ in $\cC$ we have $F(gf)=F(g)F(f)$.
\end{enumerate}
\end{definition}

We will  write $F\colon \cC\to \cD$ to denote that $F$ is a functor from $\cC$ to $\cD$.

Note that if $F\colon \cC\to \cD$ and $G\colon \cD \to \cE$ are functors, then  the composite $GF\colon \cC \to \cE$ defined by $(GF)(X) := G(F(X))$ and $(GF)(f) := G(F(f))$ is also a functor. 

To avoid overloading notation, we will often write $FX$ and $Ff$ instead of $F(X)$ and $F(f)$.

\section{Many examples}

\begin{example}[Identity]
For every category $\cC$  there is an identity functor
$\id_\cC \colon \cC \to \cC$ with $\id_\cC(X) = X$ and $\id_\cC(f)= f$. 
\end{example}

\begin{example}[Forgetful functors]\label{exa:forgetful}
Let $R$ be a ring. Then we have a functor
\[
	F\colon {}_R\Mod \to \Ab
\]
defined by $F(M):=M$ (as an abelian group) and $F(f):=f$. 
An $R$-module $M$ is an abelian group equipped with extra structure, and this functor `forgets' the extra structure. Other examples of forgetful functors are the obvious functors $\Top \to \Set$ (forgetting the topology), $\Ab \to \Set$ (forgetting the addition), and $\Ring \to \Ab$ (forgetting the multiplication).
\end{example}
 
\begin{example}[Hom functor]\label{exa:covariant-hom-functor}
Let $\cC$ be a category in which the $\Hom$ classes are sets, and let $X$ be an object of $\cC$. Then we define a functor $F\colon \cC\to \Set$ as follows. For an object $Y$ in $\cC$ we define
\[
	F(Y) := \Hom_\cC(X,Y),
\]
and for a morphism $f\colon Y_1 \to Y_2$ in $\cC$ we define
\[
	F(f) \colon \Hom_\cC(X,Y_1) \to \Hom_\cC(X,Y_2),\, g \mapsto fg.
\]
One easily checks that $F$ is a functor. We  will denote it by $\Hom_\cC(X,-)$.
\end{example}


\begin{example}[Abelianization of a group]\label{exa:abelianization}
If $G$ is a group then we denote by $[G,G]$ its commutator subgroup. This is the subgroup generated by the elements $sts^{-1}t^{-1}$ with $s,t\in G$. It is a normal subgroup, and the quotient group
\[
	G^{\ab} := G/[G,G]
\]
is abelian. It is called the \emph{abelianization} of $G$. If $f\colon G\to H$ is a group homomorphism, then $f([G,G]) \subset [H,H]$,  hence $f$ induces a group homomorphism 
\[
	f^\ab \colon G^\ab \to H^\ab.
\]
Together, these constructions define a functor $(-)^\ab\colon \Grp\to\Ab$.
\end{example}

\begin{example}[Free module]\label{exa:free-module-functor}
Let $R$ be a ring. For every set $I$ we have the free module 
\[
	R^{(I)} := \left\{ (x_i)_{i\in I} \in R^I \mid \text{$x_i=0$ for all but finitely many $i$} \right\},
\]
see Example \ref{ex:free-module}. If $f\colon I\to J$ is a map of sets, then we have an induced map
$R^{(I)} \to R^{(J)}$ 
determined by requiring that $f$ maps the standard basis vector $e_i$ of $R^{(I)}$ to the standard basis vector $e'_{f(i)}$ of $R^{(J)}$. This construction defines a functor $\Set \to {}_R\Mod$.
\end{example}



\begin{example}[The fundamental group of a pointed space] Let $\Top_\ast$ be the category of \emph{pointed topological spaces}. Objects in $\Top_\ast$ are pairs $(X,x)$ with $X$ a topological space and $x\in X$. A morphism from $(X,x)$ to $(Y,y)$ is a continuous map $f\colon X\to Y$ such that $f(x)=y$.

Then the fundamental group defines a functor
\[
	\pi_1\colon \Top_\ast \to \Grp.
\]
On the level of objects it is simply defined by mapping a pair $(X,x)$ to the fundamental group $\pi_1(X,x)$. On the level of morphisms
it is defined as follows. Let $(X,x)$ and $(Y,y)$ be pointed spaces, and let $f \colon X\to Y$ be a continuous map such that $f(x)=y$. Then
we defined $\pi_1(f)$ by
\[
	\pi_1(f) \colon \pi_1(X,x) \to \pi_1(Y,y),\, [\gamma] \mapsto [f\circ \gamma],
\]
where $[\gamma]$ denotes the class of a loop $\gamma\colon [0,1] \mapsto X$ based at $x$.
\end{example} 

Note that the definition of fundamental group requires a base point. Although for a path-connected space $X$ and points $x,y\in X$ the fundamental groups $\pi_1(X,x)$ and $\pi_1(X,y)$ are isomorphic,
the isomorphism is not unique, as it depends on the choice of a path. 
%See also Exercise \ref{exc:fundamental-group-requires-base-point}.


\begin{example}[Solutions to polynomial equations]\label{exa:pol-eq}
Let $f_1,\ldots, f_m \in \bZ[X_1,\ldots, X_n]$. Then we can consider
solutions to the system of polynomial equations $f_1=f_2=\cdots =f_m=0$ in arbitrary commutative rings. Varying the commutative ring, one obtains a functor
\[
	F\colon \CRing \to \Set
\]
defined on the level of objects by  
\[
	R \mapsto \{ x=(x_1,\ldots, x_n) \in R^n \mid f_1(x)=f_2(x)=\cdots =f_m(x) =0 \}.
\]
and on the level of morphisms by 
\[
	\big[ \varphi\colon R\to S\big] \mapsto \big[ (x_1,\ldots,x_n) \mapsto 
	(\varphi(x_1),\ldots,\varphi(x_n))\big].
\]
Indeed, if $\varphi\colon R\to S$ is a ring homomorphism, and if $x\in R^n$ is a solution to the system of equations, then also $(\varphi(x_1),\ldots, \varphi(x_n) ) \in S^n$ is a solution to the system of equations.
\end{example}

\begin{example}[Commutative diagrams as functors] \label{exa:diagrams-as-functors}
Let $\cC$ be the small category 
\[
\begin{tikzpicture}
 \node (X) [label=above:{$x$}] at (-.8,.8) {};
 \fill (X) circle[radius=2pt];
 \node (Y) [label=above:{$y$}] at (.8,.8) {};
 \fill (Y) circle[radius=2pt];
 \node (Z) [label=below:{$z$}] at (-.8,-.6) {};
 \fill (Z) circle[radius=2pt];
 
%  \path (X) edge [loop left, out=-150,in=150,min distance=7mm, ->] node {$\id_x$} (X);
%  \path (Y) edge  [loop right, out=30,in=-30,min distance=7mm, ->] node {$\id_y$} (Y);
%  \path (Z) edge  [loop right, out=150,in=210,min distance=7mm, ->] node[left] {$\id_z$} (Z);
   \path (X) edge [->] node[above] {$f$} (Y);
   \path (X) edge [->] node[left] {$g$} (Z);
   \path (Z) edge [->] node[below] {$h$} (Y);
\end{tikzpicture}
\]
with $f=hg$ (and where we omitted $\id_x$, $\id_y$ and $\id_z$ from the picture). Let $\cD$ be any category. Then a functor $F\colon \cC \to \cD$ consists of objects $X:=F(x)$, $Y:=F(y)$, $Z := F(z)$ together with morphisms $F(f)$, $F(g)$ and $F(h)$ between them, such that $F(f)=F(h)F(g)$. In other words, a functor $F\colon \cC \to \cD$ is the same as a triangular commutative diagram
 \[
\begin{tikzcd}
X \arrow{r} \arrow{d} & Y  \\
Z \arrow{ru} 
\end{tikzcd}
\]
in the category $\cD$. With a similar construction, a commutative diagram  in $\cD$ of any shape can be thought of as a functor from some small category $\cC$ to $\cD$.
\end{example}

\section{Contravariant functors}

\begin{definition}
Let $\cC$ and $\cD$ be categories. A \emph{contravariant functor} from $\cC$ to $\cD$ consists of the
data of
\begin{enumerate}
\item for every object $X$ in $\cC$ an object $F(X)$ in $\cD$,
\item for every morphism $f\colon X\to Y$ in $\cC$ a morphism $F(f)\colon F(Y) \to F(X)$ in $\cD$
\end{enumerate}
subject to the conditions
\begin{enumerate}
\item[(F1)] for every $X$ in $\cC$ we have $F(\id_X) = \id_{F(X)}$
\item[(F2')] for every $f\colon X\to Y$ and $g\colon Y\to Z$ in $\cC$ we have $F(gf)=F(f)F(g)$.
\end{enumerate}
\end{definition}

To stress the difference, one sometimes calls an ordinary functor a \emph{covariant functor}.


\begin{remark}
The only difference with the notion of a functor is that $F$ reverses the order of composition. In other words, a contravariant functor $F$ from $\cC$ to $\cD$ is the same as a functor $F\colon \cC^\opp \to \cD$. 

To avoid clashing notation, we will reserve the notation $F\colon \cC \to \cD$ for a functor from $\cC$ to $\cD$, and will write $F\colon \cC^\opp \to \cD$ for a contravariant functor from $\cC$ to $\cD$.
\end{remark}

\begin{example}[Contravariant Hom functor]\label{exa:contravariant-hom-functor}
Let $\cC$ be a category and $X$ an object of $\cC$. If $f\colon Y_1\to Y_2$ is a morphism in $\cC$, then we have an induced map of 
sets
\[
	\Hom_\cC(Y_2,X) \to \Hom_\cC(Y_1,X),\, g \mapsto gf,
\]
and varying $Y$ we obtain a contravariant functor from $\cC$ to $\Set$ given by
\[
	\Hom_\cC(-,X)\colon \cC^\opp \to \Set,\,Y \mapsto \Hom_\cC(Y,X).
\]
This is a contravariant variation on Example \ref{exa:covariant-hom-functor}.
\end{example}

\begin{example}
Another (related) example of a contravariant functor is the `dual vector space'
\[
	(-)^\vee\colon \Vec_K^\opp \to \Vec_K
\]
which maps a $K$-vector space $V$ to its dual $V^{\vee} := \Hom_K(V,K)$ and a linear map
$f\colon V\to W$ to the induced map
\[
	f^\vee\colon W^\vee \to V^\vee,\,
	\varphi \mapsto \varphi\circ f.
\]
\end{example}

\begin{example}[Ring of functions on a space]\label{exa:functions-on-a-space}
Yet another (related) example is the functor
\[
	C\colon \Top^\opp \to \CRing,\, X \mapsto C(X) = \{ \varphi\colon X\to \bR \mid \text{$f$ continuous} \},
\]
mapping a topological space $X$ to the ring of continuous $\bR$-valued functions on $X$ (with point-wise addition and multiplication). On the level of maps it is defined as follows: if $f\colon X\to Y$ is continuous, then we have an induced map
\[
	C(f) \colon C(Y) \to C(X),\, \varphi \mapsto \varphi\circ f,
\]
which is clearly a ring homomorphism. 
\end{example}

\section{Functors with multiple arguments}


It is sometimes useful to consider functors with multiple arguments, living in various categories. This can easily be formalized using the notion of product category (see Definition \ref{def:product-category}). For example, a functor
\[
	F\colon \cC_1 \times \cC_2 \to \cD
\]
assigns to any pair of objects $(X_1,X_2)$ with $X_i\in \ob \cC_i$ an object $F(X_1,X_2)$ in $\cD$, and to any pair of morphisms $(f_1,f_2)$ with $f_i\colon X_i\to Y_i$ in $\cC_i$ a morphism $F(f_1,f_2) \colon F(X_1,X_2) \to F(Y_1,Y_2)$ in $\cD$. 

This can be combined with the notion of a contravariant functor. 

\begin{example}\label{exa:hom-in-two-arguments}
A typical example is the functor
\[
	\Hom(-,-) \colon \cC^\opp \times \cC \to \Set,
\]
which can be defined for any locally small category $\cC$. If $f\colon X'\to X$ and $g\colon Y\to Y'$ are morphisms in $\cC$ (take note of the directions of the arrows), then we have an induced morphism
\[	
	\Hom(f,g)\colon \Hom(X,Y) \to \Hom_R(X',Y'),
\]
which is given by $\alpha \mapsto g\alpha f$. One says that $\Hom(-,-)$ is contravariant in the first, and covariant in the second argument.
\end{example}





\newpage
\section*{Exercises}



\begin{exercise}
Let $\cC$ and $\cD$ be categories, let $F\colon \cC \to \cD$ be a functor. Let $f$ be an isomorphism in $\cC$. Show that $F(f)$ is an isomorphism
in~$\cD$.  Give an example where $F(f)$ is an isomorphism but $f$ is not.
\end{exercise}

\begin{exercise}
Verify that if $f\colon R\to S$ is a ring homomorphism, then $f$ restricts to a group homomorphism $R^\times \to S^\times$. Use this to construct a functor $\Ring \to \Grp$, mapping a ring $R$ to its group of units $R^\times$.
\end{exercise}


\begin{exercise}Verify the claims in Example \ref{exa:abelianization}.
\end{exercise}

\begin{exercise}
Let $S$ and $T$ be pre-ordered sets, defining categories $\cC$ and $\cD$ respectively (see Example \ref{exa:pre-ordered}). Describe the functors from $\cC$ to~$\cD$.
\end{exercise}

\begin{exercise}\label{exc:functor-GLn}
For a non-negative integer $n$ we denote by $\GL_n(R)$ the group of invertible $n$ by $n$ matrices with entries in $R$. In other words, the group of units in the ring $\Mat_n(R)$. For example $\GL_1(R) = R^\times$.

Show that $\GL_n$ defines a functor from the category of commutative rings $\CRing$ to the category of groups $\Grp$. 
\end{exercise}


\begin{exercise}[Center is not a functor\ldots]
Show that there exist morphisms $f\colon S_2 \to S_3$ and $g\colon S_3 \to S_2$ in $\Grp$ with the property that $gf=\id_{S_2}$.
Deduce that there is no functor $F\colon \Grp \to \Ab$ such that for every group $G$ we have that $FG$ is isomorphic to the center of $G$.
\end{exercise}

\begin{exercise}[\ldots but it is when we restrict to isomorphisms]
Let $\Grp^\times$ be the category whose objects are groups, and whose morphisms are \emph{isomorphisms} of groups (see also Exercise \ref{exc:iso-category}). Show that there is a functor $Z\colon \Grp^\times \to \Ab^\times$ that maps
a group $G$ to its center.
\end{exercise}
%
%\begin{exercise}[Fundamental group requires a base point\ldots]\label{exc:fundamental-group-requires-base-point}
%Show that there is {no} functor
%\[
%	F \colon \{ \text{path connected spaces} \} \to \Grp
%\]
%such that $F(X)\cong \pi_1(X,x)$ for every path connected space $X$ and $x\in X$.
%\end{exercise}




\chapter{Morphisms of functors}

\section{Morphisms of functors}


\begin{definition}
Let $\cC$ and $\cD$ be categories, and let $F\colon \cC\to \cD$ and $G\colon \cC \to \cD$ be functors. A \emph{morphism} or \emph{natural transformation} $\eta$ from $F$ to $G$ consists of the data of
\begin{enumerate}
\item for every object $X$ in $\cC$ a morphism $\eta_X \colon FX \to GX$ in $\cD$
\end{enumerate}
subject to the condition 
\begin{enumerate}
\item[(N1)]  for every morphism $f\colon X\to Y$ in $\cC$ the square
\[
\begin{tikzcd}
FX \arrow{r}{Ff} \arrow{d}{\eta_X} & FY \arrow{d}{\eta_Y} \\
GX \arrow{r}{Gf} & GY
\end{tikzcd}
\]
in $\cD$ commutes.
\end{enumerate}
An \emph{isomorphism} from $F$ to $G$ is a morphism of functors $\eta$ such that $\eta_X$ is an isomorphism in $\cD$ for every $X$ in $\cC$.
\end{definition}

We will  write $\eta\colon F\to G$ to denote that $\eta$ is a morphism from the functor $F$ to the functor $G$.

\begin{example}[double dual]
Let $K$ be a field and $V$ a $K$-vector-space. Then we have a natural map
\[
	\eta_V\colon V \to V^{\vee\vee} = \Hom_K(\Hom_K(V,K),K),\,
	v \mapsto \left( \varphi \mapsto \varphi(v) \right).
\]
The word `natural' is usually used in an informal sense, meaning `not depending on the choice of a basis'. But it also has a precise mathematical meaning, namely that the collection of maps $(\eta_V)_V$ with $V$ running over all the vector spaces forms a morphism
\[
	\eta\colon \id_{\Vec_K} \to (-)^{\vee\vee}
\]
from the functor $\id\colon \Vec_K \to \Vec_K$ to the functor $(-)^{\vee\vee}\colon \Vec_K \to \Vec_K$.
\end{example}

\begin{example}
Consider the forgetful functor $F\colon \Grp \to \Set$. Let $n$ be an integer. Then we have a morphism of functors $\eta\colon F \to F$ defined by
\[
	\eta_G \colon G \to G,\,g \mapsto g^n.
\]
Note that $g\mapsto g^n$ is a morphism of sets, but in general not a morphism of groups.
\end{example}

\begin{remark}\label{rmk:2-cat}
Note that the world of categories has three layers:
\begin{enumerate}
\item[(0)] categories
\item[(1)] functors between categories
\item[(2)] morphisms (natural transformations) between functors
\end{enumerate}
A similar picture arises in topology, where one distinguishes
\begin{enumerate}
\item[(0)] topological spaces
\item[(1)] continuous maps between spaces
\item[(2)] homotopies between continuous maps
\end{enumerate}
There is more to this than just an analogy and modern category theory and algebraic topology are heavily intertwined.

\end{remark}


\section{Equivalences of categories}




\begin{definition}
An functor $F\colon \cC \to \cD$ is called \emph{an equivalence} or \emph{an equivalence of categories} if there exists a functor $G\colon \cD \to \cC$ and isomorphisms of functors
\[
	\epsilon\colon FG \isomto \id_\cD,\quad \eta\colon GF\isomto \id_\cC.
\]
A functor $G$ with this property is called a \emph{quasi-inverse} of $F$. If there exists an equivalence from $\cC$ to $\cD$ then $\cC$ and $\cD$ are called \emph{equivalent} categories.
\end{definition}

\begin{remark}
Note that this is formally very similar to the notion of a homotopy equivalence in topology. See also Remark \ref{rmk:2-cat}.
\end{remark}

Equivalent categories tend to be `indistinguishable' from the point of view of category theory. See Exercise \ref{exc:equivalence-final} for an example. It is however often difficult to decide if a functor $F$ is an equivalence from the definition, since it can be hard to construct a quasi-inverse functor. We end this chapter with a powerful criterion for testing if a functor $F$ is an equivalence.

\begin{definition}
Let $F\colon \cC \to \cD$ be a functor. We say that $F$ is
\begin{enumerate}
\item \emph{full} if for every $X,Y$ in $\cC$ the map
\[
	\Hom_\cC(X,Y) \to \Hom_\cD(FX,FY),\, f \mapsto Ff
\]
 is surjective;
\item \emph{faithful} if for every $X,Y$ in $\cC$ the map
\[
	\Hom_\cC(X,Y) \to \Hom_\cD(FX,FY),\, f \mapsto Ff
\]
is injective;
\item \emph{essentially surjective} if for every object $Z$ in $\cD$ there is an $X$ in $\cC$ such that $FX$ and $Z$ are isomorphic in $\cD$.
\end{enumerate}
A functor which is full and faithful is often called \emph{fully faithful}.
\end{definition}



\begin{theorem}\label{thm:equivalence-of-categories}
A functor $F\colon \cC \to \cD$ is an equivalence of categories if and only if it is full, faithful and essentially surjective.
\end{theorem}

We start with two lemmas. 

\begin{lemma}\label{lemma:functor-iso-to-id}
Let $H\colon \cC \to \cC$ be a functor, and assume that $H$ is isomorphic to the functor $\id_\cC$. Then for all objects $X$ and $Y$ of $\cC$ the map
\[
	\Hom_\cC(X,Y) \to \Hom_\cC(HX, HY),\, f \mapsto Hf
\]
is a bijection.
\end{lemma}

\begin{proof}
By assumption, there exists an isomorphism $\varphi\colon \id_\cC \to H$. By definition, this means that for all $f\colon X\to Y$ in $\cC$ the square
\[
\begin{tikzcd}
	X \arrow{r}{\varphi_X} \arrow{d}{f} & HX \arrow{d}{Hf} \\
	Y \arrow{r}{\varphi_Y} & HY.
\end{tikzcd}
\]
commutes. Note that $\varphi_X$ and $\varphi_Y$ are isomorphisms (since $\varphi$ is an isomorphism). In particular, we have $Hf = \varphi_Y \circ f \circ \varphi_X^{-1}$. One can now check directly that the map
\[
	\Hom_\cC(HX,HY) \to \Hom_\cC(X,Y),\, g \mapsto \varphi_Y^{-1} \circ g \circ \varphi_X
\]
is a two-sided inverse to the map $f\mapsto Hf$.
\end{proof}

\begin{lemma}\label{lemma:two-out-of-three}
Let 
\[
	S \overset\alpha\longto T \overset\beta\longto U \overset\gamma\longto V
\]
be functions. Assume that the compositions $\beta\alpha$ and $\gamma\beta$ are bijections. Then $\alpha$, $\beta$, and $\gamma$ are bijections.
\end{lemma}

\begin{proof}
Since $\beta\alpha$ is surjective, $\beta$ must be surjective. Since $\gamma\beta$ is injective $\beta$ must also be injective. So $\beta$ is a bijection. But then $\alpha$ and $\gamma$ must be bijections too.
\end{proof}



\begin{proof}[Proof of Theorem~\ref{thm:equivalence-of-categories}]
Assume $F$ is an equivalence of categories, with quasi-inverse $G$ and isomorphisms
\[
	\epsilon\colon FG \to \id_\cD,\quad \eta\colon GF\to \id_\cC.
\]
Then for every $Y$ in $\cD$ the object $X := GY$ satisfies $FX = FGY \cong Y$ (via $\epsilon_Y$), hence $F$ is essentially surjective. 

To see that $F$ is full and faithful, let $X$ and $Y$ be objects of $\cC$ and consider
the composition
\[
	\Hom_{\cC}(X,Y) \to \Hom_{\cD}(FX,FY) \to \Hom_{\cC}(GFX,GFY).
\]
By Lemma~\ref{lemma:functor-iso-to-id} this composition is a bijection (since $GF\cong \id_\cC$). Similarly, the composition
\[
	\Hom_{\cD}(FX,FY) \to \Hom_{\cC}(GFX,GFY) \to \Hom_{\cD}(FGFX,FGFY).
\]
is a bijection (since $FG\cong \id_{\cC}$). But now applying Lemma~\ref{lemma:two-out-of-three} we conclude that 
\[
	\Hom_{\cC}(X,Y) \to \Hom_{\cD}(FX,FY)
\]
is a bijection for all $X$ and $Y$, and hence that $F$ is full and faithful.



Conversely, assume that $F$ is full, faithful and essentially surjective. Essential surjectivity means that for every $X$ in $\cD$ there exists an object $GX$ in $\cC$ and an isomorphism $\alpha_{X}\colon F(GX) \to X$. Using a suitable form of axiom of choice, we choose such a pair $(GX, \alpha_X)$ for every $X$ in $\cD$.

We want to make the construction $X \mapsto GX$ into a functor (which will be a quasi-inverse to $F$). For this, we need to define for every $f\colon X\to Y$  a map $Gf\colon GX \to GY$. Note that the map $f$ induces a map
\[
	 \alpha_Y^{-1} f\alpha_X \colon F(GX) \to F(GY).
\]
But since the functor $F$ is full and faithful the map
\[
	\Hom_{\cC}(GX,GY) \to \Hom_\cD(F(GX),F(GY)),\, g \mapsto Fg
\]
is a bijection. Hence there exists a unique $g \colon GX \to GY$ with $Fg = \alpha_Y^{-1} f \alpha_X$, and we define $Gf := g$. One verifies that this $G$ defines indeed satisfies the axioms for a functor from $\cD$ to $\cC$.

Finally, to see that $G$ is a quasi-inverse to $F$, note that the $\alpha_X$ define an isomorphism of functors $\alpha\colon FG \to \id_{\cD}$. In the other direction, given an object $X$ in $\cC$ we define
\[
	\beta_X \colon GFX \to X
\]
to be the pre-image of $\alpha_{FX}$ under the bijection
\[
	\Hom_\cC(GFX,X) \to \Hom_\cD(FGFX,FX).
\]
Then one verifies that the $\beta_X$ define an isomorphism of functors $GF \to \id_{\cC}$, and we conclude that $F$ is an equivalence of categories.
\end{proof}



\begin{example}
Let $K$ be a field, let $\FVec_K$ be the category of finite-dimensional $K$-vector spaces. Let $\cC$ be the category of matrices over $K$, see Example \ref{exa:category-of-matrices}. Consider the functor
\[
	F\colon  \cC \to \FVec_K
\]
that maps an object $n$ of $\cC$ to the vector space $K^n$ and a matrix
\[
	A \in \Hom_\cC(n,m) = \Mat_{m,n}(K)
\]
to the corresponding linear map $K^n \to K^m$. We claim that $F$ is an equivalence of categories.

Indeed, using Theorem \ref{thm:equivalence-of-categories} it suffices to observe that the functor is essentially surjective, since every finite-dimensional vector space is isomorphic to $K^n$ for some $n$, and that the functor is full and faithful, since for every $m$ and $n$ the map
\[
	\Mat_{m,n}(K) \to \Hom_K(K^n,K^m)
\]
is a bijection.

Note that $\cC$ and $\FVec_K$ are not \emph{isomorphic} as categories. The category $\FVec_K$ is much bigger than $\cC$ since for every $n$ there are infinitely many $n$-dimensional vector spaces (all isomorphic, but not \emph{equal}). In fact, $\cC$ is a small category, whereas $\FVec_K$ is not.
\end{example}



\newpage
\section*{Exercises}




\begin{exercise}\label{exc:endo-identity-ab}
Let $\eta \colon \id_\Ab \to \id_\Ab$ be a morphism of functors. Show that there is an integer $n$ such that  for every $A \in \ob \Ab$ and for every $x \in A$ the identity $\eta_A(x) =nx$ holds.
\end{exercise}



\begin{exercise}
Show that taking determinants defines a morphism 
\[
	\det\colon\!\GL_n \to \GL_1
\]
between functors from $\CRing$ to $\Grp$. (See Exercise~\ref{exc:functor-GLn}).
\end{exercise}


\begin{exercise}
Let $G$ and $H$ be groups, and $\rB G$ and $\rB H$ the corresponding one-object categories (see Example \ref{exa:BG}). Show that a functor $F\colon \rB G \to \rB H$ is the same as a group homomorphism $f\colon G\to H$, and a morphism of functors $\eta\colon F_1 \to F_2$ is the same as an element $h\in H$ such that $h f_1(g) h^{-1} = f_2(g)$ for all $g\in G$.
\end{exercise}


\begin{exercise}
Let $R$ be a ring. Recall that the \emph{center} of a ring is the subring
\[
	Z(R) = \{ z \in R \mid zr=rz \text{ for all $r \in R$} \}.
\]
Denote by $\cC$ the category of left $R$-modules.
\begin{enumerate}
\item Let $z\in Z(R)$. Show that $\eta_{z,M} \colon M \to M,\, x \mapsto zx$ defines a morphism of functors $\eta_z\colon \id_\cC \to \id_\cC$.
\item Let $\eta \colon \id_\cC \to \id_\cC$ be a morphism of functors. Show that there is a $z\in Z(R)$ with $\eta = \eta_z$.
\end{enumerate}
\end{exercise}

\begin{exercise}
Show that there are precisely two morphisms of functors $\id_\Grp \to \id_\Grp$.  
\end{exercise}

\begin{exercise}
Consider categories and functors as in the following diagram:
\[
\begin{tikzcd}
\cC \arrow[bend left]{r}{F_0} \arrow[bend right,swap]{r}{F_1} & \cD \arrow{r}{G} & \cE
\end{tikzcd}
\]
Let $\eta\colon F_0 \to F_1$ be a morphism of functors. Construct a morphism of functors $GF_0 \to GF_1$.
\end{exercise}

\begin{exercise}[Equivalence is an equivalence relation]
Let $F\colon \cC \to \cD$ and $G\colon \cD\to \cE$ be equivalences of categories. Show that $GF\colon \cC \to \cE$ is an equivalence of categories.\end{exercise}

\begin{exercise}\label{exc:fully-faithful-isomorphism}
Let $F\colon \cC \to \cD$ be a full and faithful functor. Let $X$ and $Y$ be objects in $\cC$. 
\begin{enumerate}
\item Let $f \colon X\to Y$ be a morphism. Show that $f$ is an isomorphism if and only if $Ff$ is an isomorphism.
\item Show that $X$ and $Y$ are isomorphic if and only if $FX$ and $FY$ are isomorphic.
\end{enumerate}
\end{exercise}

\begin{exercise}
For a category $\cC$, we denote by $[\cC]$ the class of isomorphism classes of objects in $\cC$. 
Let $F\colon \cC \to \cD$ be a functor. Show that $F$ induces a map $[F]\colon [\cC] \to [\cD]$. Show that if $F$ is fully faithful, then $[F]$ is injective, and if $F$ is essentially surjective, then $[F]$ is surjective.
\end{exercise}

\begin{exercise}
Let $\cC$ be a non-empty locally small category in which all objects are isomorphic and in which every morphism is an isomorphism. Show that there is a group $G$ and an equivalence of categories $\rB G \to \cC$.
\end{exercise}


\begin{exercise}[Fundamental groupoid]
Let $X$ be a topological space. Let $\Pi_1(X)$  be the category with
\begin{enumerate}
\item objects: $\ob \Pi_1(X) = X$ 
\item morphisms: $\Hom_{\Pi_1(X)}(x,y)$ the set of homotopy classes of paths from $x$ to $y$
\item composition: composition of paths
\end{enumerate}
Verify that this indeed defines a category. It is called the \emph{fundamental groupoid} of $X$.

Assume that $X$ is path connected, and let $x\in X$. Show that $\Pi_1(X)$ is equivalent with the category $\rB \pi_1(X,x)$. (See Example \ref{exa:BG}).
\end{exercise}



\begin{exercise}
Give for every $n\in \bN$ a category $\cC_n$ so that $\cC_n$ has exactly $n$ objects, and all the $\cC_n$ are equivalent.
\end{exercise}


\begin{exercise} \label{exc:equivalence-final}
Let $F\colon \cC \to \cD$ be an equivalence of categories. Show that $\cC$ has a final object if and only if $\cD$ has a final object.
\end{exercise}


\begin{exercise}
Let $R$ and $S$ be rings. Show that the categories ${}_R\Mod\times {}_S\Mod$ and ${}_{R\times S}\Mod$ are equivalent.
\end{exercise}


\begin{exercise}[Morita equivalence ($\star$)]
Let $R$ be a ring and $n$ a positive integer.  If $M$ is an $R$-module, then we can consider elements of $M^n$ as length $n$ column matrices with entries in $M$. In this way, we have an action
\[
	\Mat_n(R) \times M^n \to M^n,\, (A,    \begin{pmatrix} x_1 \\ x_2 \\ \vdots \\ x_n \end{pmatrix} ) \mapsto
	A \cdot   \begin{pmatrix} x_1 \\ x_2 \\ \vdots \\ x_n \end{pmatrix}
\]
This makes $M^n$ into a left $\Mat_n(R)$-module. Verify that this defines a functor
\[
	{}_R\Mod \to {}_{\Mat_n(R)}\Mod,\, M \mapsto M^n.
\]
Show that this functor is an equivalence of categories.
\end{exercise}

%\begin{exercise}[$\star$]
%Let $R$ and $S$ be commutive rings. Assume that the categories ${}_R\Mod$ and ${}_S\Mod$ are equivalent. Prove that $R$ and $S$ are isomorphic. (Hint: do exercise \ref{exc:endo-identity-ab} first). 
% WARNING: addition on End(id_C) needs additive structure on C
%\end{exercise}

%\begin{exercise}[$\star$]\label{exc:equivalence-preserves-mono}
%Let $F\colon \cC \to \cD$ be an equivalence of categories. Let $f\colon X\to Y$ be a morphism in $\cC$. Show that $f$ is a monomorphism (epimorphism) if and only if $F(f)$ is a monomorphism (epimorphism).
%\end{exercise}



\begin{exercise}[Abelianized fundamental group without base point ($\star$)]
Let $\cC$ be the category of path connected topological spaces. Let $P$ be the functor
from $\Top_\ast$ to $\cC$ that maps a pair $(X,x)$ the the path component of $x\in X$.
Show that there is a functor $F\colon \cC \to \Ab$
and an isomorphism between the functors
\[
	\pi_1^\ab \colon \Top_\ast \to \Ab,\, (X,x) \mapsto \pi_1(X,x)^\ab
\]
and $F\circ P$. Bonus question: show that there is \emph{no} functor $F\colon \cC \to \Grp$ and 
an isomorphism between $F\circ P$ and $\pi_1$.
\end{exercise}




%%%%%%%%%%%%%%%%%%%%%%%%%%
% TENSOR PRODUCT
%%%%%%%%%%%%%%%%%%%%%%%%%%



\chapter{Tensor product}


\section{Tensor product of a right and a left module}

\begin{definition}
Let $R$ be a  ring, $M$ a right $R$-module, and $N$ a left $R$-module. Let $A$ be an abelian group. A map
\[
	f\colon M\times N \to A
\]
is said to be $R$-bilinear if for all $x, x_1,x_2\in M$, $y, y_1,y_2\in N$ and $r\in R$ the following hold:
\begin{enumerate}
\item $f(x_1+x_2,y)=f(x_1,y) + f(x_2,y)$
\item $f(x,y_1+y_2) = f(x,y_1) + f(x,y_2)$
\item $f(xr,y)=f(x,ry)$ 
\end{enumerate}
\end{definition}

\begin{remark}
These conditions imply that moreover
\begin{enumerate}
\item [(4)] $f(x,0)=0$, and
\item[(5)] $f(0,y) =0$
\end{enumerate}
hold for all $x\in M$ and $y\in N$. See Exercise~\ref{exc:bilinear-zero}.
\end{remark}


Note that if $f\colon M\times N\to A$ is bilinear, and $h\colon A\to B$ is a homomorphism of abelian groups, then the composition $hf\colon M\times N \to B$ is also a bilinear map. The following theorem states that there is a `universal' bilinear map, from which all others can be obtained by a unique composition with a homomorphism of abelian groups.

\begin{theorem}\label{thm:existence-of-tensor-product}
Let $R$ be a ring, $M$ a right $R$-module and $N$ a left $R$-module. Then there exists an abelian group $T$ and an $R$-bilinear map
\[
	g\colon M\times N \to T
\]
such that for every abelian group $A$ and every $R$-bilinear map $f\colon M\times N \to A$ there is a unique group homomorphism
$h\colon T \to A$ with $f=hg$:
\[
\begin{tikzcd}
M\times N\arrow{r}{g} \arrow{d}{f} & T \arrow[dashed]{ld}{h} \\
A
\end{tikzcd}
\]
Moreover, the pair $(T,g)$ is unique up to unique isomorphism in the following sense: if both $(T_1,g_1)$ and $(T_2,g_2)$ satisfy the above property, then there is a unique isomorphism $h\colon T_1 \to T_2$ such that $g_2=hg_1$.
\end{theorem}


\begin{definition}We will call the abelian group $T$ (unique up to unique isomorphism) the \emph{tensor product} of $M$ and $N$, and denote it by $M\otimes_R N := T $. For $x\in M$ and $y\in N$ we denote the image of $(x,y)$ in $M\otimes_R N$ by $x\otimes y := g(x,y)$. 
\end{definition}

% TODO: verplaatsen naar na het bewijs, korte discussie over wat 'het tensor product is' (iedereen mag eigen keuze maken, bvb die uit het bewijs, zolang maar aan universele eigenschap is voldaan. Vergelijking met reele getallen: je hoeft niet te weten of iemand anders' reele getallen Dedekindsneden of cauchy-rij-klassen zijn.

\begin{proof}[Proof of Theorem \ref{thm:existence-of-tensor-product}]
\emph{Uniqueness}. This is purely formal:  everything defined by a universal property is unique up to unique isomorphism, by an argument  that is basically the same as the proof of Proposition \ref{prop:final-object-uniquely-unique}: 
Assume $(T_1,g_1)$ and $(T_2,g_2)$ both satisfy the required property. Since $g_1\colon M\times N \to T_1$ is bilinear there is a unique map $h_1\colon T_2 \to T_1$ with $g_1=h_1g_2$. Reversing the roles of $T_1$ and $T_2$, we get a unique map $h_2\colon T_1\to T_2$. Moreover, both the compositions $h_1h_2$ and $h_2h_1$ must be the identity, so we conclude that $h_1$ is an isomorphism.

\emph{Existence}. This part is certainly not formal! The proof is a bit messy, but in a way very natural: we just construct an abelian group $T$ (and a map $g$) with all the desired properties built-in. 

Let $F:=\bZ^{(M\times N)}$ be the free $\bZ$-module on the set $M\times N$. Given an element $(x,y) \in M\times N$, we denote by $e_{(x,y)} \in \bZ^{(M\times N)}$ the corresponding basis vector, see \ref{ex:free-module}. There is a canonical map of sets
\[
	M\times N \to F,\, (x,y) \mapsto e_{(x,y)}.
\]
This map has no reason to be bilinear. We will force it to become bilinear by dividing out the necessary relations. Let $G\subset F$ be the subgroup generated by the elements
\begin{gather*}
	e_{(x_1+x_2,y)}-e_{(x_1,y)}-e_{(x_2,y)},\\
	e_{(x,y_1+y_2)}-e_{(x,y_1)}-e_{(x,y_2)},\\
	e_{(xr,y)}-e_{(x,ry)},
\end{gather*}
for all $x_1,x_2,x\in M$ and $y_1,y_2,y\in N$ and $r\in R$. Let $T$ be the quotient group $F/G$, and consider the composition
\[
\begin{tikzcd}
M\times N \arrow{r} \arrow{rd}{g} & F \arrow{d} \\
& T
\end{tikzcd}
\]
Then the map $g$ is bilinear by construction.

We now show that $(T,g)$ is a tensor product. Let $f\colon M\times N \to A$ be a billinear map. Then there is a unique homomorphism
$f' \colon F \to A$, which sends the basis vector $e_{(x,y)}$ to $f(x,y)$, see Proposition \ref{prop:universal-property-free-module}.
Since $f$ is bilinear, we have that $f'$ vanishes on all the generators of $G$, and therefore that $f'(G)=0$. Hence $f'$ induces a homomorphism $h\colon T \to A$ with $f=hg$. To see that a map $h$ with this property is unique, note that $T$ is generated by the images of the elements $(x,y)\in M\times N$, and that $h$ must send the image of $(x,y)$ to $f(x,y)$.
\end{proof}


\begin{remark}
In practice it is often easier not to use the actual construction of the tensor product in the proof of Theorem \ref{thm:existence-of-tensor-product}, but only the defining universal property in the statement of Theorem  \ref{thm:existence-of-tensor-product}, together with the fact that the tensor product is generated by the elements of the form $x\otimes y$ with $x\in M$ and $y\in N$.
\end{remark}

\begin{remark}
Elements of $M\otimes_R N$ are finite sums of elements of the form $x\otimes y$, but these are not independent. In fact, 
the map
\[
	M\times N \to M\otimes_R N,\, (x,y) \mapsto x\otimes y
\]
is $R$-bilinear (by definition of the tensor product), so that for all  $x, x_1,x_2\in M$, $y,y_1,y_2\in N$ and $r\in R$  the identities
\begin{gather*}
	(x_1+x_2) \otimes y = (x_1 \otimes y) + (x_2 \otimes y) \\
	x\otimes (y_1+y_2) = (x \otimes y_1) + (x \otimes y_2) \\
	(xr) \otimes y = x \otimes (ry) \\
	x \otimes 0 = 0 \\
	0 \otimes y = 0 
\end{gather*}
hold in $M\otimes_R N$. 
\end{remark}


We end this section with a few examples of tensor products.

\begin{example}
Let $M$ be a right $R$-module, then we claim $M \otimes_R \{0\} \cong \{0\}$. Indeed, the tensor product is generated as an abelian group by the elements $x\otimes 0$, but these all are equal to $0$ in $M\otimes_R \{0\}$. 
\end{example}


\begin{example}\label{exa:tensor-with-R}
We claim that for any left $R$-module $M$ there is a unique isomorphism 
\[
	f\colon R\otimes_R M \isomto M
\]
satisfying
\[
	r\otimes x \mapsto rx.
\]
Indeed, the map $R\times M \to M,\, (r,x) \mapsto rx$
is $R$-bilinear, and hence induces an $R$-linear homomorphism
$f\colon R\otimes_R M \to M$ with $f(r\otimes x)=rx$. Conversely, the map
$M \to R\otimes_R M$ given by $x \mapsto 1\otimes x$
is $R$-linear, and is a two-sided inverse to the map $f$.

Of course, in the same way one can produce an isomorphism
\[
	 M\otimes_R R \isomto M,\, x \otimes r \mapsto xr
\]
for every \emph{right} $R$-module $M$.
\end{example}

\begin{example}
The tensor product of two non-zero modules can be zero. For example, we have
\[
	(\bZ/2\bZ) \otimes_\bZ (\bZ/3\bZ) = 0.
\]
Indeed, the tensor product is generated by elements of the form $x\otimes y$ with $x\in \bZ/2\bZ$ and $y\in \bZ/3\bZ$. But since
$3x=x$ for all $x\in \bZ/2\bZ$ we find 
\[
	x\otimes y = (3x) \otimes y  = x \otimes (3y) = x\otimes 0 =  0.
\]
Similarly we have $(\bZ/n\bZ)\otimes_\bZ (\bZ/m\bZ)=0$ whenever $n$ and $m$ are co-prime. See also Exercise \ref{exc:tensor-Z-mod-nZ}.
\end{example}

% --- updated towards noncommutative setting up to here

\section{Tensor products and bimodules}

In many cases, the tensor product of two modules is not just an abelian group, but itself again a module. 

\begin{definition}
Let $R$ and $S$ be rings.
An \emph{$(R,S)$-bimodule} is an abelian group $M$ equipped with operations
\[
	R\times M \to M,\, (r,x) \mapsto rx
\]
and 
\[
	M \times S \to M,\, (x,s) \mapsto xs
\]
such that 
\begin{enumerate}
\item[(B1)] the first operation makes $M$ into a left $R$-module 
\item[(B2)] the second operation makes $M$ into a right $S$-module
\item[(B3)] for all $r\in R$, $s\in S$ and $x\in M$ the identity $r(xs)=(rx)s$ holds in $M$.
\end{enumerate}
A map  $f\colon M \to N$  is called a \emph{morphism} of $(R,S)$-bimodules if it is both a morphism
of left $R$-modules and a morphism of right $S$-modules. We denote the category of $(R,S)$-bimodules by ${}_R\Mod_S$.
\end{definition}

Because of axiom (B3), we can simply write $rxs$ for $r(xs)=(rx)s$, and we will frequently describe the structure of an $(R,S)$-bimodule by the map
\[
	R\times M \times S \to M,\, (r,x,s) \mapsto rxs,
\]
simultaneously encoding the left and right module structures.

\begin{example}An abelian group $A$ has a unique structure of $(\bZ,\bZ)$-bimodule, which is given by 
$nam := (nm)a$ for all $n,m\in \bZ$ and $a\in A$. It follows that a $(\bZ,\bZ)$-bimodule is the same thing as an abelian group. Similarly, an $(R,\bZ)$-bimodule is the same as a left $R$-module, and a $(\bZ,R)$-bimodule is the same as a right $R$-module.
\end{example}

\begin{example}$R$ be a \emph{commutative} ring and $M$ an $R$-module. Then $M$ also is an $(R,R)$-bimodule by setting
\[
	rxs := rsx
\]
for all $r,s\in R$ and $x\in M$.
\end{example}

\begin{example}\label{exa:Rn-bimodule}
Let $R$ be a ring and $n$ a non-negative integer. Then $R^n$ becomes an $(R,R)$-bimodule with
\[
	r(x_1,\ldots, x_n)s := (rx_1s, \ldots, rx_ns).
\]
\end{example}

\begin{example}Let $R$ be a commutative ring and $n$ and $m$ non-negative integers. Then the additive group $M=\Mat_{m,n}(R)$ of $n$ by $m$ matrices is a $(\Mat_n(R),\Mat_m(R))$-bimodule where for $A\in \Mat_{n}(R)$, $B\in \Mat_{m}(R)$ and $X\in M$ the element $AXB$ of $M$ is defined by matrix multiplication.
\end{example}


\begin{example}\label{exa:Hom-bimodule}
Let $M$ a left $S$-module and $N$ be a left $R$-module. Consider the group $\Hom(M,N)$ of homomorphisms of abelian groups. This group carries a natural structure of $(R,S)$-bimodule with
\[
	rfs := \big[ M \to N,\, x \mapsto rf(sx) \big]
\] 
for all $r\in R$, $f\in \Hom(M,N)$ and $s\in S$.
\end{example}

\begin{proposition}
Let $R$, $S$, and $T$ be rings. Let $M$ be an $(R,S)$-bimodule, and let $N$ be an $(S,T)$-bimodule. Then the tensor product $M\otimes_S N$ has a unique structure of an $(R,T)$-bimodule satisfying
\[
	r(x\otimes y)t = (rx) \otimes (yt)
\]
for all $r\in R$, $x\in M$, $y\in N$ and  $t\in T$. 
\end{proposition}

If $R$ is a commutative ring then any $R$-module is canonically an $(R,R)$-bimodule, and we find that
the tensor product of two $R$-modules over $R$ is naturally an $R$-module. When dealing with commutative rings, there is no essential distinction between left and right modules, and we will usually treat the tensor product as an operation that produces a left $R$-module out of two left $R$-modules.





\section{Tensor product as a functor}

Let $R$ be a ring and let $f\colon M_1\to M_2$ be a morphism of right $R$-modules, and let $g\colon N_1\to N_2$ be a morphism of left $R$-modules. Then the map
\[
	M_1\times N_1 \to M_2\otimes_R N_2,\, (x,y) \mapsto f(x)\otimes g(y)
\]
is $R$-bilinear. Hence, by the universal property of the tensor product, there exists a unique homomorphism of abelian groups
\[
	f\otimes g\colon M_1\otimes_R N_1 \to M_2\otimes_R N_2
\]
that maps an element $x\otimes y$ to $f(x)\otimes g(y)$. This upgrades  the tensor product from being just a construction on pairs of modules to a functor
\[
	-\otimes_R-\colon \Mod_R \times {}_R\Mod \to \Ab.
\]
Similarly, if $R$, $S$ and $T$ are rings, then we have a functor
\[
	-\otimes_S- \colon {}_R\Mod_S \times {}_S\Mod_T \to {}_R\Mod_T.
\]
	
\begin{proposition}\label{prop:tensor-right-exact}
Let $R$ be a ring and let $N$ be a left $R$-module. If 
\begin{equation}\label{eq:pre-tensor-right-exact}
	M_1 \overset{f}{\longto} M_2 \overset{g}{\longto} M_3 \longto 0
\end{equation}
is an exact sequence of right $R$-modules, then the induced sequence
\begin{equation}\label{eq:tensor-right-exact}
	M_1\otimes_R N \overset{f\otimes \id}{\longto}
	M_2\otimes_R N \overset{g\otimes \id}{\longto}
	M_3\otimes_R N \longto 0
\end{equation}
of abelian groups is exact. Similarly, if
\[
	N_1 \longto N_2 \longto N_3 \longto 0
\]
is an exact sequence of left $R$-modules, and $M$ a right $R$-module, then the induced sequence
\[
	M \otimes_R N_1 \longto M \otimes_R N_2 \longto M \otimes_R N_3 \longto 0
\]
is exact.
\end{proposition}

The functor $-\otimes_R N$ in general does \emph{not} preserve short exact sequences: even if
\[
	0\to M_1 \to M_2 \to M_3 \to 0
\]
is a short exact sequence, then we only obtain a partial exact sequence
\[
	 M_1\otimes_R N \to M_2\otimes_R N \to M_3\otimes_R N \to 0.
\]
See Exercise \ref{exc:tensor-not-exact}.

\begin{proof}[Proof of Proposition \ref{prop:tensor-right-exact}]
For all $x\in M_1$ and $y\in N$ we have $gf(x)\otimes y=0$, hence $\im (f\otimes \id) \subset \ker (g\otimes \id)$. In particular, we have an induced map
\begin{equation}\label{eq:tensor-exact-induced}
	\Phi\colon \frac{M_2\otimes_R N}{\im (f\otimes \id)} \longto M_3 \otimes_R N.
\end{equation}
To show that the sequence (\ref{eq:tensor-right-exact}) is exact, it suffices to show that the above map is an isomorphism.  We will verify this by constructing an inverse map.

For every $x\in M_3$ choose an $x'\in M_2$ with $g(x')=x$ (note that $g$ is surjective). We define a map
\[
	  M_3 \times N \to \frac{M_2\otimes_R N}{\im (f\otimes \id)},\,  (x,y) \mapsto x'\otimes y.
\]
This is well-defined, since if $x''$ is another element with $g(x'')=x$, then
\[
	(x''\otimes y) - (x'\otimes y) = (x''-x') \otimes y \in \im (f\otimes \id),
\]
using the exactness of the original sequence (\ref{eq:pre-tensor-right-exact}).
Moreover, the map is bilinear, so it induces a homomorphism 
\[
	M_3\otimes_R N \to \frac{M_2\otimes_R N}{\im (f\otimes \id)}
\]
and one verifies that this is a two-sided inverse to the map $\Phi$.
\end{proof}

\begin{example}
Proposition \ref{prop:tensor-right-exact} can be a powerful tool in computing tensor products. As an example, let us use it to compute $M\otimes_\bZ (\bZ/n\bZ)$ for any $\bZ$-module $M$. We have an exact sequence
\[
	\bZ \overset{n}{\longto} \bZ \longto \bZ/n\bZ \longto 0
\]
of $\bZ$-modules, which induces an exact sequence
\[
	\bZ\otimes_\bZ M \overset{n\otimes \id}{\longto} \bZ \otimes_\bZ M
	 \longto \bZ/n\bZ \otimes_\bZ M \longto 0.
\]
We have $\bZ\otimes_\bZ M \cong M$ (see Example \ref{exa:tensor-with-R}), and the above exact sequence is isomorphic with the exact sequence
\[
	M \overset{n}{\longto} M \longto \bZ/n\bZ \otimes_\bZ M \longto 0,
\]
from which we conclude that $\bZ/n\bZ \otimes_\bZ M$ is isomorphic with $M/nM$.
\end{example}


\section{The adjunction}

Let $R$ and $S$ be rings. If $N$ is an $(R,S)$-bimodule, and $P$ a right $S$-module, then $\Hom_S(N,P)$ is naturally a right $R$-module, with the action of $R$ defined by:
\[
	fr\colon N \to P,\, x \mapsto f(rx).
\]
See Exercise \ref{exc:hom-bimodule}.


\begin{theorem}[Tensor-Hom adjunction]\label{thm:tensor-hom-adjunction}
Let $R$ and $S$ be rings. Let $M$ be a right $R$-module, $N$ an $(R,S)$-bimodule, and $P$ be a right $S$-module. Then the map of abelian groups
\[
	\Hom_S(M\otimes_R N, P) \to \Hom_R(M,\Hom_S(N,P))
\]
given by
\[
	f \mapsto \big( x \mapsto ( y \mapsto f(x\otimes y) )\big)
\]
is an isomorphism.
\end{theorem}

\begin{remark}
The $\Hom_R$ and $\Hom_S$ in the theorem denote the set of homomorphisms in the categories of \emph{right} $R$- and $S$-modules. 
\end{remark}


\begin{proof}[Proof of Theorem \ref{thm:tensor-hom-adjunction}]
Given an $R$-linear map $f\colon M\to \Hom_S(N,P)$ we obtain a map
\[
	M\times N \to P,\,  (x,y) \mapsto f(x)(y)
\]
which is $R$-bilinear, hence it defines a homomorphism
\[
	f'\colon M\otimes_R N \to P
\]
with the property that it maps $x\otimes y$ to $f(x)(y)$. This map is $S$-linear. This construction defines a homomorphism 
\[
	\Hom_R(M,\Hom_S(N,P)) \to \Hom_S(M\otimes_R N, P),\, f\mapsto f'.
\]
We leave it to the reader to verify that this is a two-sided inverse to the map in the theorem.
\end{proof}

\begin{remark}\label{thm:alternate-adjunction}
There is completely analogous theorem about tensoring on the left with a fixed $(S,R)$-bimodule $N$. It gives an isomorphism
\[
	\Hom_S(N\otimes_R M, P) \to \Hom_R(M, \Hom_S(N,P))
\]
where $M$ is a left $R$-module and $P$ a left $S$-module. In this version, $\Hom_S$ and $\Hom_R$ denote the sets of homomorphisms in the categories of \emph{left} $S$- and $R$-modules.
\end{remark}

%\section{Restriction and extension of scalars}\label{sec:restriction-extension-of-scalars}
%
%Let $f\colon R\to S$ be a morphism of rings.
%If $M$ is an $S$-module, then $M$ also has the structure of an $R$-module, by $rx := f(r)x$. We obtain a functor
%\[
%	{}_S\Mod \to {}_R\Mod,\, M \mapsto M
%\]
%which is called \emph{restriction of scalars}. 
%
%\begin{example}
%If $f\colon \bR \to \bC$ is the inclusion, then the corresponding restriction of scalars functor maps an $n$-dimensional complex vector space $V$, to the $2n$-dimensional real vector space $V$.
%\end{example}
%
%\begin{example}
%If $f\colon \bZ\to R$ is the canonical ring morphism from $\bZ$ to a commutative ring, then the restriction of scalars functor coincides with the forgetful functor
%\[
%	{}_R\Mod \to \Ab,\, M \mapsto M
%\]
%which maps a module to the underlying abelian group.
%\end{example}
%
%There is also a construction in the other direction.  Let $M$ be a (left) $R$-module. Note that $S$ is an $(S,R)$-bimodule where $s\in S$ acts by left multiplication on $S$, and $r\in R$ by right multiplication with $f(r)$. Hence the tensor product $S\otimes_R M$ is a left $S$-module and we obtain a functor
%\[
%	{}_R\Mod \to {}_S\Mod,\, M \mapsto S\otimes_R M
%\]
%called \emph{extension of scalars}. 
%
%\begin{example}
%If $M$ is free with basis $x_1,\ldots, x_n$, then $S\otimes_R M$ is free with basis $1\otimes x_1,\ldots,1\otimes x_n$. That is, every element $y\in S\otimes_R M$ can be uniquely written as
%\[
%	y = s_1\otimes x_1 + \cdots + s_n \otimes x_n
%\]
%with $s_i \in S$. For example, if $f\colon \bR \to \bC$ is the inclusion and $V=\bR^n$ then we have $\bC\otimes_\bR V \cong \bC^n$ as complex vector spaces.
%\end{example}
%
%
%
%\begin{proposition}[Universal property of extension of scalars]\label{prop:universal-property-extension-of-scalars}
%Let $f\colon R\to S$ be a ring homomorphism. Let $M$ be a left $R$-module and let $N$ be a left $S$-module. Let $\varphi\colon M\to N$ be an $R$-linear map. Then there exists a unique $S$-linear map $\tilde\varphi\colon S\otimes_R M \to N$ such that the triangle
%\[
%\begin{tikzcd}
%S\otimes_R M \arrow[dashed]{r}{\tilde\varphi} & N \\
%M \arrow{u}{x\mapsto 1\otimes x} \arrow[swap]{ru}{\varphi}
%\end{tikzcd}
%\]
%commutes.
%\end{proposition}
%
%\begin{proof}We first prove uniqueness. For all $x\in M$ we must have
%\[
%	\tilde\varphi(s\otimes x) = s\tilde\varphi(1\otimes x) = s\varphi(x),
%\]
%and since $S\otimes_R M$ is generated by elements $s\otimes x$ we see that there can be at most one $\tilde\varphi$. To show existence, note that
%\[
%	S\times M \to N,\, (s,x) \mapsto s\varphi(x)
%\]
%is $R$-bilinear and hence defines a map
%\[
%	\tilde\varphi\colon S\otimes_R M \to N
%\]
%satisfying $\tilde\varphi(s\otimes x) = s\varphi(x)$. For all $s,s'\in S$ and $x\in M$ we have
%\[
%	\tilde\varphi(s'(s\otimes x)) = \tilde\varphi(s's\otimes x) = s's\varphi(x) = s'(s\varphi(x)),
%\]
%which shows that $\tilde\varphi$ is indeed $S$-linear. 
%\end{proof}
%
%A useful reformulation of the above proposition is the following corollary, which can also be seen as a special case of Theorem \ref{thm:tensor-hom-adjunction}, using the canonical isomorphism $\Hom_S(S,N)=N$.
%
%\begin{corollary}[Adjunction between extension and restriction of scalars]
%Let $f\colon R\to S$ be a ring homomorphism. Let $M$ be an $R$-module and let $N$ be an $S$-module. Then the map
%\[
%	\Hom_S(S\otimes_R M,N) \to \Hom_R(M, N),\,
%	f \mapsto \left( x \mapsto f(1\otimes x) \right)
%\]
%is a bijection.\qed
%\end{corollary}
%
%


\newpage
\section*{Exercises}

% TODO: if R commutative, show that bilinear map to A implies A is R-module, and is bilinear in the usual sense.  FALSE! Can postcompose with any additive map A-->A'.

\begin{exercise}\label{exc:bilinear-zero}
Prove that if $f\colon M\times N \to A$ is $R$-bilinear, then $f(x,0)=0$ for all $x\in M$ and $f(0,y)=0$ for all $y\in N$.
\end{exercise}


\begin{exercise}Let $R$ be a ring, let $M$ be a left $R$-module and let $n$ be a non-negative integer.  
\begin{enumerate}
\item Show that the map $R^n \times M \to M^n$ given by 
\[((r_1,\ldots,r_n),x) \mapsto (r_1x,\ldots, r_nx)\]
is $R$-bilinear.
\item Show that the above map is universal, and conclude that $R^n \otimes_R M \cong M^n$.
\end{enumerate}
\end{exercise}
%
%
%\begin{exercise}Let $R$ be a commutative ring. Show that $R^n \otimes_R R^m$ is isomorphic to $R^{nm}$ as $R$-module.
%\end{exercise}
%
\begin{exercise} \label{exc:tensor-Z-mod-nZ}
Let $n$ and $m$ be positive integers with greatest common divisor $d$. Show that 
\begin{enumerate}
\item $(\bZ/n\bZ) \otimes_\bZ (\bZ/m\bZ) \cong \bZ/d\bZ$;
\item $\bQ\otimes_\bZ (\bZ/m\bZ) \cong 0$;
\item $\bQ \otimes_\bZ \bQ \cong \bQ$.
\end{enumerate}
\end{exercise}

\begin{exercise}
Let $K$ be a field and $R:=\Mat_n(K)$ the ring of $n$ by $n$ matrices. Let $M=K^n$ be the right $R$-module of length $n$ row vectors, and $N=K^n$ the left $R$-module of length $n$ column vectors. Show that the abelian group $M\otimes_R N$ is isomorphic to $K$.
\end{exercise}


\begin{exercise}
Let $R$ be a commutative ring and $f,g\in R$. Show that
\[
	R/(f) \otimes_R R/(g) \cong R/(f,g)
\]
as $R$-modules.
\end{exercise}

\begin{exercise}
Let $R$ be a ring, let $I\subset R$ be a (two-sided) ideal and let $M$ be a left $R$-module. Show that 
$(R/I) \otimes_R M$ is isomorphic to $M/IM$.
\end{exercise}
\begin{exercise}
Let $R$ be a commutative ring. Show that the functors ${}_R\Mod \times {}_R\Mod \to {}_R\Mod$ given by $(M,N) \mapsto M\otimes_R N$ and $(M,N) \mapsto (N\otimes_R M)$ are isomorphic. 
\end{exercise}



\begin{exercise}Verify that the bimodule in Example \ref{exa:Hom-bimodule} indeed satisfies the bimodule axioms.
\end{exercise}

\begin{exercise}\label{exc:hom-bimodule}
Let $R$ and $S$ be rings. Let $M$ be a left $R$-module and $N$ an $(R,S)$-bimodule. Show that $\Hom_R(M,N)$ is naturally a right $S$-module and that $\Hom_R(N,M)$ is naturally a left $S$-module.
\end{exercise}


\begin{exercise}
Let $R$ be a ring and let $M_1$, $M_2$ be right $R$-modules and $N$ a left $R$-module. Show that there is a unique isomorphism
\[
	 (M_1\oplus M_2) \otimes_R N \to (M_1\otimes_R N) \oplus (M_2 \otimes_R N)
\]
such that 
\[
	(x,y)\otimes z \mapsto (x\otimes z, y\otimes z)
\]
for all $x\in M_1$, $y\in M_2$ and $z\in N$.
\end{exercise}


\begin{exercise}
Let $R$ and $S$ be rings, let $M_1$ be a right $R$-module, $M_2$ an $(R,S)$-bimodule, and $M_3$ a left $S$-module. Show that there is a unique isomorphism
\[
	 (M_1\otimes_R M_2) \otimes_S M_3 \to M_1 \otimes_R (M_2 \otimes_S M_3)
\]
such that 
\[
	(x\otimes y)\otimes z \mapsto x\otimes(y\otimes z)
\]
for all $x\in M_1$, $y\in M_2$ and $z\in M_3$.
\end{exercise}



\begin{exercise}
Let $R$ be an integral domain with fraction field $K$. Let $M$  be an $R$-module. Show that every element of $K\otimes_R M$ is of the form $\lambda\otimes x$ with $\lambda \in K$ and $x\in M$.
\end{exercise}


\begin{exercise}
Let $R$ be a ring, $M$ a right and $N$ a left $R$-module. Show that there is a surjective morphism of 
abelian groups
\[
	M\otimes_\bZ N \to M\otimes_R N
\]
which maps $x\otimes y$ to $x\otimes y$. 
\end{exercise}

\begin{exercise}
Let $R$, $S$, $T$ be rings. Let $M$ be an $(R,S)$-bimodule and $N$ an $(S,T)$-bimodule. Let $P$ be an $(R,T)$-bimodule and let $f\colon M\times N\to P$ be an $S$-bilinear map. Let $h\colon M\otimes_S N \to P$ be the unique additive map such that $h(x\otimes y)=f(x,y)$ for all $x\in M$, $y\in N$. 
\begin{enumerate}
\item show that $h$ is a homomorphism of $(R,T)$-bimodules if and only if the bilinear map $f$ satisfies $f(rx,y) = rf(x,y)$ and $f(x,yt) = f(x,y)t$ for all $x\in M$, $y\in N$, $r\in R$ and $t \in T$. 
\item conclude that the map $M\times N \to M\otimes_S N$ is universal amongst bilinear maps $f\colon M\times N \to P$ to $(R,T)$-bimodules satisfying the identities in (1). 
\end{enumerate}
\end{exercise}


\begin{exercise}\label{exc:tensor-not-exact}
Consider the short exact sequence of $\bZ$-modules
\[
	0 \longto \bZ \overset{2}{\longto} \bZ \longto \bZ/2\bZ \longto 0.
\]
Show that the sequence obtained by applying the functor $-\otimes_\bZ \bZ/2\bZ$ is \emph{not} exact.
\end{exercise}



\begin{exercise}\label{exc:tensor-with-free-is-exact}
Let $R$ be a ring and let $0\to M_1\to M_2 \to M_3\to 0$ be a short exact sequence of right $R$-modules. Let $N$ be a \emph{free} left $R$-module. Show that the induced sequence
\[
	0 \to M_1\otimes_R N \to M_2 \otimes_R N \to M_3 \otimes_R N \to 0
\]
is a short exact sequence.
\end{exercise}


\begin{exercise}
Let $R$ be a ring and let $0\to M_1\to M_2 \to M_3\to 0$ be a split short exact sequence of right $R$-modules. Let $N$ be a left $R$-module. Show that the induced sequence
\[
	0 \to M_1\otimes_R N \to M_2 \otimes_R N \to M_3 \otimes_R N \to 0
\]
is a short exact sequence.
\end{exercise}

\begin{exercise}Let $R$ be a principal ideal domain and let
\[
	M \cong R^m \oplus R/p_1^{e_1}R \oplus \cdots \oplus R/p_n^{e_n}R
\]
 be a finitely generated $R$-module (with the notation of Corollary \ref{cor:structure-fg-mod-over-PID}). 
 \begin{enumerate}
 \item Let $K$ be the fraction field of $R$. Compute $\dim_K(K\otimes_R M)$.
 \item Let $p\in R$ be irreducible. Compute $\dim_{R/pR}(R/pR \otimes_R M)$.
 \end{enumerate}
\end{exercise}

%
%\begin{exercise}
%Deduce the universal property of extension of scalars (Proposition \ref{prop:universal-property-extension-of-scalars}) from the tensor-hom adjunction (in the form stated in Remark \ref{thm:alternate-adjunction}).
%\end{exercise}
%
%% TODO: add hint, be careful with left and right
%
%\begin{exercise}[Extension of scalars preserves finite generation]
%Let $f\colon R\to S$ be a morphism of rings. Let $M$ be a finitely generated $R$-module. Show that $S\otimes_R M$ is a finitely generated $S$-module.
%\end{exercise}
%
%\begin{exercise}[Restriction of scalars does not preserve finite generation]
%Give an example of a morphism $f\colon R\to S$ of commutative rings, and a finitely generated $S$-module $M$ such that $M$ is not finitely generated as an $R$-module.
%\end{exercise}
%
%
%\begin{exercise}
%Let $f\colon R\to R'$ and $g\colon R'\to R''$ be morphisms of rings, and consider the extension of scalar functors
%\[
%	{}_R\Mod \to {}_{R'}\Mod \to {}_{R''}\Mod.
%\]
%Show that the composite of these functors is isomorphic to the extension of scalar functor
%\[
%	{}_R\Mod \to {}_{R''}\Mod
%\]
%for the ring morphism $gf\colon R \to R''$.
%\end{exercise}
%
%\begin{exercise}
%Let $f\colon R\to S$ be a \emph{surjective} morphism of rings. Show that the restriction of scalars functor
%\[
%	{}_S\Mod \to {}_R\Mod,\, M \mapsto M
%\]
%is full and faithful. Give an example to show that it need not be essentially surjective. 
%\end{exercise}



%%%%%%%%%%%%%%%%%%%%%%%%%
% ADJOINT PAIRS OF FUNCTORS
%%%%%%%%%%%%%%%%%%%%%%%%%


\chapter{Adjoint functors}\label{chapter:adjoint-functors}

\section{Adjoint pairs of functors}

\begin{definition}
Let $F\colon \cC \to \cD$ and $G\colon \cD \to \cC$ be functors between locally small categories. An \emph{adjunction} between $F$ and $G$ is an isomorphism 
\[
	\alpha\colon \Hom_\cD( F(-), - ) \longisomto \Hom_\cC( -, G(-) )
\]
of functors $\cC^\opp \times \cD \to \Set$ (see Example \ref{exa:hom-in-two-arguments}). If such an adjunction exists, we say  $F$ is \emph{left adjoint} to $G$, and  $G$ is \emph{right adjoint} to $F$. 
\end{definition}


In other words, an adjunction from $F$ to $G$ consists of the data of a bijection
\begin{equation}\label{eqn:adjunction}
	\alpha_{X,Y}\colon \Hom_\cD( FX, Y ) \longisomto \Hom_\cC( X, GY ),
\end{equation}
for every $X$ in $\cC$ and $Y$ in $\cD$, such that for every $f\colon X_1\to X_2$ the square
\[
\begin{tikzcd}
\Hom_\cD( FX_1, Y ) \arrow{r}{\alpha_{X_1,Y}} & \Hom_\cC( X_1, GY ) \\
\Hom_\cD( FX_2, Y ) \arrow{r}{\alpha_{X_2,Y}} \arrow{u}{-\circ Ff}
	 & \Hom_\cC( X_2, GY ) \arrow{u}{-\circ f}
\end{tikzcd}
\]
commutes, and for every $g\colon Y_1\to Y_2$ the square
\[
\begin{tikzcd}
\Hom_\cD( FX, Y_1 ) \arrow{r}{\alpha_{X,Y_1}}  \arrow{d}{g\circ-}
	& \Hom_\cC( X, GY_1 ) \arrow{d}{Gg\circ -} \\
\Hom_\cD( FX, Y_2 ) \arrow{r}{\alpha_{X,Y_2}}
	 & \Hom_\cC( X, GY_2 )
\end{tikzcd}
\]
commutes. 

\begin{remark}
The terminology comes from an analogy with linear algebra: if $V$ and $W$ are vector spaces equipped with inner products, then linear maps $f\colon V\to W$ and $g\colon W\to V$ are called adjoint if we have
\[
	\langle f(v), w \rangle_W = \langle v, g(w) \rangle_V
\]
for all $v\in V$ and $w\in W$.
\end{remark}


Assume that $F\colon \cC \to \cD$ is a left adjoint of $G\colon \cD \to \cC$, with an adjunction $\alpha$. 
Taking $Y=FX$ in (\ref{eqn:adjunction}), we obtain a bijection 
\[
	\alpha_{X,FX}\colon \Hom_\cD(FX,FX) \longisomto \Hom_\cC(X,GFX).
\]
The image of $\id_{FX}$ under this map gives a map
\[
	\eta_X \colon X \to GFX
\]
in $\cC$. Using the fact that $\alpha$ is a morphism of functors, one shows that the $\eta_X$ form a morphism of functors
\[
	\eta\colon \id_\cC \to GF.
\]
Similarly, taking $X=GY$ in (\ref{eqn:adjunction}) we obtain a morphism of functors
\[
	\epsilon\colon FG \to \id_\cD.
\]
The morphisms of functors $\eta$ and $\epsilon$ are called the \emph{unit} and \emph{co-unit} of the adjunction between $F$ and $G$.


\section{Many examples}


The main reason that adjunctions between functors are interesting, is that they are ubiquitous: they arise surprisingly often in multiple branches of mathematics. Here is a short list of examples.

\begin{example}[Cartesian product and set of maps]
Fix a set $A$. Then for all sets $X$ and $Y$ we we have a canonical bijection
\[
	\alpha_{X,Y}\colon \Hom(X\times A, Y ) \isomto \Hom(X, \Hom(A,Y))
\]
given  by mapping a function $f\colon X\times A \to Y$ to the function
\[
	X \to \Hom(A,Y),\, x \mapsto \left( a \mapsto f(x,a) \right).
\]
An inverse is given by mapping a function $g\colon X\to \Hom(A,Y)$ to
\[
	X\times A\to Y,\, (x,a) \mapsto g(x)(a).
\]
It is easy to check that $\alpha$ defines an adjunction, making the functor
\[
	\Set \to \Set,\,X \mapsto X \times A
\]
into a left adjoint to the functor
\[
	\Set \to \Set,\, Y \mapsto \Hom(A,Y).
\]
The unit $\eta\colon \id \to \Hom(A,-\times A)$ of this adjunction is given by
\[
	\eta_X \colon X \to \Hom(A,X\times A),\, x \mapsto \left( a \mapsto (x,a) \right)
\]
and the co-unit $\epsilon\colon \Hom(A,-)\times A\to \id$ is given by
\[
	\epsilon_X \colon \Hom(A,X)\times A \to X,\, (f,a) \mapsto f(a).
\]
\end{example}

\begin{example}[Tensor product and Hom]
This is a variation on the previous example. Let $R$ and $S$ be rings, and let  $A$ be an $(R,S)$-bimodule. Then the functor
\[
	\Mod_{R} \to \Mod_S,\, M \mapsto  M \otimes_R A
\]
is left adjoint to the functor
\[
	\Mod_S \to \Mod_R,\, N \mapsto \Hom_S(A,N),
\]
which comes down to the functorial isomorphism
\[
	\alpha_{M,N}\colon \Hom_S( M \otimes_R A,\, N ) \longisomto \Hom_R( M,\, \Hom_S(A,N) )
\]
of Theorem \ref{thm:tensor-hom-adjunction}.
\end{example}


\begin{example}[Free module and forgetful functor]\label{exa:free-forgetful-adjunction}
Let $R$ be a ring. Let $M$ be an $R$-module and let $R^{(I)}$ be the free $R$-module on  a set $I$ (see Example \ref{exa:free-module-functor}). Then we have a canonical map
\[
	\alpha_{I,M} \colon \Hom_{{}_R\Mod}( R^{(I)},  M ) \longto \Hom_{\Set}(  I,  M  ),
\]
given by restricting a module homomorphism $\varphi \colon R^{(I)}\to M$ to the standard basis  $\{e_i\colon i\in I\}$. This map is a bijection, since a module homomorphism $R^{(I)} \to M$ is uniquely determined by the images of the basis vectors $e_i$, and conversely, given a map of sets $f\colon I \to M$ we obtain an $R$-module homomorphism
\[
	R^{(I)} \to M,\, \sum_{i\in I} r_i e_i \mapsto \sum_{i \in I} r_i f(i).
\]
This is just a reformulation of the familiar fact from linear algebra: to give a linear map from $V$ to $W$ is the same as to give the images of the vectors in a basis of $V$.

If we denote by
\[
	G\colon {}_R\Mod \to \Set,\, M \mapsto M
\]
the forgetful functor (see Example \ref{exa:forgetful}) and by
\[
	F\colon \Set \to {}_R\Mod,\,I \mapsto R^{(I)}
\] 
the free module functor, then $\alpha$ defines a bijection
\[
	\alpha_{I,M}\colon \Hom_{{}_R\Mod}( FI, M ) \longisomto \Hom_{\Set}( I,  GM ),
\]
and one can verify directly that this defines an adjunction, making the free module functor $F$ into a left adjoint to the forgetful functor $G$. The unit of this adjunction is the morphism $\eta\colon \id \to GF$ given by the function
\[
	\eta_I \colon I \to R^{(I)},\, i \mapsto e_i,
\]
for every set $I$.
\end{example}

\begin{example}[Discrete topology, forgetful functor, trivial topology]\label{exa:discrete-forgetful-trivial}
Any function \emph{from} a discrete topological space is automatically continuous. Likewise, any function \emph{to} a trivial topological space is automatically continuous. That is, we have 
\[
	\Hom_\Top(X_{\mathrm{disc}}, Y ) = \Hom_\Set( X, Y )
\]
and
\[
	\Hom_\Set( X, Y ) = \Hom_\Top( X, Y_{\mathrm{triv}}),
\]
and we see that the discrete topology functor
\[
	\Set \to \Top,\, X\mapsto X_{\mathrm{disc}}
\]
is left adjoint to the forgetful functor $\Top \to \Set$, and that the trivial topology functor
\[
	\Set \to \Top,\, Y \mapsto Y_{\mathrm{triv}}
\]
is right adjoint to the forgetful functor.
\end{example}


\begin{example}[Frobenius reciprocity]
Let $k$ be a field,  let $G$ be a group and let $H\subset G$ be a subgroup. Then Frobenius reciprocity gives for every $k$-linear representation $V$ of $H$ and $W$ of $G$ a canonical isomorphism
\[
	\Hom_{{k[G]}}( \Ind^G_H V,\, W ) \isomto \Hom_{{k[H]}}( V,\, \Res^G_H W ),
\]
which makes the functor $\Ind^G_H$ into a left adjoint to $\Res^G_H$.
\end{example}


\section{Yoneda and uniqueness of adjoints}
Let $\cC$ be a locally small category.
If $X$ is an object in $\cC$, then we have a functor 
\[
	h_X := \Hom_\cC(-,X)\colon \cC^\opp \to \Set,
\]
see also \ref{exa:contravariant-hom-functor}.
Now the  functors from $\cC^\opp$ to $\Set$ form themselves the objects of a category $\Fun(\cC^\opp,\Set)$, in which the morphisms are the morphisms of functors. We obtain a functor
\[
	h\colon \cC \to \Fun(\cC^\opp,\Set),\, X \mapsto h_X
\]
On the level of morphisms it is given by sending a map $f\colon X \to Y$ to the natural transformation $h_f\colon h_X \to h_Y$ given by
\[
	h_{f,T}\colon \Hom_\cC(T,X) \to \Hom_\cC(T,Y),\, g \mapsto fg
\]
for every $T$ in $\cC$.

\begin{theorem}[Yoneda's Lemma]\label{thm:yoneda}
The functor
\[
	\cC \to \Fun(\cC^\opp,\Set),\, X \mapsto h_X
\]
is fully faithful. 
\end{theorem}

\begin{proof}
In other words, we need to show that for all pairs of objects $X$, $Y$ in $\cC$ the map
\begin{equation}\label{eqn:yoneda-fully-faithfull}
	\Hom_\cC(X,Y) \to \Hom_{\Fun(\cC^\opp,\Set)}(h_X,h_Y), f \mapsto h_f
\end{equation}
is a bijection. 

We show this by constructing an inverse bijection. Let $\varphi\colon h_X\to h_Y$ be a morphism of functors. Then for every $T$ we have a map $\varphi_T\colon h_X(T) \to h_Y(T)$, and in particular, taking $T=X$, we have a map
\[
	\varphi_X \colon \Hom_\cC(X,X) \to \Hom_\cC(X,Y)
\]
and the image of $\id_X$ defines an element $\varphi_X(\id_X)$ in $\Hom_\cC(X,Y)$. We obtain a map
\begin{equation}\label{eqn:yoneda-inverse}
	\Hom_{\Fun(\cC^\opp,\Set)}(h_X,h_Y) \to \Hom_\cC(X,Y),\, \varphi \mapsto \varphi_X(\id_X).
\end{equation}

Using the definition of $h_f$, we see that for a morphism $f\colon X\to Y$ we have
\[
	h_{f,X}(\id_X) = f \circ \id_X = f,
\]
and hence that the composition of (\ref{eqn:yoneda-fully-faithfull}) followed by (\ref{eqn:yoneda-inverse}) is the
identity on $\Hom_\cC(X,Y)$.

To see that the other composition is the identity, let $\varphi\colon h_X\to h_Y$ be a morphism of functors. Under (\ref{eqn:yoneda-inverse})
it is mapped to $\varphi_X(\id_X)$, which in turn under (\ref{eqn:yoneda-fully-faithfull}) is mapped to $h_{\varphi_X(\id_X)}$. To see that the morphisms of functors $\varphi$ and $h_{\varphi_X(\id_X)}$ coincide, it suffice to verify that for all $T$ in $\cC$ 
the maps $h_{\varphi_X(\id_X),T}$ and $\varphi_T$ from $h_X(T)=\Hom(T,X)$ to $h_Y(T)=\Hom(T,Y)$ coincide.

So let $g\in \Hom(T,X)$. We have
\[
	h_{\varphi_X(\id_X),T}(g) = \varphi_X(\id_X) \circ g. 
\]
Since $\varphi$ is a morphism of functors, the square
\[
\begin{tikzcd}
\Hom_\cC(X,X) \arrow{r}{\varphi_X} \arrow{d}{-\circ g} & \Hom_\cC(X,Y) \arrow{d}{-\circ g} \\
\Hom_\cC(T,X) \arrow{r}{\varphi_T} & \Hom_\cC(T,Y)
\end{tikzcd}
\]
commutes. Tracing the element $\id_X$ under the two paths from $\Hom_\cC(X,X)$ to $\Hom_\cC(T,Y)$ we find
\[
	\varphi_X(\id_X) \circ g = \varphi_T(g)
\]
and hence
\[
	h_{\varphi_X(\id_X),T}(g) = \varphi_T(g),
\]
as we had to show.
\end{proof}

\begin{remark}
There is something quite striking in the proof. The inverse bijection $\varphi \mapsto \varphi_X(\id_X)$ proceeds by first 
removing from the morphism of functors $\varphi=(\varphi_T)_T$ all components except for the one at $T=X$, and then restricting the remaining function
$\varphi_X$ to just the element $\id_X$ of $\Hom(X,X)$. Yet, despite this apparent massive loss of information, $\varphi \mapsto \varphi_X(\id_X)$ is a bijection, so that $\varphi$ can be completely recovered from $\varphi_X(\id_X)$.
\end{remark}


\begin{corollary}\label{cor:yoneda-iso}
If $h_X$ and $h_Y$ are isomorphic functors, then $X$ and $Y$ are isomorphic objects in $\cC$. 
\end{corollary}

\begin{proof}
See Exercise \ref{exc:fully-faithful-isomorphism}.
\end{proof}

% TODO: this needs a proof / or reference to exercise on isomorphisms and fully faithful categories

\begin{corollary}[Uniqueness of right adjoints]
If both $G_1\colon \cD \to \cC$ and $G_2\colon \cD \to \cC$ are right adjoints to a functor $F\colon \cC \to \cD$, then $G_1$ and $G_2$ are isomorphic functors.
\end{corollary}

\begin{proof}
Choose adjunctions between $F$ and $G_1$ and between $F$ and $G_2$. Then we obtain isomorphisms
\[
	 \Hom_\cC( X, G_1Y ) \isomfrom \Hom_\cD( FX, Y ) \longisomto \Hom_\cC( X, G_2Y ),
\]
functorial in $X$ and $Y$. Composing these, we find for all $X$ in $\cC$ and $Y$ in $\cD$ an isomorphism
\[
	 \Hom_\cC( X, G_1Y ) \to \Hom_\cC( X, G_2Y ).
\]
Functoriality in $X$ implies that for every $Y$ in $\cD$ we find an isomorphism 
\[
	 \Hom_\cC( -, G_1Y) \longisomto \Hom_\cC( -, G_2Y )
\]
of functors $\cC^\opp \to \Set$, which by Yoneda's Lemma and Exercise \ref{exc:fully-faithful-isomorphism} 
comes from a unique isomorphism
\[
	\gamma_{Y} \colon G_1Y \longisomto G_2Y
\]
in $\cC$. Functoriality in $Y$ implies that the collection $(\gamma_Y)_Y$ defines an isomorphism of functors
\[
	\gamma\colon G_1\longisomto G_2
\]
which finishes the proof.
\end{proof}

There is (of course) a dual of Yoneda's lemma. Given an object $X$ in $\cC$, consider the functor
\[
	h^X := \Hom_\cC(X,-)\colon \cC \to \Set.
\]
We have a functor
\[
	\cC^\opp \to \Fun(\cC,\Set),\, X \mapsto h^X
\]
On the level of morphisms it is given by sending a map $f\colon X \to Y$ to the natural transformation $h^f\colon h^Y \to h^X$ given by
\[
	h^f_T\colon \Hom_\cC(Y,T) \to \Hom_\cC(X,T),\, g \mapsto gf
\]
for every $T$ in $\cC$.

\begin{theorem}[co-Yoneda's Lemma]\label{thm:co-yoneda}
The functor
\[
	\cC^\opp \to \Fun(\cC,\Set),\, X \mapsto h^X
\]
is fully faithful. \qed
\end{theorem}

\begin{corollary}[Uniqueness of left adjoints]
If both $F_1\colon \cC \to \cD$ and $F_2\colon \cC \to \cD$ are left adjoints to a functor $G\colon \cD \to \cC$, then $F_1$ and $F_2$ are isomorphic functors. \qed
\end{corollary}


\newpage
\section*{Exercises}

\begin{exercise}
Verify that the unit $\eta$ and the co-unit $\epsilon$ of an adjunction are indeed morphisms of functors.
\end{exercise}

\begin{exercise}
Let $F\colon \cC \to \cD$ be an equivalence  with quasi-inverse $G\colon \cD \to \cC$. Show that $F$ is both left and right adjoint to $G$.
\end{exercise}

\begin{exercise}\label{exc:abelianization-adjunction}
Show that the abelianization functor $G\mapsto G^\ab$ (see Example \ref{exa:abelianization}) is a left adjoint to the inclusion functor $\Ab \to \Grp$. What are the unit and co-unit of this adjunction?
\end{exercise}

\begin{exercise}\label{exc:ordered-Z-and-R}
Let $\cR$ be the category with $\ob \cR  = \bR$ and 
\[
	\Hom_\cR(x,y) = \begin{cases}
	\{\star\} & x\leq y \\
	\emptyset & x > y
	\end{cases}
\]
for all $x,y\in \bR$ (see also Example \ref{exa:pre-ordered}). Let $\cZ$ be the full subcategory with $\ob \cZ =\bZ$ and let $F\colon \cZ \to \cR$ be the inclusion functor. Does this functor have a left adjoint? And a right adjoint? 
\end{exercise}

\begin{exercise}
Assume $F\colon \cC_1\to \cC_2$ is left adjoint to $G\colon \cC_2\to \cC_1$ and $F'\colon \cC_2\to \cC_3$ is left adjoint to $G'\colon \cC_3\to\cC_2$. Show that $F'F$ is left adjoint to $GG'$. 
\end{exercise}

\begin{exercise}
Let $\{\star\}$ be the `one-point category' consisting of a unique object $\star$ and a unique morphism $\id_\star$. Let $\cC$ be an arbitrary category. When does the (unique) functor $\cC \to \{\star\}$ have a left adjoint? And a right adjoint?
\end{exercise}


\begin{exercise}
For a set $I$ denote by $\bZ[X_i \mid i\in I]$ the polynomial ring in variables $(X_i)$ indexed by $I$. Elements of $\bZ[X_i \mid i\in I]$ are finite $\bZ$-linear combinations of monomials in finitely many of the variables. Verify that $I\mapsto \bZ[X_i \mid i\in I]$ defines a functor $\Set \to \CRing$ which is left adjoint to the forgetful functor $\CRing \to \Set$.
\end{exercise}


\begin{exercise} 
Let $F\colon \cC \to \cD$ be left adjoint to $G\colon \cD \to \cC$. Let $X$ be a cofinal object in $\cC$, show that $FX$ is cofinal in $\cD$. Similarly, if $Y$ is final in $\cD$, show that $GY$ is final in $\cC$. (Compare with Exercise \ref{exc:equivalence-final}).
\end{exercise}

\begin{exercise}\label{exc:forget-base-point}
Show that the forgetful functor $\Top_\star \to \Top$ has a left adjoint but not a right adjoint.
\end{exercise}


\begin{exercise}
Look up the definition of Stone-\v{C}ech compactification, and verify that it gives a left adjoint to the inclusion functor from the category of compact Hausdorff spaces to $\Top$. 
\end{exercise}

\begin{exercise}
Let $G$ be a group and let $R$ be a  ring. Show that restriction defines a bijection between the sets of
\begin{enumerate}
\item ring homomorphisms $f\colon \bZ[G] \to R$ 
\item group homomorphisms $G \to R^\times$.
\end{enumerate}
Interpret this bijection as an adjunction between functors between the categories of groups and rings.
\end{exercise}

\begin{exercise}[Triangle identities ($\star$)]
Let $F\colon \cC \to \cD$ be a left adjoint to $G\colon \cD \to \cC$, with unit $\eta\colon \id_\cC \to GF$ and co-unit $\epsilon\colon FG \to \id_\cD$. Show that the diagrams
\[
\begin{tikzcd}
FX \arrow{r}{F\eta_X} \arrow[swap]{rd}{\id} & FGFX \arrow{d}{\epsilon_{FX}} 
	& & GY \arrow{r}{\eta_{GY}} \arrow[swap]{rd}{\id} & GFGY \arrow{d}{G\epsilon_Y} \\
& FX & & & GY 
\end{tikzcd}
\]
commute for every $X$ in $\cC$ and $Y$ in $\cD$. Conversely, assume that  $F\colon \cC \to \cD$ and $G\colon \cD \to \cC$
are functors, and that  $\eta\colon \id_\cC \to GF$ and $\epsilon\colon FG \to \id_\cD$ are morphisms of functors for which the above triangles commute. Show that $F$ and $G$ form an adjoint pair of functors.
\end{exercise}

\begin{exercise}
A functor $F\colon \cC^\opp \to \Set$ is called \emph{representable} if there exists an object $X$ in $\cC$ with $h_X \cong F$. We say that $F$ is \emph{represented} by $X$. Show that $\cC$ has a final object if and only if the constant functor $\cC^\opp \to \Set,\, T\mapsto \{\star\}$ is representable. 
\end{exercise}

\begin{exercise}
Let $M$ be a right $R$-module and $N$ a left $R$-module. Describe a functor $F\colon \Ab^\opp \to \Set$ which is represented by the abelian group $M\otimes_R N$. Use this to verify that if $R$ is commutative, then $M\otimes_R N \cong N\otimes_R M$.
\end{exercise}


\begin{exercise}
A functor $F\colon \cC\to \Set$ is called \emph{co-representable} if there exists an object $X$ in $\cC$ with $h^X\cong F$. 
Let $f_1,\ldots, f_m \in \bZ[X_1,\ldots,X_n]$. Show that the functor
\[
	\CRing \to \Set,\, R \mapsto \{ x\in R^n \mid f_1(x)=\cdots=f_m(x)=0 \}
\]
of Example \ref{exa:pol-eq} is co-representable.
\end{exercise}

\begin{exercise}[$\star$]
Show that the functor
\[
	\GL_n\colon \CRing \to \Set,\, R \mapsto \GL_n(R)
\]
of Exercise \ref{exc:functor-GLn} is co-representable. (Hint: first show that the functor $\GL_1\colon R \mapsto R^\times$ is isomorphic to $h^{R_1}$ with $R_1=\bZ[X,Y]/(XY-1)$.) Let $R_n$ be the commutative ring such that $\GL_n \cong h^{R_n}$. By the co-Yoneda lemma there is a unique ring homomorphism
$R_1\to R_n$ inducing the natural transformation $\det\colon \GL_n\to \GL_1$. Describe  this ring homomorphism explicitly. 
\end{exercise}

\begin{exercise}[$\star$]
For topological spaces $A$ and $Y$ define $C(A,Y)$ to be the set of continuous maps from $A$ to $Y$.  For every compact $K\subset A$ and open $U\subset Y$ let $C_{K,U}\subset C(A,Y)$ be the subset consisting of those $f\colon A\to Y$ with $f(K)\subset U$. We give $C(A,Y)$ the topology generated by the subsets $C_{K,U}$. (It is known as the compact-open topology). 

Show that if $A$ is compact and Hausdorff, then the functor
\[
	\Top \to \Top \colon X \mapsto X\times A
\]
(giving $X\times A$ the product topology) is left adjoint to the functor
\[
	\Top \to \Top\colon Y \mapsto C(A,Y).
\]
\end{exercise}


%%%%%%%%%%%%%%%%%%%%%%%%%%%%%%%%%%%%%%%%%
% CHAPTER: LIMITS AND COLIMITS 
%%%%%%%%%%%%%%%%%%%%%%%%%%%%%%%%%%%%%%%%%

\chapter{Limits and colimits}\label{chapter:limits-and-colimits}





\section{Product and coproduct}

\begin{definition}
Let $\cC$ be a category, let $X$ and $Y$ be objects in $\cC$.  A \emph{product} of $X$ and $Y$
consists of the data of
\begin{enumerate}
\item an object $P$
\item  morphisms $\pi_X\colon P \to X$ and $\pi_Y\colon P \to Y$
\end{enumerate}
such that for all objects $T$ in $\cC$, and for all $f\colon T\to X$ and $g\colon T\to Y$ there is a unique map $h\colon T \to P$ making the following diagram commute
\[
\begin{tikzcd}
& & X \\
T \arrow[out=40,in=185]{rru}{f} \arrow[out=-40,in=175,swap]{rrd}{g} \arrow[dashed]{r}{h}
	& P \arrow[swap]{ru}{\pi_X} \arrow{rd}{\pi_Y} & \\
& & Y 
\end{tikzcd}
\]
\end{definition}

A product need not exist, but if it does, it is unique up to unique isomorphism. In particular, we will refer to any product of $X$ and $Y$ as \emph{the product} of $X$ and $Y$. We will usually denote it by $X\times Y$, omitting the morphisms $\pi_X$ and $\pi_Y$. The proof of uniqueness is formal, and essentially identical to the proofs of Proposition  \ref{prop:final-object-uniquely-unique} and the uniqueness part of Theorem
\ref{thm:existence-of-tensor-product}. We repeat the argument for the last time.

\begin{proposition}\label{prop:product-uniquely-unique}
Let $(P,\pi_X,\pi_Y)$ and $(P',\pi'_X,\pi'_Y)$ be products of $X$ and $Y$. Then there exists a unique isomorphism $h\colon P \to P'$ such that the diagram
\[
\begin{tikzcd}
& X & \\
P \arrow{ru}{\pi_X} \arrow[swap]{rd}{\pi_Y} \arrow[dashed]{rr}{h}
	 & &  P' \arrow[swap]{lu}{\pi'_X} \arrow{ld}{\pi'_Y} \\
& Y &
\end{tikzcd}
\]
commutes.
\end{proposition}


\begin{proof}[Proof of Proposition \ref{prop:product-uniquely-unique}]
Since $P'$ is a product, there exists a unique $h\colon P \to P'$ making the diagram commute. Likewise, since $P$ is a product, there exists a unique $h'\colon P'\to P$ making the diagram commute. By the same reasoning, there are unique maps $i$ and $i'$ making the diagrams
\[
\begin{tikzcd}
& X & & & X & \\
P \arrow{ru}{\pi_X} \arrow[swap]{rd}{\pi_Y} \arrow[dashed]{rr}{i}
	 & &  P \arrow[swap]{lu}{\pi_X} \arrow{ld}{\pi_Y} 
& P' \arrow{ru}{\pi'_X} \arrow[swap]{rd}{\pi'_Y} \arrow[dashed]{rr}{i'}
	 & &  P' \arrow[swap]{lu}{\pi_X'} \arrow{ld}{\pi_Y'} \\
& Y & & & Y &
\end{tikzcd}
\]
commute. Combining both parts, we see that both $i=\id_P$ and $i=h'h$ make the first diagram commute, and that both $i'=\id_{P'}$ and $i'=hh'$ make the second diagram commute. By unicity we have $h'h=\id_P$ and $hh'=\id_{P'}$, hence $h$ is an isomorphism.
\end{proof}


\begin{examples}
In $\Set$ the product is the cartesian product, together with the usual projections. In $\Top$ the product is the cartesian product equipped the product topology, together with the usual projections. The product of two rings  $R$ and $S$ in $\Ring$ is the product ring $R\times S$.  The product of two modules in ${}_R\Mod$ is the cartesian product  $M \times N$.
\end{examples}



The dual notion of a product is a coproduct  (sometimes called sum).

\begin{definition}
Let $\cC$ be a category, let $X$ and $Y$ be objects in $\cC$.  A \emph{coproduct} of $X$ and $Y$
consists of the data of
\begin{enumerate}
\item an object $S$
\item  morphisms $\iota_X\colon X \to S$ and $\iota_Y\colon Y \to S$
\end{enumerate}
such that for all objects $T$ in $\cC$, and for all $f\colon X\to T$ and $g\colon Y\to T$ there is a unique map $h\colon S \to T$ making the diagram 
\[
\begin{tikzcd}
X \arrow[swap]{rd}{\iota_X} \arrow[out=-5,in=140]{rrd}{f} & & \\
& S \arrow[dashed]{r}{h} & T \\
Y \arrow{ru}{\iota_Y} \arrow[out=5,in=-140]{rru}{g} & & 
\end{tikzcd}
\]
commute.
\end{definition}

By the same argument as before, a coproduct, if it exists, is unique up to isomorphism. We talk about \emph{the} coproduct, and denote it with $X\amalg Y$.

\begin{example}Let $X$ and $Y$ be sets. Then the coproduct of $X$ and $Y$ in $\Set$ is the disjoint union $X\amalg Y$, together with the canonical inclusions $\iota_X \colon X\to X \amalg Y$ and $\iota_Y\colon Y \to X \amalg Y$. 

Similarly, the coproduct in $\Top$ of topological spaces $X$ and $Y$ is the disjoint union $X \amalg Y$, with the natural topology (in which $X$ and $Y$ are both open and closed).
\end{example}

\begin{example}
Let $R$ be a ring. Then the coproduct of $R$-modules $M$ and $N$ is the direct sum $M\oplus N$, and hence in this case the  product and the coproduct coincide (although the former is considered together with the projections, and the latter with the inclusions). 
\end{example}

More generally one can define products and coproducts of arbitrary (possibly infinite) families of objects.

\begin{definition}
The \emph{product} of a family $(X_i)_{i\in I}$  of objects in $\cC$ is an object $\prod_{i\in I} X_i$ together with maps $\pi_n \colon \prod_{i\in I} X_i \to X_n$ such that for every $T$ and for every collection of morphisms $f_i \colon T\to X_i$ there is a unique $h\colon T\to \prod_{i\in I} X_i$ such that $f_i = \pi_ih$ for all $i\in I$.
\end{definition}

\begin{definition}
The \emph{coproduct} of a family $(X_i)_{i\in I}$  of objects in $\cC$ is an object $\coprod_{i\in I} X_i$ together with maps $\iota_n \colon X_n\to \coprod_{i\in I} X_i$ such that for every $T$ and for every collection of morphisms $f_i \colon X_i\to T$ there is a unique $h\colon \coprod_{i\in I} X_i \to T$ such that $f_i = h\iota_i$ for all $i\in I$.
\end{definition}

Again, products and coproducts of arbitrary families need not exist, but if they do they are unique up to unique isomorphism.

\begin{example}[Product and coproduct of modules]
Let $R$ be a ring and $(M_i)_{i\in I}$ a collection of $R$-modules. Then the product of $(M_i)$ is the $R$-module
\[
	\prod_{i\in I} M_i = \{ (x_i)_{i\in I} \mid x_i \in M_i \}
\]
together with the projection maps $\pi_n\colon \prod_{i\in I} M_i \to M_n$. 
The universal property of products gives a natural (\emph{i.e.} functorial) bijection
\[
	\Hom_R( N,  \prod_{i\in I} M_i ) \longisomto \prod_{i\in I} \Hom_R(N, M_i ).
\]
The coproduct of $(M_i)_i$ is the direct sum $\bigoplus_{i\in I} M_i$ 
together with the inclusion maps $\iota_n\colon M_n \to \bigoplus_{i\in I} M_i $ given by
\[
	\iota_n(x)_i = \begin{cases} x & i=n \\ 0 & i \neq n \end{cases}
\]
for all $x\in M_n$ and $i\in I$. Indeed, the natural  bijection
\[
	\Hom_R( \bigoplus_{i\in I} M_i, N ) \longisomto \prod_{i\in I} \Hom_R(M_i, N )
\]
of Exercise \ref{exc:universal-property-direct-sum} shows that for every collection of morphisms $(f_i\colon M_i 
\to N)_i$ there is a unique morphism $h\colon  \bigoplus_{i\in I} M_i \to N$ with $h \iota_i = f_i$.
\end{example}



\section{Pullback and pushout}

\begin{definition}
Let $\cC$ be a category and let  $f\colon X\to Z$ and $g\colon Y\to Z$ be morphisms in $\cC$. The \emph{pullback} or \emph{fibered product} of $f$ and $g$ consists of 
\begin{enumerate}
\item an object $P$ in $\cC$
\item morphisms $\pi_X\colon P \to X$ and $\pi_Y\colon P \to Y$
\end{enumerate}
such that $f\pi_X=g\pi_Y$ as maps $P\to Z$, and such that for every object $T$ and for every pair of maps $s\colon T\to X$, $t\colon T\to Y$ with $fs=gt$
there is a unique morphism $h\colon T\to P$ making the diagram
\[
\begin{tikzcd}
& & X \arrow[swap]{rd}{f} \\
T \arrow[out=40,in=185]{rru}{s} \arrow[out=-40,in=175,swap]{rrd}{t} \arrow[dashed]{r}{h}
	& P \arrow{ru}{\pi_X} \arrow[swap]{rd}{\pi_Y} & & Z  \\
& & Y \arrow{ru}{g}
\end{tikzcd}
\]
commute.
\end{definition}

 If it exists, the pullback is unique up to unique isomorphism. It is usually denoted by $P=X\times_Z Y$, but care should be taken since the pullback does depend on the maps $f$ and $g$. Alternatively, one  says that the square
 \[
 \begin{tikzcd}
 	P \arrow{r}{\pi_Y} \arrow[swap]{d}{\pi_X} & Y \arrow[swap]{d}{g} \\
	X \arrow{r}{f} & Z
\end{tikzcd}
\]
is \emph{cartesian} if $(P,\pi_X,\pi_Y)$ is the fibered product of $f$ and $g$.
 
\begin{example}
In the category of sets, the fiber product of any pair of maps $f\colon X\to Z$ and $g\colon Y\to Z$ exists. It is given by
\[
	X\times_Z Y = \{ (x,y) \in X\times Y \mid f(x) = g(y) \},
\]
together with the projection maps. Indeed, if $s\colon T\to X$ and $t\colon T\to Y$ satisfy $fs=gt$, then the  map
\[
	h\colon T \to X\times_Z Y,\, u \mapsto (s(u),t(u)).
\]
is the unique map making the diagram of the definition commute. 

In the  case that $X=\{\star\}$ and $f$ the map $\star \mapsto z$, then we find
\[
	\{\star\} \times_Z Y = g^{-1}(z),
\]
the fiber of $Y$ over $z$.

In the case that $X, Y$ are subsets of  $Z$ and $f$ and $g$ are the respective inclusions we find
$X\times_Z Y = X \cap Y$.
\end{example}

\begin{example}
Similarly, in $\Top$ we have
\[
	X\times_Z Y = \{ (x,y) \in X\times Y \mid f(x) = g(y) \}
\]
with the induced topology from the product topology on $X\times Y$.
\end{example}

\begin{example}
Similarly, if $f_1\colon M_1 \to N$ and $f_2\colon M_2 \to N$ are $R$-module homomorphisms, then
\[
	\{ (x_1,x_2) \in M_1\times M_2 \mid f_1(x_1)=f_2(x_2) \}
\]
is a sub-$R$-module of $M_1\times M_2$, and one verifies that it is the pullback of $f_1$ and $f_2$.
\end{example}

\begin{example}In particular, the kernel of a morphism $f\colon M\to N$ is the pull-back of $f$ and the map $0\to N$. In other words, the square
\[
	\begin{tikzcd}
		\ker f \arrow{r} \arrow{d} & M \arrow{d}{f} \\
		0 \arrow{r} & N
	\end{tikzcd}
\]
is cartesian in ${}_R\Mod$.
\end{example}

The dual notion of pullback/fibered product is pushout/fibered coproduct. The definition is obtained by reversing all the arrows.

\begin{definition}
Let $\cC$ be a category and let  $f\colon Z\to X$ and $g\colon Z\to Y$ be morphisms in $\cC$. The \emph{pushout} or \emph{fiber coproduct} or \emph{fiber sum} of $f$ and $g$ consists of 
\begin{enumerate}
\item an object $S$ in $\cC$
\item morphisms $\iota_X\colon X \to S$ and $\iota_Y\colon Y \to S$
\end{enumerate}
such that $\iota_Xf=\iota_Yg$ as maps $Z\to S$, and such that for every object $T$ and for every pair of maps $s\colon X\to T$, $t\colon Y\to T$ with $sf=tg$
there is a unique morphism $h\colon S\to T$ making the diagram
\[
\begin{tikzcd}
&X \arrow[swap]{rd}{\iota_X} \arrow[out=-5,in=140]{rrd}{s} & & \\
Z \arrow{ru}{f} \arrow[swap]{rd}{g} && S \arrow[dashed]{r}{h} & T \\
 & Y \arrow{ru}{\iota_Y} \arrow[out=5,in=-140]{rru}{t} & & 
\end{tikzcd}
\]
commute.
\end{definition}

If it exists, the pushout is unique up to unique isomorphism. 
 
\begin{example}Pushouts exist in $\Set$, and can be constructed as follows. Let $f\colon Z\to X$ and $g\colon Z\to Y$ be functions. Consider the equivalence relation $\sim$ on the disjoint union $X\amalg Y$ generated by $f(z) \sim g(z)$. Let $S$ be the quotient $(X\amalg Y)/\sim$. Then we claim that $S$ is the pushout of $f$ and $g$. Indeed, assume that we have a commutative diagram
\[
\begin{tikzcd}
&X \arrow[swap]{rd}{\iota_X} \arrow[out=-1,in=140]{rrd}{s} & & \\
Z \arrow{ru}{f} \arrow[swap]{rd}{g} && S \arrow[dashed]{r}{h} & T \\
 & Y \arrow{ru}{\iota_Y} \arrow[out=1,in=-140]{rru}{t} & & 
\end{tikzcd}
\]
Then there is only one possibility for the map $h$:
\[
	h\colon S \to T,\,
	\begin{cases} [x] \mapsto s(x) & x\in X \\ [y] \mapsto t(y) & y\in Y
	\end{cases}
\]
This is indeed well-defined, since for any $z\in Z$ we have $s(f(z))=t(g(z))$.
\end{example}

\begin{example}[gluing]
The same construction as in $\Set$ defines a pushout in $\Top$, it suffices to put the quotient topology on $X\amalg Y/\sim$.  This construction is particularly useful in topology when both $f\colon Z\to X$ and $g\colon Z\to Y$ are injective. In this case it constructs a space $S$ by `gluing' $X$ and $Y$ along their common subspace $Z$. 

Conversely, if $S$ is a topological space and $X$ and $Y$ subspaces with $S = X \cup Y$, then $S$ is the pushout of the inclusion maps $X\cap Y \to X$ and $X\cap Y \to Y$.
\end{example}

\begin{example}The cokernel of a morphism $f\colon M\to N$ is the pushout of $f$ and the map $M\to 0$. See also Exercise~\ref{exc:fibered-coproduct-in-Ab}.
\end{example}

%
%\section{Equalizer and co-equalizer}
%
%\begin{definition} Let $\cC$ be a category, let $X$ and $Y$ be objects in $\cC$ and let $f\colon X\to Y$ and $g\colon X\to Y$ be morphisms in $\cC$. The \emph{equalizer} of $f$ and $g$ is a pair $(E,\iota)$ consisting of 
%\begin{enumerate}
%\item an object $E$ in $\cC$,
%\item a morphism $\iota\colon E\to X$ in $\cC$,
%\end{enumerate}
%such that $f\iota=g\iota$, and such that for every morphism $t\colon T\to X$ with $ft=gt$ there is a unique $h\colon T\to E$ making the diagram
%\[
%\begin{tikzcd}
%	T \arrow[dashed]{d}{h} \arrow{dr}{t} \\
%	E \arrow{r}{\iota} & X \arrow[out=8,in=172]{r}{f} \arrow[out=-8,in=-172,swap]{r}{g} & Y
%\end{tikzcd}
%\]
%commute.
%\end{definition}
%
%The equalizer of $f$ and $g$, if it exists, is unique up to unique isomorphism.
%
%\begin{example}In $\Set$ the equalizer of $f,g\colon X\to Y$ is given by $E=\{x \in X \mid f(x)=g(x)\}$ and $\iota\colon E\to X$ the inclusion. Also in $\Ring$, $\CRing$, $\Top$, $\Grp$, ${}_R\Mod$ equalizers exist, and are given by the same recipe. For example, the equalizer of a pair of ring homomorhpisms $f,g\colon R\to S$ is the subring
%\[
%	\{ r\in R \mid f(r) = g(r) \} \subset R,
%\]
%together with its inclusion in $R$.
%\end{example}
%
%\begin{example}In ${}_R\Mod$ the equalizer of $f,g\colon M\to N$ is nothing but $\ker (f-g)$. In particular, taking $g=0$ we see that a kernel of an $R$-module homomorphism is a special case of an equalizer.
%\end{example}
%
%The dual notion of equalizer is co-equalizer.
%
%
%\begin{definition} Let $\cC$ be a category, let $X$ and $Y$ be objects in $\cC$ and let $f\colon X\to Y$ and $g\colon X\to Y$ be morphisms in $\cC$. The \emph{co-equalizer} of $f$ and $g$ is a pair $(C,\pi)$ consisting of 
%\begin{enumerate}
%\item an object $C$ in $\cC$,
%\item a morphism $\pi\colon Y\to C$ in $\cC$,
%\end{enumerate}
%such that $\pi f=\pi g$, and such that for every morphism $t\colon Y\to T$ with $tf=tg$ there is a unique $h\colon C\to T$ making the diagram
%\[
%\begin{tikzcd}
%	 X \arrow[out=8,in=172]{r}{f} \arrow[out=-8,in=-172,swap]{r}{g} 
%	 	& Y \arrow{r}{\pi} \arrow[swap]{rd}{t} & C \arrow[dashed]{d}{h}\\
%	 & & T 
%\end{tikzcd}
%\]
%commute.
%\end{definition}
%
%The co-equalizer of $f$ and $g$, if it exists, is unique up to unique isomorphism.
%
%\begin{example}In $\Set$, the co-equalizer of $f,g\colon X\to Y$ is the quotient $C=Y/\!\sim$ of $Y$ by the equivalence relation generated by $f(x)\sim g(x)$ for all $x\in X$. 
%\end{example}
%
%\begin{example}In ${}_R\Mod$ the co-equalizer of $f,g\colon M\to N$ is $\coker(f-g)$. In particular, taking $g=0$ we see that cokernels of $R$-linear maps are examples of co-equalizers.
%\end{example}
%
\section{Limits and colimits}

The above constructions are examples of two dual general classes of categorical constructions called limits and colimits.

Let $\cI$ be a small category (see Definition \ref{def:small-category}), let $\cC$ be a category, and let $X\colon \cI \to \cC$ be a functor. We will often denote the image of an object $i \in \cI$ by $X_i$.

It is useful to think of the functor $X\colon \cI\to \cC$ informally as a diagram in $\cC$, indexed by $\cI$. For example, if $\cI$ is the three-object category
\[
\begin{tikzcd}[row sep=small]
1 \arrow{rd}{\varphi}  \\ & 2 \\ 3 \arrow[swap]{ru}{\psi}
\end{tikzcd}
\]
then a functor $X\colon \cI\to \cC$ can be thought of as  a diagram
\[
\begin{tikzcd}[row sep=small]
X_1 \arrow{rd}{\!X(\varphi)}  \\ & X_2 \\ X_3 \arrow[swap]{ru}{\!X(\psi)}
\end{tikzcd}
\]
of objects and morphisms in $\cC$. See also Example \ref{exa:diagrams-as-functors}.

\begin{definition}The \emph{limit} of a functor $X\colon \cI \to \cC$  consists of
\begin{enumerate}
\item an object $\lim_\cI X$ in $\cC$
\item for every object $i$ in $\cI$ a morphism $\pi_i \colon \lim_\cI X \to X_i$
\end{enumerate}
such that
\begin{enumerate}
\item for every $\varphi\colon i \to j$ in $\cI$ we have $\pi_j = X(\varphi) \circ \pi_i$
\item for every $T$ in $\cC$ and for every collection of morphisms $t_i\colon T\to X_i$ satisfying $t_j = X(\varphi) \circ t_i$ for all $\varphi\colon i\to j$, there is a unique $h\colon T\to \lim_\cI X$ with $t_i = \pi_i h$ for all $i$.
\end{enumerate}
\end{definition}

Of course, the limit of $\cI \to \cC$, if it exists, is unique up to unique isomorphism. The limit is sometimes written $\lim_i X_i$ or $\lim_\cI X_i$, but one should be careful to remember that it depends on the full functor $X$, and not just on the objects $(X_i)_{i \in \ob \cI}$.


\begin{examples}\label{exa:three-limits}
If $\cI$ the discrete category on a set $I$ (see \ref{exa:discrete}), then $\lim_\cI X$ is the product $\prod_{i\in I} X_i$ (if it exists). If $\cI$ is empty, then the limit of the unique functor $\emptyset\to \cC$ is the final object of $\cC$ (if it exists). Taking for $\cI$ the category
\[
\begin{tikzcd}[row sep=small]
	1 \arrow{rd} & \\  & 2, \\ 3 \arrow{ru} & 
\end{tikzcd}
\]
recovers the notion of  fibered product.  
\end{examples}


There is no surprise in the definition of colimit: 

\begin{definition}The \emph{colimit} of a functor $X\colon \cI \to \cC$  consists of
\begin{enumerate}
\item an object $\colim_\cI X$ in $\cC$,
\item for every object $i$ in $\cI$ a morphism $\iota_i \colon X_i \to \colim_\cI X$
\end{enumerate}
such that
\begin{enumerate}
\item for every $\varphi\colon i \to j$ in $\cI$ we have $\iota_j \circ X(\varphi) = \iota_i$,
\item for every $T$ in $\cC$ and for every collection of morphisms $t_i\colon X_i\to T$ satisfying $t_j \circ X(\varphi) = t_i$ for all $\varphi\colon i\to j$, there is a unique $h\colon \colim_\cI X\to T$ with $t_i = h \iota_i$ for all $i$.
\end{enumerate}
\end{definition}

If it exists, it is unique up to unique isomorphism.

\begin{examples}
If $\cI$ the discrete category on a set $I$, then $\colim_\cI X$ is the coproduct $\coprod_{i\in I} X_i$ (if it exists). If $\cI$ is empty, then the colimit of the unique functor $\emptyset\to \cC$ is the cofinal object of $\cC$ (if it exists). Taking for $\cI$ the category
\[
\begin{tikzcd}[row sep=small]
	& 1  \\  2 \arrow{ru} \arrow{rd}  & \\ & 3  
\end{tikzcd}
\]
recovers the notion of pushout. 
\end{examples}

We end with an example of an infinite diagram in $\Set$ and its limit and colimit.

\begin{example}\label{exa:descending-chain-of-inclusions}
Let $\cI$ be the category with $\ob \cI = \bN$ and such that for all $i<j$ we have $\Hom(i,j) = \emptyset$, and for all $i\geq j$ we have $\Hom(i,j)=\{\star\}$. In a picture:
\[
	0 \longfrom 1 \longfrom 2 \longfrom 3 \longfrom \cdots.
\]
Let $S_0 \supset S_1 \supset S_2 \supset \cdots $ be a decreasing chain of sets. Then this defines a functor
\[
	S \colon \cI \to \Set,\, i \mapsto S_i
\]
which for $i\geq j$ maps the unique map $i\to j$ to the inclusion $S_i \injto S_j$.

To give a collection of maps $t_i\colon T\to S_i$ such that for all $i\geq j$ the diagram
\[
\begin{tikzcd}
	T \arrow{d}{t_j} \arrow{dr}{t_i} \\
	S_j  & S_i \arrow{l}
\end{tikzcd}
\]
commutes, is the same as to give a map $t\colon T \to \bigcap_i S_i$. From this it follows easily that $\lim_i S_i = \bigcap_i S_i$.

For the colimit, note that a collection of maps $t_i\colon S_i \to T$ such that for all $i\geq j$ the diagram
\[
\begin{tikzcd}
	S_j  \arrow[swap]{dr}{t_j}  & S_i \arrow{d}{t_i} \arrow{l} \\
	& T
\end{tikzcd}
\]
commutes is completely determined by $t_0\colon S_0\to T$ (since $t_i$ must be the restriction of $t_0$ to the subset $S_i\subset S_0$), and one easily verifies that $\colim_i S_i = S_0$.
\end{example}

\section{Yoneda and limits and colimits of sets}

The category $\Set$ has all limits and colimits.

\begin{proposition}\label{prop:limit-in-Set}
Let $\cI$ be a small category and let $X\colon \cI \to \Set,\, i \mapsto X_i$ be a functor. Then
$\lim_\cI X_i$ exists, and is given by
\[
	\big\{ (x_i)_i \in \! \prod_{i\in \ob\cI} \! X_i \mid
	\text{$X(\varphi)(x_i) = x_j$ for all $\varphi\colon i \to j$} \big\}
\]
together with the projection maps to the sets $X_i$.
\end{proposition}

\begin{proof}The proof is a straightforward verification that the set $L:=\{ (x_i)_i \in \! \prod_{i} \! X_i \mid \cdots \}$  described in the proposition, with the projection maps
\[
	\pi_n\colon L\to X_n,\, (x_i)_i \mapsto x_n
\]
satisfies the definition of a limit. The first property holds by construction: for every $x \in L$ and $\varphi\colon i \to j$ in $\cI$ we have
\[
	\pi_j (x) = x_j = X_\varphi (x_i) = ( X_\varphi \circ \pi_i )(x),
\]
hence $\pi_j = X_\varphi \circ \pi_i$. For the second property, assume that $(T,(t_i\colon T\to X_i)_i)$ satisfies 
$t_j = X_\varphi \circ t_i$ for every $\varphi\colon i\to j$. Then the map 
\[
	h \colon T \to L,\, x \mapsto  (t_i(x))_i 
\]
is well-defined (that is, $h(x)$ lands in $L \subset \prod_i X_i$), and clearly is the unique map such that $t_i = \pi_i h$ for every $i$.
\end{proof}



\begin{proposition}\label{prop:colimit-in-Set}
Let $\cI$ be a small category and let $X\colon \cI \to \Set,\, i \mapsto X_i$ be a functor. Consider on the disjoint union 
$\coprod_{i\in \ob\cI}  X_i$
the equivalence relation $\sim$ generated by $x_i \sim X(\varphi)(x_i)$ for all $\varphi\colon i \to j$ and all $x_i \in X_i$. Then 
$\colim_{\cI} X_i$ exists and is given by
\[
	\colim_{\cI} X = \Big( \!\coprod_{i\in \ob\cI}\!\!  X_i\, \Big) \big/ \! \sim
\]
together with the compositions
\[
	\iota_i \colon X_I \longinjto \!\coprod_{i\in \ob\cI} \!\! X_i \longsurjto \colim_{\cI} X.
\]
\end{proposition}

\begin{proof}
This is shown by a direct verification, somewhat similar to the proof of Proposition \ref{prop:limit-in-Set}. See also Exercise \ref{exc:colimit-in-Set}.
\end{proof}



Using the explicit descriptions of limits  of sets, we can now rephrase the universal property for limits and colimits in an arbitrary category:

\begin{theorem}\label{thm:limit-yoneda}
Let $\cI$ be a small category and let $X\colon \cI \to \cC$ be a functor. Let $L$ be an object of $\cC$. Then $\lim_\cI X$ exists and is isomorphic to $L$ if and only if there is an isomorphism
\[
	\Hom_\cC(-, L) \isomto \textstyle\lim_\cI \Hom_\cC(-,X_i),
\]
of functors $\cC^\opp \to \Set$. \end{theorem}

Note that by Yoneda (Theorem \ref{thm:yoneda})  the above theorem completely characterizes $\lim_\cI X$. 

\begin{proof}[Proof of Theorem \ref{thm:limit-yoneda}]
Let $T$ be an object in $\cC$. By the explicit description of limit of sets in Proposition \ref{prop:limit-in-Set}
we have
\[
	\lim_{i\in \cI} \Hom_\cC(T,X_i) = \big\{ (t_i\colon T\to X_i)_i \mid \forall \varphi \colon i \to j,\,X(\varphi) \circ t_i = t_j \big\}.
\]
By the universal property of the limit in $\cC$ the map 
\[
	\Hom_\cC(T, \textstyle\lim_i X_i) \longisomto
	\big\{ (t_i\colon T\to X_i)_i \mid \forall \varphi \colon i \to j,\,X(\varphi) \circ t_i = t_j \big\}
\]
given by $h \mapsto ( \pi_i \circ h)_{i \in \cI}$ is a bijection. The bijection is functorial in $T$, and hence defines an isomorphism of functors.
Conversely, if there is a functorial bijection
\[
	\Hom_\cC(T,L) \longisomto
	\big\{ (t_i\colon T\to X_i)_i \mid \forall \varphi \colon i \to j,\, X(\varphi) \circ t_i = t_j \big\},
\]
then $L$ satisfies the universal property of the limit.
\end{proof}

\begin{theorem}\label{thm:colimit-yoneda}
Let $\cI$ be a small category and let $X\colon \cI \to \cC$ be a functor. Let $C$ be an object of $\cC$. 
Then $\colim_\cI X$ exists and is isomorphic to $C$ if and only if there is an isomorphism
\[
	\Hom_\cC(C, -) = \textstyle\lim_{\cI^\opp} \Hom_\cC(X_i, -)
\]
of functors $\cC\to \Set$. \qed
\end{theorem}
Note that by co-Yoneda (Theorem \ref{thm:co-yoneda})  the above theorem completely characterizes $\colim_\cI X$. 


\section{Adjoint functors and limits}



\begin{theorem}[Left adjoints commute with colimits]\label{thm:left-adjoints-commute-with-colimits}
Let $F\colon \cC\to \cD$ be a left adjoint to $G\colon \cD\to \cC$. Let $X\colon \cI \to \cC$ be a functor, and suppose that $\colim_\cI X$ exists in $\cC$. Then $\colim_\cI (FX)$ exists and
\[
	\colim_\cI (FX) \cong F(\colim_\cI X)
\]
in $\cD$.
\end{theorem}

Informally, we say that `$F$ commutes with colimits'.

\begin{proof}[Proof of Theorem \ref{thm:left-adjoints-commute-with-colimits}]
Using Theorem \ref{thm:colimit-yoneda} and the definition of adjoint functors we find for every $T$ in $\cD$ a chain of isomorphisms
\begin{align*}
	\Hom_\cD(F(\colim_\cI X_i), T)
		&\cong \Hom_\cC( \colim_\cI X_i , GT ) \\
		&\cong \textstyle\lim_{\cI^\opp} \Hom_\cC( X_i, GT) \\
		&\cong \textstyle\lim_{\cI^\opp} \Hom_\cD( FX_i, T),
\end{align*}
functorial in $T$,  and hence an isomorphism of functors
\[
	\Hom_\cD(F(\colim_\cI X_i), - ) \cong 
	 \textstyle\lim_{\cI^\opp} \Hom_\cD( FX_i, -)
\]
which by Theorem \ref{thm:colimit-yoneda} shows that $F(\colim_\cI X_i)$ is the colimit of the
diagram $\cI \to \cD,\, i \mapsto FX_i$.
\end{proof}

The co-theorem states that right adjoints commute with limits:

\begin{theorem}[Right adjoints commute with limits]\label{thm:right-adjoints-commute-with-limits}
Let $F\colon \cC\to \cD$ be a left adjoint to $G\colon \cD\to \cC$. Let $X\colon \cI \to \cD$ be a functor, and suppose that $\lim_\cI X$ exists in $\cD$. Then $\lim_\cI (GX)$ exists and
\[
	\textstyle\lim_\cI (GX) \cong G(\textstyle\lim_\cI X)
\]
in $\cC$.\qed
\end{theorem}




\begin{example}
Let $R$ be a ring and let $A$ be an $(S,R)$-bimodule.
By Theorem \ref{thm:tensor-hom-adjunction} the functor
\[
	{}_R\Mod \to {}_S\Mod,\, M \mapsto A \otimes_R M
\]
is left adjoint (to the functor $\Hom_S(A,-)$), and hence by Theorem \ref{thm:left-adjoints-commute-with-colimits} it commutes with 
colimits. In particular, it commutes with coproducts:
\[
	A\otimes_R (\oplus_{i\in I} M_i) = \oplus_{i\in I} ( A\otimes_R M_i )
\]
and with cokernels:
\[
	\coker( \id\otimes f\colon A\otimes_R M \to A\otimes_R N ) =
	A\otimes_R \coker(f\colon M \to N).
\]
(Note that the latter gives a one-line proof of Proposition \ref{prop:tensor-right-exact}!). 

There is a priori no reason for the functor to commute with limits, and indeed in general $A\otimes_R -$ does not respect kernels (see Exercise \ref{exc:tensor-not-exact}) or (infinite) products (see Exercise \ref{exc:tensor-does-not-preserve-products}).
\end{example}

\begin{example}The forgetful functor ${}_R\Mod \to \Set$ is right adjoint to the free module functor 
(see Example \ref{exa:free-forgetful-adjunction}) and hence by Theorem \ref{thm:right-adjoints-commute-with-limits} it commutes with limits. For example, this implies that the underlying set of a product of modules is the product of the underlying sets of the modules. 

On the other side of the adjunction, we see that the free module functor $I \mapsto R^{(I)}$ must commute with colimits. For example, this contains the (trivial) statement that a basis for the direct sum of free modules is given by the disjoint union of their bases, that is
\[
	R^{(I)} \oplus R^{(J)} \cong R^{(I\amalg J)}
\]
as $R$-modules.
\end{example}

\begin{example}We have seen in Example \ref{exa:discrete-forgetful-trivial} that the forgetful functor $\Top\to \Set$ is both right adjoint (to the discrete topology functor) and left adjoint (to the trivial topology functor). It therefore commutes with both limits and colimits, and hence any limit or colimit of topological spaces can be constructed by putting a suitable topology on the limit or colimit of the underlying sets.
\end{example}




\newpage
\section*{Exercises}


\begin{exercise}
Let $X$ and $Y$ be topological spaces. Show that the cartesian product $X\times Y$ with the product topology is the product of $X$ and $Y$ in the category $\Top$.
\end{exercise}

\begin{exercise}\label{exc:fibered-coproduct-in-Ab}
Let $f_0\colon M\to N_0$ and $f_1\colon M\to N_1$ be morphisms in ${}_R\Mod$. Show that their fibered coproduct exists. (Hint: construct the fibered coproduct as a quotient module of $N_0\oplus N_1$).
\end{exercise}


\begin{exercise}Does the category of pointed topological spaces $\Top_\ast$ have products and/or coproducts? And if so, what are they? Does the forgetful functor $\Top_\ast \to \Top$ commute with products and/or coproducts? 
\end{exercise}

\begin{exercise}\label{exc:product-as-final-object}
Let $\cC$ be a category and let $X$ and $Y$ be objects in $\cC$. Find a category $\cP$ in which the products of $X$ and $Y$ are precisely the final objects. Conclude that Proposition \ref{prop:product-uniquely-unique} can be deduced directly from Proposition  \ref{prop:final-object-uniquely-unique}.
\end{exercise}


\begin{exercise}
Show that the pushout of the inclusion map $\{0,1\} \to [0,1]$ and the map $\{0,1\} \to \{\star\}$ in $\Top$ is the circle.
\end{exercise}

\begin{exercise}
Let $\cC$ be a category. Consider the diagonal functor 
\[
	\Delta\colon \cC \to \cC \times \cC
\]
defined by $X\mapsto (X,X)$ and $f\mapsto (f,f)$. When does $\Delta$ have a left adjoint? And a right adjoint?
\end{exercise}


\begin{exercise}[Coproduct of commutative rings]
Let $R$ and $S$ be commutative rings.
\begin{enumerate}
\item Show that there is a unique ring structure on the $\bZ$-module $R\otimes_\bZ S$ such that
\[
	(r_1 \otimes s_1)(r_2\otimes s_2) = r_1 r_2 \otimes s_1s_2.
\]
\item Show that $R\to R\otimes_\bZ S,\, r \mapsto r \otimes 1$ and
$S\to R\otimes_\bZ S,\, s \mapsto 1 \otimes s$ are ring homomorphisms.
\item Show that $R\otimes_\bZ S$ is the coproduct of $R$ and $S$ in $\CRing$. 
\end{enumerate}
\end{exercise}


\begin{exercise}\label{exc:module-pushout}
Let $R$ be a ring and let
\[
\begin{tikzcd}
	M \arrow{r}{f} \arrow{d}{g} & P \arrow{d}{\varphi} \\
	Q \arrow{r}{\psi} & N 
\end{tikzcd}
\]
be a commutative square of $R$-modules. 
\begin{enumerate}
\item Show that $M$ (with the maps $f$ and $g$) is the pullback of $P\to N$ and $Q\to N$ if and only if the sequence
\[
	0 \longto M \overset{(f,g)}{\longto} P \oplus Q \overset{\varphi-\psi}{\longto} N
	\phantom{\longto 0}
\]
is exact.
\item Show that $N$ (with the maps $\varphi$ and $\psi$) is the pushout of $M\to P$ and $M\to Q$ if and only if the sequence
\[
	\phantom{0\longto} M \overset{(f,g)}{\longto} P \oplus Q \overset{\varphi-\psi}{\longto} N \longto 0
\]
is exact.
\end{enumerate}

\end{exercise}


%\begin{exercise}
%Let $\cC$ be a category which has both products and fibered products. Show that $\cC$ has equalizers. (Hint: given $f,g\colon X\to Y$ consider their fibered product $Z := X\times_Y X$. Show that there are maps $Z\to X\times X$ and $X\to X\times X$ whose fibered product is an equalizer of $f$ and $g$.)
%\end{exercise}

\begin{exercise}\label{exc:tensor-does-not-preserve-products}
Show that the element
\[
	1 \otimes (1, 1, \ldots )	
\]
of the module  $\bQ\otimes_\bZ \left( \prod_{n>0} \bZ/n\bZ \right)$
is non-zero. Conclude that the functor $\bQ\otimes_\bZ -$ from $\Ab$ to $\Ab$ does not commute with infinite products.
\end{exercise}

\begin{exercise}
Let $X,Y\colon \cI \to \cC$ be functors, and let $\eta\colon X \to Y$ be a morphism of functors. Assume that $\lim_\cI X$ and $\lim_\cI Y$ exist. Show that $\alpha$ induces a morphism $\lim_\cI X \to \lim_\cI Y$ in $\cC$. Formulate and prove the analogous statement for colimits. 
\end{exercise}

\begin{exercise}
Show that the $\lim$ and $\colim$ in Example \ref{exa:descending-chain-of-inclusions} coincide with those described by
Propositions \ref{prop:limit-in-Set} and \ref{prop:colimit-in-Set}.
\end{exercise}

\begin{exercise}\label{exc:increasing-union}
Let  $S_0 \subset S_1 \subset S_2 \cdots $
be an infinite sequence of inclusions of sets. Show that the union $\cup_i S_i$ is the colimit
of a suitably chosen diagram $\cI \to \Set$.
\end{exercise}

\begin{exercise}Let $K$ be a field. Let $\cI$ be the category with $\ob \cI = \bN$ and
\[
	\Hom_\cI(i,j) = \begin{cases} \{\star\} & j\leq i \\ \emptyset & j > i \end{cases}
\]
Consider the diagram
\[
	R\colon \cI \to \CRing,\, i \mapsto R_i := K[X]/(X^i),
\]
(where for $j\leq i$ the map $K[X]/(X^i) \to K[X]/(X^j)$ is the quotient map). Show that $\lim_\cI R_i$ exists in $\CRing$, and is isomorphic to the power series ring $K[[X]]$.
\end{exercise}

\begin{exercise}
Let $G$ be a group and let $\rB G$ the category of Example \ref{exa:BG}. Let $F\colon \rB G \to \Set$ be a functor. 
\begin{enumerate}
\item Show that $F(\star)$ is a set $X$ equipped with an action of $G$.
\item Show that $\lim F$ is the set of fixed points of the action.
\item Show that $\colim F$ is the set of orbits of the action. 
\end{enumerate}
\end{exercise}

\begin{exercise}
Let $X$ be a topological space, and let $(U_i)_{i\in I}$ be an open cover. Let $\cI$ be the category with $\ob \cI = I$ and
\[
	\Hom(i,j) = \begin{cases} \{\star\} & U_i \subset U_j \\ \emptyset & \text{otherwise} \end{cases}
\]
Assume that for all $i,j$ there is a $k\in I$ such that $U_i \cap U_j = U_k$. Show that $\colim_\cI U_i = X$ in $\Top$.
\end{exercise}

\begin{exercise}
Let $\cZ$ and $\cR$ be the categories of Exercise \ref{exc:ordered-Z-and-R}. Let $\cI$ be the category with $\ob \cI = \bN$ and
\[
	\Hom_\cI(i,j) = \begin{cases} \{\star\} & i\leq j \\ \emptyset & i > j \end{cases}
\]
Verify that a functor $\cI \to \cR$ is the same as an increasing sequence of real numbers
\[
	x_0 \leq x_1 \leq x_2 \leq \cdots
\]
When does this functor have a limit? And a colimit? Verify directly if they are preserved by the left and right adjoints of the inclusion functor $\cZ \to \cR$.
\end{exercise}

\begin{exercise}\label{exc:colimit-in-Set}
Prove Proposition \ref{prop:colimit-in-Set}.
\end{exercise}

\begin{exercise}Show that all limits and colimits exist in the category $\Top$. 
(Hint: see Propositions \ref{prop:limit-in-Set} and \ref{prop:colimit-in-Set}).
\end{exercise}

\begin{exercise}[Arbitrary limits of modules]
Let $R$ be a ring, $\cI$ a small category, and $M\colon \cI \to {}_R\Mod,\, i \mapsto M_i$ a functor. Show that there is an exact sequence
\[
	0 \longto \lim_\cI M \longto \prod_{i} M_i \longto \prod_{f\colon i\to j} M_j
\]
of $R$-modules (in particular the limit exists). Here the first product ranges over all objects $i$ in $\cI$, and the second ranges over all triples $(i,j,f)$ with $i$ and $j$ objects in $\cI$ and $f\colon i\to j$ a morphism in $\cI$. Verify by hand that your exact sequence is correct in the special cases where the limit is a product or a pullback.
\end{exercise}

\begin{exercise}[Arbitrary colimits of modules]
Formulate and prove an analogous statement for colimits of modules.
\end{exercise}

\begin{exercise}
Let $R$ and $S$ be rings, let $A$ be an $(R,S)$-module and let 
\[
	0 \longto M_1 \longto M_2 \longto M_3
\]
be an exact sequence of $R$-modules. Use Theorem \ref{thm:right-adjoints-commute-with-limits} to prove that
the induced sequence
\[
	0 \longto \Hom_R(A,M_1) \longto \Hom_R(A,M_2) \longto \Hom_R(A,M_3)
\]
is exact in ${}_S\Mod$. (See Exercise \ref{exc:covariant-short-exact-hom} for a more direct approach). 
\end{exercise}

\begin{exercise}Consider the functor $F\colon \Grp \to \Ab,\, G\mapsto G^\ab$. (See Example \ref{exa:abelianization} and Exercise \ref{exc:abelianization-adjunction}).
\begin{enumerate}
\item Let $f\colon G_1\to H$ and $g\colon G_2\to H$ be group homomorphisms. Show that the pullback of $f$ and $g$ exists, and is isomorphic to
$\{ (s,t)\in G\times G \mid f(s)=g(t) \}$.
\item Let $f\colon \bZ/3\bZ \to S_3$ be an injective homomorphism and let $g\colon \{1\}\to S_3$ be the trivial homomorphism. Compute the pullback in $\Grp$ of $f$ and $g$, as well as the pullback in $\Ab$ of $F(f)$ and $F(g)$.
\item Conclude that $F$ does not have a left adjoint.
\item Show that $F$ does commute with finite products. 
\end{enumerate}
\end{exercise}


%%%%%%%%%%%%%%%%%%%%%%%%
% CHAIN COMPLEXES
%%%%%%%%%%%%%%%%%%%%%%


\chapter{Chain complexes}

\section{Chain complexes and their homology modules}


\begin{definition}
Let $R$ be a ring. A \emph{chain complex} of $R$-modules consists of
\begin{enumerate}
\item for every $i\in \bZ$ an $R$-module $M_i$,
\item for every $i\in \bZ$ an $R$-linear map $d_i\colon M_i \to M_{i-1}$
\end{enumerate}
such that for every $i$ the identity $d_i \circ d_{i+1} = 0 $ holds.
\end{definition}

We depict a chain complex as a diagram
\[
	\cdots \longto M_2 \overset{d_2}{\longto} M_1 \overset{d_1}{\longto}  M_0 
	\overset{d_0}{\longto} M_{-1} \longto \cdots 
\]
and often denote the chain complex by $M_\bullet$.

\begin{definition}
A \emph{morphism of chain complexes} $M_\bullet \to M'_\bullet$ consists of an $R$-module
homomorphism $f_i\colon M_i \to M_i'$ for every $i$, such that the resulting diagram
\[
\begin{tikzcd}
\cdots \arrow{r}
	& M_{2} \arrow{r}{d_{2}} \arrow{d}{f_{2}} 
	& M_{1} \arrow{r}{d_{1}} \arrow{d}{f_{1}} 
	& M_{0} \arrow{r} \arrow{d}{f_{0}} 
	& \cdots \\
\cdots \arrow{r}
	& M'_{2} \arrow{r}{d'_{2}} 
	& M'_{1} \arrow{r}{d'_{1}} 
	& M'_{0} \arrow{r} 
	& \cdots 
\end{tikzcd}
\]
commutes.  The resulting category of chain complexes of left $R$-modules is denoted 
${}_R\Ch$. The similarly-defined category of chain complexes of right $R$-modules is denoted $\Ch_R$.
\end{definition}

The condition $d_i \circ d_{i+1} = 0$ in a chain complex
 \[
	\cdots \longto M_{i+1} \overset{d_{i+1}}{\longto} M_i \overset{d_i}{\longto}  M_{i-1}
	\longto \cdots
\]
implies that $\im d_{i+1} \subset \ker d_i$ inside $M_i$. 

\begin{definition}
Let $M_\bullet$ be a chain complex of $R$-modules and $i$ an integer. The \emph{$i$-th homology module of $M_\bullet$} is the $R$-module
\[
	\rH_i(M_\bullet) := \frac{\ker d_i}{\im d_{i+1}}.
\]
\end{definition}

If $f\colon M_\bullet \to M'_\bullet$ is a morphism of chain complexes, then $f_i\colon M_i \to M_i'$ induces a morphism
\[
	\rH_i(f)  \colon \rH_i(M_\bullet) \to \rH_i(M'_\bullet)
\]
(see Exercise \ref{exc:induced-morphism-on-homology}). We obtain for every $i$ a functor
\[
	\rH_i \colon {}_R\Ch \to {}_R\Mod.
\]

These modules measure the failure of the sequence $M_\bullet$ to be exact, in the following sense.

\begin{lemma}
A chain complex
\[
	\cdots \longto M_2 \overset{d_2}{\longto} M_1 \overset{d_1}{\longto}  M_0 
	\overset{d_0}{\longto} M_{-1} \longto \cdots 
\]
is exact if and only if $\rH_i(M_\bullet)=0$ for all $i\in \bZ$. \qed
\end{lemma}

\section{The long exact sequence}

Let $M_\bullet$, $N_\bullet$ and $P_\bullet$ be chain complexes of $R$-modules. 
We say that a sequence of morphisms  
\[
	0 \longto M_\bullet \overset{\alpha}{\longto} N_\bullet 
	\overset{\beta}{\longto} P_\bullet \longto 0
\]
is a short exact sequence in ${}_R\Ch$ if it is termwise exact. In other words, 
if in the commutative diagram
\[
\begin{tikzcd}
	& 0  \arrow{d}
	& 0 \arrow{d}
	& 0 \arrow{d}
	&  \\
\cdots \arrow{r}
	& M_{i+1} \arrow{r}{} \arrow{d}{\alpha_{i+1}} 
	& M_{i} \arrow{r}{} \arrow{d}{\alpha_{i}} 
	& M_{i-1} \arrow{r} \arrow{d}{\beta_{i-1}} 
	& \cdots \\
\cdots \arrow{r}
	& N_{i+1} \arrow{r}{} \arrow{d}{\beta_{i+1}} 
	& N_{i} \arrow{r}{} \arrow{d}{\beta_{i}} 
	& N_{i-1} \arrow{r} \arrow{d}{\beta_{i-1}} 
	& \cdots \\
\cdots \arrow{r}
	& P_{i+1} \arrow{r}{} \arrow{d}
	& P_{i} \arrow{r}{}  \arrow{d}
	& P_{i-1} \arrow{r} \arrow{d}
	& \cdots \\
	& 0 
	& 0 
	& 0  
	&  
\end{tikzcd}
\]
all the columns are short exact sequences of $R$-modules. 

\begin{theorem}\label{thm:long-exact-sequence}
Let 
\[
	0 \longto M_\bullet \overset{\alpha}{\longto} N_\bullet 
	\overset{\beta}{\longto} P_\bullet \longto 0
\]
be a short exact sequence of chain complexes of $R$-modules. Then there exists
an exact sequence
\[
\begin{tikzcd}
 	\cdots \arrow{r}
 	& \rH_{i+1}(N_\bullet) \arrow{r}{\rH_{i+1}(\beta)}
 	& \rH_{i+1}(P_\bullet) \arrow[out=-5, in=175]{dll} \\
 	\rH_{i}(M_\bullet) \arrow{r}{\rH_{i}(\alpha)}
 	& \rH_{i}(N_\bullet) \arrow{r}{\rH_{i}(\beta)}
 	& \rH_{i}(P_\bullet) \arrow[out=-5, in=175]{dll} \\
 	\rH_{i-1}(M_\bullet) \arrow{r}{\rH_{i-1}(\alpha)}
 	& \rH_{i-1}(N_\bullet) \arrow{r}
 	& \cdots 
\end{tikzcd}
\]
of $R$-modules.
\end{theorem}

The resulting exact sequence of homology modules is called the \emph{long exact sequence of homology} associated to the short exact sequence of chain complexes.

\begin{proof}[Proof of Theorem \ref{thm:long-exact-sequence}]
We omit the proof, which is a tedious diagram chase in the style of the proof of the Snake Lemma (Theorem \ref{thm:snake-lemma}). In fact, the Snake Lemma is a special case of the theorem, obtained by assuming the $M_i$, $N_i$ and $P_i$ vanish for $i\not\in \{0,1\}$.
\end{proof}

\section{The homotopy category}

\begin{definition}
Let $R$ be a ring and let $f\colon M_\bullet \to M'_\bullet$ and $g\colon M_\bullet \to M'_\bullet$ be morphisms of chain complexes of $R$-modules. A \emph{homotopy} from $f$ to $g$ consists of
a collection of $R$-linear maps $h_i\colon M_i \to M'_{i+1}$ indexed by $i\in \bZ$, such that for every $i\in \bZ$ the identity
\[
	g_i - f_i =  d'_{i+1} h_i + h_{i-1} d_i
\]
holds in $\Hom_R( M_i, M'_i )$. We say that $f$ and $g$ are \emph{homotopic}, and write $f\sim g$, if there exists a homotopy from $f$ to $g$. 
\end{definition}

It is convenient to keep track of these maps in a diagram:
\[
\begin{tikzcd}[row sep=large, column sep=large]
\cdots \arrow{r}
	& M_{2} \arrow{r}{d_{2}} 
		\arrow[transform canvas={xshift=-.5ex},swap]{d}{f_{2}} 
		\arrow[transform canvas={xshift=.5ex}]{d}{g_{2}} 
		\arrow[dashed]{dl}{}
	& M_{1} \arrow{r}{d_{1}}  
		\arrow[transform canvas={xshift=-.5ex},swap]{d}{f_{1}} 
		\arrow[transform canvas={xshift=.5ex}]{d}{g_{1}} 
		\arrow[dashed]{dl}[description]{h_1}
	& M_{0} \arrow{r} 
		\arrow[transform canvas={xshift=-.5ex},swap]{d}{f_{0}} 
		\arrow[transform canvas={xshift=.5ex}]{d}{g_{0}} 
		\arrow[dashed]{dl}[description]{h_0}
	& \cdots \arrow[dashed]{dl}{} \\
\cdots \arrow{r}
	& M'_{2} \arrow[swap]{r}{d'_{2}} 
	& M'_{1} \arrow[swap]{r}{d'_{1}} 
	& M'_{0} \arrow[swap]{r} 
	& \cdots 
\end{tikzcd}
\]
but note that in the definition of homotopy it is \emph{not} required that the $h_i$'s commute with the horizontal maps $d$ in any way.

The equation defining homotopy can be remembered as
\[
	g-f=dh+hd,
\]
omitting the indices which can be reinserted in only one meaningful way. 

\begin{proposition}\label{prop:homotopy-equivalence-relation}
Let $R$ be a ring and $M_\bullet$, $M'_\bullet$ chain complex of $R$-modules. Then homotopy is an equivalence relation on the set
 $\Hom_{{}_R\Ch}(M_\bullet,M'_\bullet)$. \qed
\end{proposition}



\begin{proposition}\label{prop:homotopy-composition}
Let $R$ be a ring and $f,g \colon M_\bullet \to M'_\bullet$ homotopic morphisms of chain complexes of $R$-modules. Then
\begin{enumerate}
\item for any morphism $s\colon M'_\bullet \to N_\bullet$ in ${}_R\Ch$, the compositions $sf$ and $sg$ are homotopic;
\item for any morphism $t\colon N_\bullet \to M_\bullet$ in ${}_R\Ch$, the compositions $ft$ and $gt$ are homotopic.
\end{enumerate}
\end{proposition}


\begin{proof}
Let $(h_i)_i$ be a homotopy from $f$ to $g$. In the first case, one  verifies that $(s_{i+1}h_i)_i$ is a homotopy from $sf$ to $sg$, and in the second case,  that $(h_it_i)_i$ is a homotopy from $ft$ to $gt$. 
\end{proof}

\begin{definition}
The \emph{homotopy category of chain complexes of $R$-modules}, denoted ${}_R\Ho$,  is the
category with
\begin{enumerate}
\item $\ob {}_R\Ho := \ob {}_R\Ch$
\item $\Hom_{{}_R\Ho}(M_\bullet,N_\bullet) := \Hom_{{}_R\Ch}(M_\bullet, N_\bullet) / \sim$
\end{enumerate}
where composition and identity maps are inherited from composition and identity maps in ${}_R\Ch$.
\end{definition}

Proposition  \ref{prop:homotopy-composition} guarantees that composition in ${}_R\Ho$ is well-defined. 

\begin{proposition}\label{prop:homotopic-maps-agree-on-homology}
Let $f,g\colon M_\bullet \to M'_\bullet$ be homotopic maps in $\Ch_R$, and let $i$ be an integer. Then $\rH_i(f)=\rH_i(g)$ as maps $\rH_i(M_\bullet) \to \rH_i(M'_\bullet)$.
\end{proposition}

\begin{proof}
By definition, an element of $\rH_i(M_\bullet)$ is a coset
\[
	 \bar{x} := x + \im(d_{i+1})
\]
for some  $x\in \ker(d_i)$. We have
\begin{align*}
	\rH_i(f)(\bar{x}) - \rH_i(g)(\bar{x}) &=  f(x) - g(x) + \im (d'_{i+1})  \\
	&= d'_{i+1}(h_i(x)) + h_{i-1}(d_i(x))  +  \im (d'_{i+1}).
\end{align*}
Since $d'_{i+1}(h_i(x))  \in \im(d'_{i+1})$, the first term vanishes in $\rH_i(M'_\bullet)$. Since $x\in \ker(d_i)$, also the second term vanishes, and  $\rH_i(f)=\rH_i(g)$.
\end{proof}

A consequence of Proposition \ref{prop:homotopic-maps-agree-on-homology} is that the functors $\rH_i$ on ${}_R\Ch$ induce  functors
\[
	\rH_i\colon {}_R\Ho \to {}_R\Mod,\, M_\bullet \mapsto \rH_i(M_\bullet)
\]
on the homotopy category of chain complexes.

\begin{remark}As the terminology suggests, there is a close relationship with the notions of homotopic maps and homology groups in algebraic topology. Homology groups of topological spaces are usually defined in terms of the functor
\[
	C\colon \Top \to {}_\bZ\Ch,\, X \mapsto C_\bullet(X)
\]
that maps a space $X$ to the chain complex of \emph{singular chains}. One then defines the
$i$-th homology group of $X$ by
\[
	\rH_i(X,\bZ) := \rH_i(C_\bullet(X))
\]
and obtains functors $\rH_i\colon \Top \to \Ab$. 

If $f,g\colon X\to Y$ are homotopic continuous maps, then their induced maps $C_\bullet(X) \to C_\bullet(Y)$ are homotopic maps of chain complexes, and hence their induced maps $\rH_i(X,\bZ) \to \rH_i(Y,\bZ)$ coincide.
\end{remark}

\begin{remark}
Chain complexes form a mathematical context in which three layers play a role: objects (chain complexes), morphisms (morphisms of chain complexes), and maps between morphisms (homotopies). See Remark \ref{rmk:2-cat} for two other such contexts: topological spaces (spaces, continuous maps, homotopies), and categories (categories, functors, morphisms of functors).
\end{remark}


\newpage
\section*{Exercises}

\begin{exercise}[Functoriality of homology]\label{exc:induced-morphism-on-homology}
Let $f\colon M_\bullet \to M'_\bullet$ be a morphism of chain complexes of $R$-modules. 
\begin{enumerate}
\item Show  $f_i(\ker d_i) \subset \ker d'_i$;
\item Show $f_i(\im d_{i+1}) \subset \im d'_{i+1}$;
\item Conclude that $f_i$ induces an $R$-linear map $\rH_i(M) \to \rH_i(M')$.
\end{enumerate}
\end{exercise}

\begin{exercise}
Let 
\[
	\ldots \to M_2 \to M_1 \to M_0 \to N \to 0
\]
be an exact sequence of $R$-modules. Show that the complex
\[
	M_\bullet = \big( \ldots \to M_2 \to M_1 \to M_0  \to 0 \to \cdots \big)
\]
satisfies $\rH_0(M_\bullet)\cong N$ and $\rH_n(M_\bullet)=0$ for all $n\neq 0$.
\end{exercise}


\begin{exercise}\label{exc:an-isomorphism-in-the-homotopy-category}
Let $R$ be a non-zero ring. Show that the  two chain complexes
\[
\begin{tikzcd}[row sep=small]
\cdots \arrow{r}
	& 0 \arrow{r}
	& 0 \arrow{r}
	& 0 \arrow{r}
	& 0 \arrow{r}
	&\cdots \\ 
\cdots \arrow{r}
	& 0 \arrow{r}
	& R \arrow{r}{\id}
	& R \arrow{r}
	& 0 \arrow{r}
	&\cdots
\end{tikzcd}
\]
are isomorphic in ${}_R\Ho$.
\end{exercise}

\begin{exercise}Let $n>1$ and let $f$ be the morphism in ${}_\bZ\Ch$ given by the diagram
\[
\begin{tikzcd}
\cdots \arrow{r}
	& 0 \arrow{r} \arrow{d}
	& \bZ \arrow{r}{n} \arrow{d}
	& \bZ \arrow{r} \arrow{d}{\pi}
	& 0 \arrow{r} \arrow{d}
	&\cdots \\ 
\cdots \arrow{r}
	& 0 \arrow{r}
	& 0 \arrow{r}
	& \bZ/n\bZ \arrow{r}
	& 0 \arrow{r}
	&\cdots
\end{tikzcd}
\]
with $\pi$ the canonical map. Show that $\rH_i(f)$ is an isomorphism for all $i$, but that $f$ is not an isomorphism in ${}_\bZ\Ho$.
\end{exercise}

\begin{exercise}Let $R$ and $S$ be rings. A functor $F\colon {}_R\Mod\to {}_S\Mod$ is
called \emph{additive} if for all $R$-modules $M$, $N$ the map
\[
	F\colon \Hom_R(M,N) \to \Hom_S(FM,FN)
\]
is a homomorphism of abelian groups.  Let $A$ be an $(R,S)$-bimodule. Show that the functor
\[
	{}_R\Mod \to {}_S\Mod,\, M \mapsto \Hom_R(A,M)
\]
is additive.
\end{exercise}

\begin{exercise}
Show that an additive functor $F\colon {}_R\Mod \to {}_S\Mod$ induces functors
${}_R\Ch\to {}_S\Ch$ and ${}_R\Ho\to {}_S\Ho$.
\end{exercise}

\begin{exercise}\label{exc:contravariant-hom-on-chain-complexes}
Let $R$ and $S$ be rings and $A$ an $(R,S)$-bimodule. For a chain complex $M_\bullet$ in ${}_R\Ch$ define a chain complex $M'_\bullet$ in $\Ch_S$
by
\begin{enumerate}
\item $M'_i := \Hom_R(M_{-i},A)$ 
\item $d_i\colon M'_i \to M'_{i-1}$ the map induced from $d_{1-i} \colon M_{1-i} \to M_{-i}$
\end{enumerate}
Verify that $M'_\bullet$ is a chain complex of right $S$-modules, and that the operation $M_\bullet \mapsto M'_\bullet$ defines functors ${}_R\Ch^\opp \to \Ch_S$ and ${}_R\Ho^\opp \to \Ho_S$.
\end{exercise}

\begin{exercise}[$\star$]
Let $G$ be a group. Consider the abelian groups $C_n(G) := \bZ^{(G^n)}$. To ease the notation, denote the basis vector $e_{(g_1,\ldots,g_n)}$ by $[g_1,\ldots,g_n] \in C_n(G)$. Consider the morphisms
\[
	d_n \colon C_n(G) \to C_{n-1}(G) 
\]
defined on the basis of $C_n(G)$ by
\begin{align*}
	[g_1,\ldots, g_n] \mapsto &[g_2, g_3, \ldots, g_n] \\
	&+ \sum_{i=1}^{n-1} (-1)^i [g_1,\ldots, g_{i-1}, g_i g_{i+1}, g_{i+2}, \ldots, g_n] \\
	& + (-1)^n [ g_1,\ldots, g_{n-1} ].
\end{align*}
Set $C_n(G)=0$ for $n<0$.
\begin{enumerate}
\item Show that $C_\bullet(G)$ is a chain complex of abelian groups.
\end{enumerate}
The homology groups
$\rH_n(C_\bullet(G))$ are called the \emph{homology groups} of $G$, and are denoted $\rH_n(G,\bZ)$.
\begin{enumerate}
\item[(2)] Show that $\rH_0(G,\bZ)\cong \bZ$;
\item[(3)] Show that $\rH_1(G,\bZ)\cong G^\ab$.
\end{enumerate} 
\end{exercise}


\begin{exercise}[$\star$]
Let $K$ be a field. Let $\prod_{i\in \bZ} \Vec_K$ be the category whose objects are sequences $(V_i)_{i\in \bZ}$ of $K$-vector spaces, and whose morphisms are sequences $(f_i)_{i\in \bZ}$ of $K$-linear maps. Show that the functor
\[
	\Ho_K \to \prod_{i\in \bZ} \Vec_K,\, V_\bullet \mapsto \left( \rH_i(V_\bullet) \right)_{i}
\]
is an equivalence of categories.
\end{exercise}



\chapter{Free resolutions}

\section{Definition and existence}




\begin{definition}
Let $R$ be a ring and let $M$ be an $R$-module. A \emph{free resolution} of $M$ is an exact sequence
of $R$-modules
\[
	\cdots \longto F_2 \overset{d_2}{\longto} F_1 \overset{d_1}{\longto} F_0 \overset{\pi}{\longto} M \longto 0 
\]
in which the modules $F_i$ are free.
\end{definition}

\begin{example}
If $M$ itself is free, then the exact sequence
\[
	\cdots \longto 0 \longto 0 \longto M \overset{\id}{\longto} M \longto 0
\]
is a free resolution (with $F_0=M$ and $F_i=0$ for $i\neq 0$). We will usually suppress leading zeroes from the notation, and simply write 
\[
	0 \longto M \longto M \longto 0
\]
for the above resolution.
\end{example}

\begin{example}Let $R$ be an integral domain. 
If $I \subset R$ is a non-zero principal ideal, then $I$ is free of rank $1$ as an $R$-module (see Exercise \ref{exc:principal-ideal-in-domain}), hence the exact sequence
\[
	0 \longto I \longto R \longto R/I \longto 0
\]
is a free resolution of the $R$-module $R/I$.
\end{example}

\begin{example}
If $R$ is a principal ideal domain, and $M$ a finitely generated $R$-module, then we have seen in
Corollary \ref{cor:free-presentation} that $M$ has a free resolution of the form
\[
	0 \longto F_1 \longto F_0 \longto M \longto 0
\]
with $F_0$ and $F_1$ free $R$-modules of finite rank.
\end{example}

\begin{example}
Let $K$ be a field, and let $R=K[X,Y]$. Then the sequence
\[
	0 \longto R  \overset{d_2}{\longto} R \oplus R
	\overset{d_1}{\longto} R \overset{\pi}{\longto} K[X,Y]/(X,Y) \longto 0 
\]
with $d_2(1) = (Y,-X)$, $d_1(1,0) = X$ and $d_1(0,1)=Y$, and $\pi$ the quotient map is a free resolution of the $R$-module $K[X,Y]/(X,Y)$.
\end{example}

Every $R$-module has a free resolution, although in general the resolution need not be of finite length, and the free modules occurring in it need  not be of finite rank.

\begin{proposition}\label{prop:free-resolutions-exist}
Let $R$ be a ring. Then every $R$-module $M$ has  a  free resolution.
\end{proposition}

\begin{proof}
Choose a generating set $I$ of $M$, and let $F_0= R^{(I)}$ be the free $R$-module with basis $I$. Then we have a natural surjective map $\pi\colon F_0\to M$, and hence an exact sequence
\[
	F_0 \overset{\pi}{\longto} M \longto 0.
\]
Now choose a generating set $I_1$ of the $R$-module 
 $K_0 := \ker \pi \subset F_0$ and let $F_1=R^{(I_0)}$. The natural map $F_1 \to K_0 \subset F_0$
 extends the above to an exact sequence
 \[
 	F_1 \overset{d_1}{\longto} F_0 \overset{\pi}{\longto} M \longto 0.
\]
Repeating this argument with $K_1 := \ker d_1$ and so forth, we obtain an exact sequence
\[
	\cdots \longto F_2 \overset{d_2}{\longto} F_1 \overset{d_1}{\longto} F_0 \overset{\pi}{\longto} M \longto 0,
\]
with the $F_i$ free, as we had to show.
\end{proof}

\begin{remark}From a free resolution
\[
	\cdots \longto F_2 \overset{d_2}{\longto} F_1 \overset{d_1}{\longto} F_0 \overset{\pi}{\longto} M \longto 0 
\]
of $M$ we obtain a chain complex $F_\bullet$ of the form
\[
	\cdots \longto F_2 \overset{d_2}{\longto} F_1 \overset{d_1}{\longto} F_0 \longto 0 \longto  0 \longto \cdots
\]
by setting set $F_{i}=0$ for $i<0$.  This chain complex satisfies
\[
	 \rH_i(F_\bullet) \cong \begin{cases} M & i=0 \\ 0 & i \neq 0 \end{cases}
\]
Conversely, given a pair $(F_\bullet,\alpha)$ consisting of
\begin{enumerate}
\item a chain complex $F_\bullet$ 
\item an isomorphism $\alpha\colon \rH_0(F_\bullet) \to M$
\end{enumerate}
with $F_i$ free for all $i$ and zero for $i<0$, and with $\rH_i(F_\bullet)=0$ for all $i\neq 0$, the sequence
\[
	\cdots \longto F_2 \longto F_1 \longto F_0 \overset{\pi} \longto M \longto 0,
\]
where $\pi$ is induced by $\alpha$, is a free resolution of $M$. 

It will often be convenient to think of a free resolution as a pair $(F_\bullet,\alpha)$ consisting of a chain complex $F_\bullet$ and an isomorphism $\alpha$ as above.
\end{remark}

\section{The free resolution functor}

% TODO: make the theorem below more precise

\begin{theorem}\label{thm:functoriality-of-free-resolutions}
Let $M$ and $M'$ be $R$-modules. Let $F_\bullet$ and $F'_\bullet$ be free resolutions of $M$ and $M'$ respectively. Let $\varphi \colon M\to M'$ be a morphism of $R$-modules. Then there exists a morphism
$f\colon F_\bullet \to F'_\bullet$ such that $\rH_0(f) = \varphi$. Moreover, $f$ is unique up to homotopy.
\end{theorem}

\begin{proof}
The existence of $f$ amounts to the existence of $R$-linear maps $f_i$ making the diagram 
(with exact rows)
\[
\begin{tikzcd}
\cdots \arrow{r} & 
	F_2 \arrow[dashed]{d}{f_2} \arrow{r}{d_2} &
	F_1 \arrow[dashed]{d}{f_1} \arrow{r}{d_1} & 
	F_0 \arrow[dashed]{d}{f_0} \arrow{r}{\pi} & M \arrow{d}{\varphi} \arrow{r} & 0 \\
\cdots \arrow{r} & 
	F_2'  \arrow{r}{d'_2} & 
	F_1'  \arrow{r}{d'_1} & 
	F_0' \arrow{r}{\pi'} & M'  \arrow{r} & 0 
\end{tikzcd}
\]
commute. We will construct such a diagram inductively.

 Let $S_0\subset F_0$ be a basis of the free module $F_0$. For every $s\in S_0$, choose an $s'\in F'_0$ such that $\pi'(s')=\varphi\pi(s)$. Such $s'$ exists by the surjectivity of $\pi'$. Now, since $F_0$ is free with basis $S_0$, there exists a unique $R$-linear map $f_0\colon F_0\to F'_0$ that maps every $s\in S_0$ to its chosen counterpart $s'\in F'_0$. By construction the right-hand square commutes. 

Next, let $S_1\subset F_1$ be a basis of $F_1$. For every $s\in S_1$, we have $\pi d_1(s)=0$ by the exactness of the top-row, hence $\pi' f_0 d_1(s) = 0$ by the commutativity of the right-hand square, and hence $f_0d_1(s) \in \ker \pi'$. We conclude that there exists an element $s'\in F'_1$ with $d_1'(s') = f_0d_1(s)$. Choosing such an $s'$ for every $s\in S_1$ yields an $R$-linear map $f_1\colon F_1 \to F'_1$ as above. Repeating the argument, we construct maps $f_i$ as required.

 For the uniqueness assertion in the theorem, assume that we have chain complex homomorphisms $f\colon F_\bullet \to F'_\bullet$ and $g\colon F_\bullet \to F'_\bullet$ with $\rH_0(f)=\rH_0(g)=\varphi$. Let $\delta := g-f$. We need to show that there are $h_i$ as in the diagram below
 \[
\begin{tikzcd}
\cdots \arrow{r} & F_2 \arrow{r}{d_2} \arrow{d}{\delta_2} \arrow[dashed]{dl}
	& F_1 \arrow{d}{\delta_1} \arrow{r}{d_1} \arrow[dashed]{dl}[description]{h_1}
	& F_0 \arrow{d}{\delta_0} \arrow{r}{\pi} \arrow[dashed]{dl}[description]{h_0}
	& M \arrow{d}{0} \arrow{r} & 0 \\
\cdots \arrow{r} & F_2' \arrow[swap]{r}{d'_2} 
	& F_1'  \arrow[swap]{r}{d'_1} 
	& F_0' \arrow[swap]{r}{\pi'} 
	& M'  \arrow{r} & 0 
\end{tikzcd}
\]
 satisfying
\begin{align*}
	\delta_0 &= d_1'h_0 \\
	\delta_i &= d_{i+1}'h_i + h_{i-1} d_i \quad (i\geq 1)
\end{align*}
As in the proof of the first part of the theorem, we construct these $h_i$ inductively. 

For the base step,  choose a basis $S_0$ of $F_0$ and for every $s\in S_0$ choose an $s' \in F'_1$ with $d'_1 s' = \delta_0 s$. Such $s'$ exists, since $\pi'\delta_0 s=0$, and the bottom row in the diagram is exact. Since $F_0$ is free with basis $S_0$, there is a (unique) morphism $h_0\colon F_0\to F'_1$ with $s\to s'$ for every $s\in S_0$, and we have $\delta_0 = d_1'h_0$ by  construction.

Next, let $S_1$ be a basis of $F_1$, and choose for every $s\in S_1$ an $s'\in F_2'$ with $d'_2s' = \delta_1s - h_0d_1s$. Such $s'$ exists, since
\[
	d_1'(\delta_1s - h_0d_1s) = d_1'\delta_1 s - d_1'h_0d_1s
	= \delta_0 d_1 s - \delta_0 d_1 s = 0.
\]
As before, the collection of $s'$ defines a map $h_1\colon F_1 \to F_2'$, and repeating the argument gives a collection of maps $h_i$ defining the desired homotopy between $f$ and $g$.
\end{proof}

\begin{corollary}
If $F_\bullet$ and $F'_\bullet$ are free resolutions of $M$, then $F_\bullet$ and $F'_\bullet$ are isomorphic in ${}_R\Ho$.
\end{corollary}

\begin{proof}
The proof is `abstract nonsense' and quite similar to the argument that showed that final objects are unique up to unique isomorphism (see Proposition \ref{prop:final-object-uniquely-unique}).

Take $M':=M$ and apply Theorem \ref{thm:functoriality-of-free-resolutions} to $\id\colon M\to M'$ and $\id\colon M'\to M$ to obtain morphisms $f\colon F_\bullet\to F'_\bullet$ and $g\colon F'_\bullet \to F_\bullet$. Then apply Theorem \ref{thm:functoriality-of-free-resolutions} again to $\id_M$ and $\id_{M'}$ to show that $gf$ is homotopic to $\id_{F_\bullet}$ and $fg$ is homotopic to $\id_{F'_\bullet}$. This gives equalities $fg=\id_{F_\bullet}$ and $gf=\id_{F'_\bullet}$ in ${}_R\Ho$, which shows that $f$ and $g$ are mutually inverse isomorphisms in ${}_R\Ho$.
\end{proof}

\begin{example}
The zero module $M=\{0\}$ has the zero resolution, but also the non-trivial free resolution
\[
	0 \longto R \overset{\id}{\longto} R \overset{\pi}{\longto} M \longto 0,
\]
hence the corresponding complexes 
\[
\begin{tikzcd}[row sep=small]
\cdots \arrow{r}
	& 0 \arrow{r}
	& 0 \arrow{r}
	& 0 \arrow{r}
	& 0 \arrow{r}
	&\cdots \\ 
\cdots \arrow{r}
	& 0 \arrow{r}
	& R \arrow{r}
	& R \arrow{r}
	& 0 \arrow{r}
	&\cdots
\end{tikzcd}
\]
are isomorphic in ${}_R\Ho$. See also Exercise \ref{exc:an-isomorphism-in-the-homotopy-category}.
\end{example}


We can now summarise this section into one powerful theorem.

\begin{theorem}\label{thm:free-resolution-functor}
There exists a functor
\[
	F\colon {}_R\Mod \to {}_R\Ho,\, M \mapsto F_\bullet(M)
\]
and an isomorphism of functors
\[
	\alpha\colon \rH_0 \circ F \longisomto \id_{{}_R\Mod}
\]
such that for every $R$-module $M$, the complex $F_\bullet(M)$ together with the isomorphism $\alpha_M$ forms a free resolution of $M$.
\end{theorem}

The proof goes directly against our basic principle that `constructions depending on choices do not give rise to functors'. 

\begin{proof}[Proof of Theorem \ref{thm:free-resolution-functor}]
Using Proposition \ref{prop:free-resolutions-exist}, choose for every $R$-module $M$  a free resolution
\[
	\cdots \longto F_2(M) \longto F_1(M) \longto F_0(M) \overset{\pi_M} \longto M \longto 0.
\]
This defines for every $R$-module $M$ an object $F_\bullet(M) \in {}_R\Ho$, and an isomorphism
$\alpha_M\colon \rH_0(F_\bullet(M)) \isomto M$ (induced by $\pi_M$).

Now for every $\varphi\colon M\to N$ in ${}_R\Mod$, Theorem \ref{thm:functoriality-of-free-resolutions} gives
a \emph{unique} morphism $F_\bullet(\varphi) \colon F_\bullet(M) \to F_\bullet(N)$ such that the square
of $R$-modules
\[
\begin{tikzcd}
\rH_0(F_\bullet(M)) \arrow{r}{\alpha_M} \arrow{d}{H_0(F_\bullet(\varphi))} & M \arrow{d}{\varphi} \\
\rH_0(F_\bullet(N)) \arrow{r}{\alpha_N} & N
\end{tikzcd}
\]
commutes. This provides the necessary data for a functor $F_\bullet$, and immediately shows that $\alpha$ is an isomorphism of functors, provided that the data defining $F_\bullet$ indeed forms a functor.

For this, we need to check that $F_\bullet$ respects identity and composition. But this follows quite formally from the uniqueness statement in Theorem \ref{thm:functoriality-of-free-resolutions}. Given an $R$-module $M$, both $\id_{F_\bullet(M)}$ and $F_\bullet(\id)$ induce the identity on $\rH_0(F_\bullet(M))=M$, so they must be homotopic and hence they define the same morphism in ${}_R\Ho$. Similarly, given $f\colon M\to N$ and $g\colon N\to P$ then both
\[
	F_\bullet(g) F_\bullet(f)\colon F_\bullet(M) \to F_\bullet(N)
\]
and
\[
	F_\bullet(gf)\colon F_\bullet(M) \to F_\bullet(N)
\]
induce the map $gf\colon M\to P$ on $\rH_0$, so they must be homotopic and hence define the same morphism in ${}_R\Ho$.
\end{proof}



\newpage
\section*{Exercises}

\begin{exercise}
Consider the ring $R=\bZ[X]$. Give a free resolution of the $R$-module $\bZ[X]/(X,2)$.
\end{exercise}

\begin{exercise}Let $K$ be a field and consider the subring $R=K[X^2,X^3]$ of the polynomial ring $K[X]$. Let $M$ be the $R$-module $R/(X^2,X^3)$. Find a free resolution of $M$.
\end{exercise}


\begin{exercise}
Let $R$ be a commutative ring. 
\begin{enumerate}
\item Assume that $r\in R$ is not a zero divisor in $R$. Show that $R/rR$ has a free resolution of the form
\[
	0 \longto R \longto R \longto R/(r) \longto 0.
\]
\item Let $r,s\in R$. Assume that $r$ is not a zero divisor in $R$, and that $\bar{s}$ is not a zero divisor in $R/rR$. Show that $R/(s,r)$ has a free resolution of the form
\[
	 0 \longto R \longto R^2 \longto R \longto R/(r,s) \longto 0.
\]
\item[(3, $\star$)] Try to formulate and prove an analogous statement for modules of the form $R/(r,s,t)$, etcetera.
\end{enumerate}
\end{exercise}

\begin{exercise}\label{exc:free-resolution-finite-cyclic-group}
Let $n$ be a positive integer, and consider the ring $R := \bZ[X]/(X^n-1)$. Let $M$ be the quotient module $R/(X-1)$, $\pi \colon R\to M$ the quotient map. Show that
\[
	\cdots \overset{\beta}\longto R \overset{\alpha}\longto R \overset{\beta}\longto R \overset{\alpha}\longto R \overset{\pi}{\longto} M \longto 0,
\]
with $\alpha(r) = (X-1)r$ and $\beta(r)=(X^{n-1} + \cdots + X + 1)r$, is a free resolution of the $R$-module $M$. 
\end{exercise}



\begin{exercise}
Let $R=\bZ/4\bZ$ and let $M$ be the $R$-module $\bZ/2\bZ$. Find a free resolution of $M$.
\end{exercise}

\begin{exercise}
Let $K$ be a field, $n>1$ and let $R$ be the matrix ring $\Mat_n(K)$. Let $M=K^n$ be the left $R$-module of column vectors. Show that $M$ does not have a finite free resolution consisting of finitely generated free $R$-modules.
\end{exercise}

\begin{exercise}
Let $0\to M_1\to M_2 \to M_3 \to 0$ be a short exact sequence of $R$-modules. Show that there exist free $R$-modules $F_1$, $F_2$, $F_3$, and a commutative diagram
\[
\begin{tikzcd}
 0 \arrow{r} & F_1 \arrow{r} \arrow[two heads]{d}
 	& F_2 \arrow{r} \arrow[two heads]{d}
	& F_3 \arrow{r} \arrow[two heads]{d} & 0 \\
0 \arrow{r} & M_1 \arrow{r} & M_2  \arrow{r} & M_3 \arrow{r} & 0
\end{tikzcd}
\]
with exact rows and surjective vertical maps.
\end{exercise}

\begin{exercise}[Free resolution of a short exact sequence]\label{exc:short-exact-sequence-resolutions}
Let $0\to M_1\to M_2 \to M_3 \to 0$ be a short exact sequence of $R$-modules. Show that there exist free resolutions
\[
	\cdots \longto F_{i,2} \longto F_{i,1} \longto F_{i,0} \longto M_i \longto 0
\]
for $i=1,2,3$, and a short exact sequence
\[
	0 \longto F_{1,\bullet} \longto F_{2,\bullet} \longto F_{3,\bullet} \longto 0
\]
of chain complexes compatible with the exact sequence $0\to M_1\to M_2 \to M_3 \to 0$.
\end{exercise}

\begin{exercise}[Uniqueness of free resolution functor]
Let $F$ and $G$ be functors ${}_R\Mod \to {}_R\Ho$. Let $\alpha\colon \rH_0\circ F\isomto \id$ and $\beta\colon \rH_0\circ G \isomto \id$ be isomorphisms. Assume that for every $M$ the pairs $(F(M),\alpha)$ and $(G(M),\beta)$ are free resolutions of $M$. Show that the functors $F$ and $G$ are isomorphic.
\end{exercise}

\begin{exercise}\label{exc:resolution-of-length-two}
Let $M_\bullet$ be a chain complex of $R$-modules with $M_i=0$ for all $i\not= 0,1$. Show that there exists a chain complex $F_\bullet$ and a morphism $\alpha\colon F_\bullet\to M_\bullet$ such that
\begin{enumerate}
\item $\rH_i(\alpha)$ is an isomorphism for all $i$, and
\item $F_i$ is free for all $i$.
\end{enumerate}
\end{exercise}

\begin{exercise}[$\star$] Let $M_\bullet$ be a chain complex of $R$-modules with $M_i=0$ for all $i<0$. Show that there exists an $F_\bullet$ and $\alpha$ as in Exercise \ref{exc:resolution-of-length-two}.
\end{exercise}


\chapter{The $\Ext$ functors}

\section{The functors $\Ext^n$}

Let $R$ be a ring. If $M_\bullet$ is the complex of $R$-modules given by
\[
	\cdots \longto M_{i+1} \longto   M_i \longto M_{i-1} \longto \cdots
\]
and if $N$ is an $R$-module, then the induced sequence 
\[
	\cdots \longto \Hom_R(M_{i-1}, N) \longto
	\Hom_R(M_i, N) \longto \Hom_R(M_{i+1}, N) \longto \cdots
\]
forms a complex of abelian groups, with the group $\Hom_R(M_{-i},N)$ in degree $i$. This determines a functor
\[
	{}_R\Ch^\opp \times {}_R\Mod \to {}_\bZ \Ch
\]
and  it induces a functor 
\[
	{}_R\Ho^\opp  \times {}_R\Mod \to {}_\bZ \Ho.
\]
on the homotopy categories,  by Exercise \ref{exc:contravariant-hom-on-chain-complexes}.


\begin{definition}
Let $n$ be an integer. We define the functor
\[
	\Ext^n_R(-,-) \colon {}_R\Mod^\opp \times {}_R\Mod \to \Ab,\,
	(M,N) \mapsto \Ext^n_R(M,N)
\]
as the composition of the following functors:
\begin{enumerate}
\item the free resolution functor of Theorem \ref{thm:free-resolution-functor} (applied to the first coordinate)
\[
	F_\bullet(-) \times \id \colon {}_R\Mod^\opp \times {}_R\Mod \longto {}_R \Ho^\opp \times {}_R\Mod
\]
\item the functor induced by $\Hom_R$ as above
\[
	 {}_R \Ho^\opp \times {}_R\Mod \longto {}_\bZ \Ho
\]
\item the homology functor
\[
	\rH_{-n} \colon  {}_\bZ \Ho \to \Ab,
\]
which is well-defined by Proposition \ref{prop:homotopic-maps-agree-on-homology}.
\end{enumerate}
\end{definition}

In other words, if
\[
	\cdots \longto F_1 \longto F_0 \longto M \longto 0
\]
is a free resolution of $M$, then the group $\Ext^n_R(M,N)$ is defined as
the quotient group
\[
	\Ext^n_R(M,N) = 
	\frac{ \ker \big( \Hom_R(F_n,N) \to \Hom_R(F_{n+1},N) \big)}
	{\im \big( \Hom_R(F_{n-1},N) \to \Hom_R(F_{n},N) \big)},
\]
where the maps between the $\Hom$ groups are induced from the maps in the free resolution, and where we set $F_i=0$ for $i<0$, as before.


\begin{remark}
A priori the functor $\Ext^n_R(-,-)$ depends on the choices of free resolutions involved in $F_\bullet(-)$, but different choices give rise to isomorphic functors $\Ext^n_R(-,-)$.
\end{remark}

\begin{example}
Let $m$ be a positive integer. We compute the groups $\Ext^n_{\bZ}(\bZ/m\bZ,\,\bZ)$ using the definition. As a first step we need to find a free resolution of $\bZ/m\bZ$. The obvious choice
\[
	0 \longto \bZ \overset{m}{\longto} \bZ \longto \bZ/m\bZ \longto 0
\] 
leads to the complex
\[
	F_\bullet = \big[ \cdots \longto 0 \longto \bZ \overset{m}{\longto} \bZ \longto 0 \longto 0 \longto \cdots \big]
\]
with $F_1=F_0=\bZ$. Note that $\Hom(\bZ,\bZ)= \bZ$, so that applying the contravariant additive functor $\Hom_\bZ(-,\,\bZ)$ to this complex gives a complex of the form
\[
	H_\bullet = \big[ \cdots \longto 0 \longto 0 \longto \bZ \longto \bZ \longto 0 \longto \cdots \big]
\]
with $H_0=H_{-1} = \bZ$. One checks that the map $\bZ\to\bZ$ is multiplication by $m$. We find $\rH_i(H_\bullet)=0$ for all $i\neq -1$, and $\rH_{-1}(H_\bullet) = \bZ/m\bZ$. From this we conclude
\[
	\Ext^n_{\bZ}(\bZ/m\bZ,\,\bZ) \cong \begin{cases} \bZ/m\bZ & n = 1 \\ 0 & n \neq 1 \end{cases}
\]

\end{example}


By construction, the groups $\Ext^n(M,N)$ are zero for $n<0$. 

\begin{proposition}
The functors $\Ext^0_R(-,-)$ and $\Hom_R(-,-)$ are isomorphic.
\end{proposition}

\begin{proof}
Let
\[
	\cdots \longto F_1 \longto F_0 \longto M \longto 0
\]
be a free resolution of an $R$-module $M$. Then by Exercise \ref{exc:contravariant-short-exact-hom} we get
an exact sequence
\[
	0 \longto \Hom(M,N) \longto \Hom(F_0,N) \longto \Hom(F_1,N). 
\]
Using the definition of $\Ext^0$ we find
\[
	\Ext^0(M,N) = \rH_0( \Hom(F_\bullet,N))  = 
	\ker( \Hom(F_0,N) \longto \Hom(F_1,N) ),
\]
which gives an isomorphism $\Ext^0(M,N)=\Hom(M,N)$, functorial in $M$.
\end{proof}


We will see that the module $\Ext^1(M,N)$ is in bijection with isomorphism classes
of short exact sequences
\[
	0 \longto N \longto E \longto M \longto 0
\]
of $R$-modules. There is also an interpretation to the modules $\Ext^n(M,N)$ with $n>1$ in terms of exact sequences
\[
	 0 \longto N \longto E_1 \longto \cdots \longto E_n \longto M \longto 0,
\]
but the statement is more delicate. 


\section{The long exact sequence}

%\begin{proposition}\label{prop:short-exact-sequence-resolutions}
%Let $R$ be a ring and let
%\[
%	0 \longto M_1 \overset{\varphi}{\longto} M_2 \overset{\psi}{\longto} M_3 \longto 0
%\]
%be a short exact sequence of $R$-modules. Then there exist a commutative diagram
%\[
%\begin{tikzcd}
%	& 0  \arrow{d}
%	& 0 \arrow{d}
%	& 0 \arrow{d}
%	& 0 \arrow{d}
%	&  \\
%\cdots \arrow{r}
%	& F_{1,2} \arrow{r} \arrow{d}  
%	& F_{1,1} \arrow{r} \arrow{d}  
%	& F_{1,0} \arrow{r}{\pi_1} \arrow{d}  
%	& M_{1} \arrow{r} \arrow{d}{\varphi}
%	& 0 \\
%\cdots \arrow{r}
%	& F_{2,2} \arrow{r} \arrow{d}  
%	& F_{2,1} \arrow{r} \arrow{d}  
%	& F_{2,0} \arrow{r}{\pi_2} \arrow{d}  
%	& M_{2} \arrow{r} \arrow{d}{\psi}
%	& 0 \\
%\cdots \arrow{r}
%	& F_{3,2} \arrow{r} \arrow{d}  
%	& F_{3,1} \arrow{r} \arrow{d}  
%	& F_{3,0} \arrow{r}{\pi_3} \arrow{d}  
%	& M_{3} \arrow{r} \arrow{d}
%	& 0 \\
%	& 0 
%	& 0 
%	& 0 
%	& 0  
%	&  
%\end{tikzcd}
%\]
%with exact rows and columns, and with the $F_{i,j}$ free $R$-modules.
%\end{proposition}
%
%In other words: a short exact sequence of $R$-modules can be lifted to a short exact sequence of free resolutions.
%
%\begin{proof}[Proof of Proposition \ref{prop:short-exact-sequence-resolutions}]
%The diagram is constructed from the right to the left, following a variation on the construction of free resolutions in the proof of Proposition \ref{prop:free-resolutions-exist}. 
%
%Let $I$ be a generating set of $M_1$, which we identify with a submodule of $M_2$. Let $J\subset M_2\setminus M_1$ such that $I\cup J$ generates $M_2$. Then the image of $J$ in $M_3$ generates $M_3$. Setting $F_{1,0} = R^{(I)}$, $F_{2,0} = R^{(I\cup J)}$, and
%$F_{3,0} = R^{(J)}$ we obtain
%commutative diagram
%\[
%\begin{tikzcd}
%	& 0 \arrow{d}
%	& 0 \arrow{d}
%	&  \\
%\cdots \arrow{r}
%	& F_{1,0} \arrow{r}{\pi_1} \arrow{d}  
%	& M_{1} \arrow{r} \arrow{d}{\varphi}
%	& 0 \\
%\cdots \arrow{r}
%	& F_{2,0} \arrow{r}{\pi_2} \arrow{d}  
%	& M_{2} \arrow{r} \arrow{d}{\psi}
%	& 0 \\
%\cdots \arrow{r}
%	& F_{3,0} \arrow{r}{\pi_3} \arrow{d}  
%	& M_{3} \arrow{r} \arrow{d}
%	& 0 \\
%	& 0 
%	& 0  
%	&  
%\end{tikzcd}
%\]
%with exact rows and columns. By the snake lemma (Theorem \ref{thm:snake-lemma}), the induced sequence
%\[
%	0 \longto \ker \pi_1 \longto \ker \pi_2 \longto \ker \pi_3 \longto 0
%\]
%is exact, and using the above argument with the $M_i$ replaced by $\ker \pi_i$, we can extend the diagram to the left
%with free modules $F_{1,1}$, $F_{2,1}$, and $F_{3,1}$.
%Repeating the trick yields the desired diagram.
%\end{proof}

\begin{theorem}\label{thm:long-exact-sequence-of-Ext}
Let $N$ be an $R$-module, and let
\[
	0 \longto M_1 \longto M_2 \longto M_3 \longto 0
\]
be a short exact sequence of $R$-modules. Then there is a natural exact sequence
\[
\mkern-72mu\begin{tikzcd}[row sep=small]
	0 \arrow{r}
		& \Hom(M_3,N) \arrow{r}
		& \Hom(M_2,N) \arrow{r} 
		& \Hom(M_1,N)  \arrow[out=-5, in=175]{dll} \\
	& \Ext^1(M_3,N) \arrow{r}
		& \Ext^1(M_2,N) \arrow{r} 
		& \Ext^1(M_1,N)  \arrow[out=-5, in=175]{dll} \\
	& \Ext^2(M_3,N) \arrow{r}
		& \Ext^2(M_2,N) \arrow{r} 
		& \quad\cdots\quad
\end{tikzcd} 
\]
of abelian groups.
\end{theorem}

% TODO: more carefully phrase the supporting exercises in chapters 1/2 so they can be literally cited here

\begin{proof}
Choose free resolutions of the $M_i$ as in Exercise \ref{exc:short-exact-sequence-resolutions}, so that we have a short exact sequence of complexes
\[
	 0 \longto F_{1,\bullet} \longto F_{2,\bullet} \longto F_{3,\bullet} \longto 0.
\]
Denote by $H_{i,\bullet}$ the complex obtained by applying $\Hom(-,N)$ to $F_{i,\bullet}$, so we have
\[
	H_{i,j} = \Hom(F_{i,-j}, N).
\]
For every $i$ the exact sequence
\[
	0 \longto F_{1,i} \longto F_{2,i} \longto F_{3,i} \longto 0
\]
is split exact because $F_{3,i}$ is a free $R$-module (see Exercise \ref{exc:free-module-split-short-exact-sequence}), and hence also the induced sequence
\[
	0 \longto H_{3,-i} \longto H_{2,-i} \longto
	H_{1,-i} \longto 0
\]
is exact (see Exercise \ref{exc:hom-of-split-exact-seq}). Applying Theorem \ref{thm:long-exact-sequence}
to the short exact sequence of complexes of abelian groups
\[
	0 \longto H_{3,\bullet} \longto H_{2,\bullet} \longto H_{1,\bullet} \longto 0
\]
 we obtain a long exact sequence
 \[
\mkern-72mu\begin{tikzcd}
		& \cdots \arrow{r}
		& \rH_{i+1}(H_{2,\bullet}) \arrow{r} 
		& \rH_{i+1}(H_{1,\bullet})  \arrow[out=-5, in=175]{dll} \\
	&\rH_{i}(H_{3,\bullet})  \arrow{r}
		& \rH_{i}(H_{2,\bullet}) \arrow{r} 
		& \rH_{i}(H_{1,\bullet})  \arrow[out=-5, in=175]{dll} \\
	& \rH_{i-1}(H_{3,\bullet}) \arrow{r}
		& \rH_{i-1}(H_{2,\bullet}) \arrow{r} 
		& \quad\cdots\quad
\end{tikzcd} 
\]
which, taking into account the vanishing of $\Ext^{i}$ for $i<0$ and the fact that $\Ext^0=\Hom$, gives precisely the 
exact sequence of the theorem. One verifies that this sequence does not depend on the choice of free resolutions.
\end{proof}

\section{$\Ext^1$ and  extensions}

% TODO: get rid of the assumption that R is commutative
% just get Z(R)-Mod-valued functors
% maybe also: stress this when talking about Hom




\begin{definition}
Let $R$ be a ring and let $M$ and $N$ be $R$-modules. An \emph{extension of $M$ by $N$} is a short exact sequence
\[
	0 \longto N \longto E \longto M \longto 0
\]
of $R$-modules. Two such extensions are called \emph{equivalent} if there exists a commutative diagram
\[
\begin{tikzcd}
0 \arrow{r} & N \arrow{r} \arrow{d}{\id} & E \arrow{r} \arrow{d} & M \arrow{r} \arrow{d}{\id} & 0 \\
0  \arrow{r} & N \arrow {r} & E' \arrow{r} & M \arrow{r} & 0
\end{tikzcd}
\]
of $R$-modules. We define the set $\ext_R(M,N)$ to be the set of equivalence classes of extensions of $M$ by $N$.
\end{definition}

Note that the morphism $E\to E'$ in the above diagram is automatically an isomorphism, see Exercise \ref{exc:morphism-of-extensions-is-isomorphism}. Warning: there can be non-equivalent extensions with $E\cong E'$, see Exercise \ref{exc:extensions-Z-mod-n}.


We will now construct a map
\[
	\theta\colon \ext_R(M,N) \to \Ext^1_R(M,N)
\]
and show that it is a bijection. To define the map, consider an element $e\in \ext_R(M,N)$, represented by a 
short exact sequence
\[
	0 \longto N \longto E \longto M \longto 0.
\]
By Theorem \ref{thm:long-exact-sequence-of-Ext} this induces a long exact sequence, of which a part reads
\[
	\cdots \longto \Hom_R(N,\, N) \longto \Ext^1_R(M,\,N) \longto \cdots
\]
We define $\theta(e) \in \Ext^1_R(M,\,N)$ to be the image of $\id_N$ under the above map. 

\begin{theorem}\label{thm:main-thm-ext}
The map
\[
	\theta\colon \ext_R(M,N) \to \Ext^1_R(M,N)
\]
is a bijection.
\end{theorem}

\begin{proof}
We construct map 
\[
	\psi\colon \Ext^1_R(M,N) \to \ext_R(M,N)
\]
but omit the tedious verification that it is a two-sided inverse.

Choose a free module $F$ and a surjection $F\to M$. Let $K$ be the kernel. Then we have a short exact sequence
\[
	0 \longto K \overset{\gamma}{\longto} F \longto M \longto 0.
\]
The induced long exact sequence of Theorem \ref{thm:long-exact-sequence-of-Ext} contains
\[
	\Hom_R(F,N) \longto \Hom_R(K,N) \overset{\delta}{\longto} \Ext^1_R(M,N) \longto \Ext^1_R(F,N).
\]
Since $F$ is a free module, it has a `trivial' one-term free resolution, and one sees that $\Ext^i_R(F,N)=0$ for all $i>0$. It follows that the map $\delta$ is surjective.

Let $e \in \Ext^1_R(M,N)$. Choose an $R$-linear map $f\colon K \to N$ with $\delta(f)=e$.
In the commutative diagram
\[
\begin{tikzcd}
& 0 \arrow{r} \arrow{d} 
	& K \arrow{r}{\id} \arrow{d}{(f,\gamma)} & K \arrow{r} \arrow{d}{\gamma} & 0 \\
0 \arrow{r} & N \arrow{r}{\iota_1} & N \oplus F \arrow{r}{\pi_2} & F \arrow{r} & 0
\end{tikzcd}
\]
both rows are exact, and all the vertical maps are injective (since $\gamma$ is injective). It follows that there is an induced short exact sequence of cokernels
\[
	 0 \longto N \longto E \longto M \longto 0
\]
with $E=\coker(K \to N\oplus F)$. We define $\psi(e)\in \ext_R(M,N)$ to be this extension. One verifies that this is well-defined: if also $f'\colon K\to N$ satisfies $\delta(f)=e$, leading to an extension $E'$, then $f'=f+h$ for some linear map $h\colon F\to N$, and one shows that the isomorphism
\[
	F\oplus N \to F\oplus N,\, (x,y) \mapsto (x,y+hx)
\]
induces an isomorphism $E\to E'$ of extensions.
\end{proof}



\newpage
\section*{Exercises}

\begin{exercise}
Let $R$ be a ring, $F$ a free $R$-module, and $N$ an $R$-module. Show that $\Ext^i_R(F,N)=0$ for all $i\neq 0$.
\end{exercise}

\begin{exercise}
Let $K$ be a field. Let $M$ and $N$ be $K$-modules. Show that $\Ext^i_K(M,N)=0$ for all $i\neq 0$.
\end{exercise}



\begin{exercise}
Let $n$ and $m$ be positive integers. Compute for all $i$ the $\bZ$-modules
\begin{enumerate}
\item $\Ext^i_\bZ(\bZ,\bZ/m\bZ)$;
\item $\Ext^i_\bZ(\bZ/n\bZ,\bZ)$;
\item $\Ext^i_\bZ(\bZ/n\bZ,\bZ/m\bZ)$.
\end{enumerate}
\end{exercise}

\begin{exercise}
Let $K$ be a field.
Consider the ring $R=K[X^2,X^3] \subset K[X]$. Let $I=(X^2,X^3)$. Show that $\Ext^1_R(I,R/I)$ is non-zero, and conclude that $I$ is not a principal ideal. Show that also $\Ext^2_R(R/I,R/I)$ is non-zero.
\end{exercise}

\begin{exercise}
Let $K$ be a field. Consider the ring $R=K[X,Y]$ and the $R$-module $M=K[X,Y]/(X,Y)$. Compute $\Ext^2_R(M,R)$. Conclude that $M$ does not have a free resolution of the form $0\to F_1 \to F_0 \to M \to 0$.
\end{exercise}


\begin{exercise}
Let $R$ be a principal ideal domain. Show that for every $i\geq 2$, for every finitely generated $R$-module $M$,
and for every $R$-module $N$ we have $\Ext^i_R(M,N)=0$.
\end{exercise}


\begin{exercise}Let $R$ be a ring, $N$ an $R$-module and $n\geq 1$. Assume that $\Ext^n_R(M,N)=0$ for all $R$-modules $M$. Show that also $\Ext^{n+1}_R(M,N)=0$ for all $R$-modules $M$. (Hint: consider a short exact sequence of the form $0 \to K \to F \to M \to 0$ with $F$ a free module). 
\end{exercise}


\begin{exercise}
Let 
\[
	0 \longto N \longto E \longto M \longto 0
\]
be a short exact sequence of $R$-modules. Let $\varphi\colon N\to N'$ be a morphism of $R$-modules, and let $E'$ 
be the pushout of $N\to E$ and $\varphi\colon N\to N'$. Show that there is a short exact sequence
\[
	0 \longto N' \longto E' \longto M \longto 0
\]
of $R$-modules. (Hint, see Exercise \ref{exc:module-pushout}).
\end{exercise}


\begin{exercise}
Let 
\[
	0 \longto N \longto E \longto M \longto 0
\]
be a short exact sequence of $R$-modules. Let $\varphi\colon M'\to M$ be a morphism of $R$-modules, and let $E'$ 
be the pullback of $E\to M$ and $\varphi\colon M'\to M$. Show that there is a short exact sequence
\[
	0 \longto N \longto E' \longto M' \longto 0
\]
of $R$-modules. (Hint, see Exercise \ref{exc:module-pushout}).
\end{exercise}


\begin{exercise}
Let $R$ be a ring. Show directly (without using the relation with $\Ext^1$) that
\[
	{}_R\Mod^\opp \times {}_R\Mod \to \Set,\,
	(M,N) \mapsto \ext_R(M,N)
\]
is a functor. (In particular: explain what the functor does on the level of morphisms).
\end{exercise}

\begin{exercise}\label{exc:extensions-Z-mod-n}
Describe explicitly the $n$ elements of the set $\ext_\bZ(\bZ/n\bZ,\bZ)$ for:
\begin{enumerate}
\item $n$ a prime number;
\item $n=pq$ with $p$ and $q$ distinct primes;
\item $n=4$.
\end{enumerate}
\end{exercise}

\begin{exercise}For $i=1,2$ let
\[
	 0 \longto N \overset{\alpha_i}\longto E_i \overset{\beta_i}\longto M \longto 0
\]
be a short exact sequence of $R$-modules. Define an $R$-module $E$ as the quotient
\[
	E := \frac{\ker \big( \beta_1 - \beta_2 \colon E_1 \oplus E_2 \to M \big)}
	{\im \big( (\alpha_1,\alpha_2) \colon N \to E_1 \oplus E_2 \big)}.
\]
Show that there is a natural short exact sequence
\[
	0 \longto N \longto E \longto M \longto 0
\] 
of $R$-modules.
\end{exercise}

\begin{exercise}
Let $K$ be a field and let $\lambda_1,\lambda_2\in K$. Let $M_i:= K[X]/(X-\lambda_i)$. 
Show that $\Ext^1_{K[X]}(M_2,M_1) =0$ if $\lambda_1\neq \lambda_2$. Compute 
$\Ext^1_{K[X]}(M_2,M_1)$ when $\lambda_2=\lambda_1$.
\end{exercise}

\begin{exercise}
Let $K$ be a field, $\lambda_1,\lambda_2,\mu \in K$. Consider the matrix
\[
	A := \begin{pmatrix} \lambda_1 & \mu \\ 0 & \lambda_2 \end{pmatrix}
\]
and let $E$ be the $K[X]$-module given by $E=K^2$ on which $X$ acts by $A$. Show that
$E$ sits in a short exact sequence of $K[X]$-modules
\[
	0 \longto M_1 \longto E \longto M_2 \longto 0
\]
where $M_i=K$ with $X$ acting as $\lambda_i$. Show that this sequence splits if and only if
 $\lambda_2\neq \lambda_1$ or $\mu=0$. Relate this to the previous exercise.
\end{exercise}


\begin{exercise}
Let $n$ be a positive integer, and consider the ring $R := \bZ[X]/(X^n-1)$. Let $M$ be the quotient module $R/(X-1)$. Compute the $R$-modules $\Ext^i_R(M,M)$ and $\Ext^i_R(M,R)$. (Hint: use the free resolution from Exercise \ref{exc:free-resolution-finite-cyclic-group}).
\end{exercise}


%%%%%%%%%%%
% BIBLIOGRAPHY %
%%%%%%%%%%%

\begin{thebibliography}{100}

\bibitem{AtiyahMacDonald}{\sc Atiyah, M.F. \& MacDonald, I.G.}
	{\it Introduction To Commutative Algebra},
	Westview Press, 1969.

\bibitem{Lang}{\sc Lang, S.} ---
	{\it Algebra},
	Graduate Texts in Mathematics~211, Springer, 2002.
	
\bibitem{Moerdijk}{\sc Moerdijk, I.} ---
	{\it Notes on Homological Algebra},
	Lecture notes, 2008.

\bibitem{Stevenhagen}{\sc Stevenhagen, P.} ---
	{\it Algebra 2}, {\url{http://websites.math.leidenuniv.nl/algebra/algebra2.pdf}}.

\bibitem{Leinster}{\sc Leinster, T.} ---
	{\it Basic Category Theory},
	Cambridge Studies in Advanced Mathematics,  Cambridge University Press, 2014.
	
\end{thebibliography}

\printindex
\end{document}
